% Options for packages loaded elsewhere
\PassOptionsToPackage{unicode}{hyperref}
\PassOptionsToPackage{hyphens}{url}
%
\documentclass[
  openany]{book}
\usepackage{lmodern}
\usepackage{amssymb,amsmath}
\usepackage{ifxetex,ifluatex}
\ifnum 0\ifxetex 1\fi\ifluatex 1\fi=0 % if pdftex
  \usepackage[T1]{fontenc}
  \usepackage[utf8]{inputenc}
  \usepackage{textcomp} % provide euro and other symbols
\else % if luatex or xetex
  \usepackage{unicode-math}
  \defaultfontfeatures{Scale=MatchLowercase}
  \defaultfontfeatures[\rmfamily]{Ligatures=TeX,Scale=1}
\fi
% Use upquote if available, for straight quotes in verbatim environments
\IfFileExists{upquote.sty}{\usepackage{upquote}}{}
\IfFileExists{microtype.sty}{% use microtype if available
  \usepackage[]{microtype}
  \UseMicrotypeSet[protrusion]{basicmath} % disable protrusion for tt fonts
}{}
\makeatletter
\@ifundefined{KOMAClassName}{% if non-KOMA class
  \IfFileExists{parskip.sty}{%
    \usepackage{parskip}
  }{% else
    \setlength{\parindent}{0pt}
    \setlength{\parskip}{6pt plus 2pt minus 1pt}}
}{% if KOMA class
  \KOMAoptions{parskip=half}}
\makeatother
\usepackage{xcolor}
\IfFileExists{xurl.sty}{\usepackage{xurl}}{} % add URL line breaks if available
\IfFileExists{bookmark.sty}{\usepackage{bookmark}}{\usepackage{hyperref}}
\hypersetup{
  pdftitle={Comprehensive Crap Guide},
  pdfauthor={Deependra Dhakal, Samita Paudel},
  hidelinks,
  pdfcreator={LaTeX via pandoc}}
\urlstyle{same} % disable monospaced font for URLs
\usepackage{longtable,booktabs}
% Correct order of tables after \paragraph or \subparagraph
\usepackage{etoolbox}
\makeatletter
\patchcmd\longtable{\par}{\if@noskipsec\mbox{}\fi\par}{}{}
\makeatother
% Allow footnotes in longtable head/foot
\IfFileExists{footnotehyper.sty}{\usepackage{footnotehyper}}{\usepackage{footnote}}
\makesavenoteenv{longtable}
\usepackage{graphicx}
\makeatletter
\def\maxwidth{\ifdim\Gin@nat@width>\linewidth\linewidth\else\Gin@nat@width\fi}
\def\maxheight{\ifdim\Gin@nat@height>\textheight\textheight\else\Gin@nat@height\fi}
\makeatother
% Scale images if necessary, so that they will not overflow the page
% margins by default, and it is still possible to overwrite the defaults
% using explicit options in \includegraphics[width, height, ...]{}
\setkeys{Gin}{width=\maxwidth,height=\maxheight,keepaspectratio}
% Set default figure placement to htbp
\makeatletter
\def\fps@figure{htbp}
\makeatother
\setlength{\emergencystretch}{3em} % prevent overfull lines
\providecommand{\tightlist}{%
  \setlength{\itemsep}{0pt}\setlength{\parskip}{0pt}}
\setcounter{secnumdepth}{5}
\usepackage{booktabs}

\usepackage{geometry} % for custom layout of book and landscape geometry support
\usepackage{amsthm}
\makeatletter
\def\thm@space@setup{%
  \thm@preskip=8pt plus 2pt minus 4pt
  \thm@postskip=\thm@preskip
}
\makeatother

\usepackage{longtable}
\usepackage{booktabs}
\usepackage{dcolumn}
\usepackage{tabularx}
\usepackage{array}
\usepackage{multirow}
% \usepackage[table]{xcolor}
\usepackage{wrapfig}
\usepackage{float}
\usepackage{colortbl}
\usepackage{pdflscape}
\usepackage{tabu}
\usepackage{threeparttable}
\usepackage[normalem]{ulem}
\usepackage{rotating}
\newcommand{\blandscape}{\begin{landscape}}
\newcommand{\elandscape}{\end{landscape}}
\usepackage[format=hang,labelfont=bf,margin=0.5cm,justification=centering]{caption}
% \usepackage{exam} % this and exam.sty cause error
\usepackage{pdfpages}
\usepackage{siunitx}

% \usepackage{subcaption} % doesn't work with tinytex in windows
% \newcommand{\subfloat}[2][need a sub-caption]{\subcaptionbox{#1}{#2}}

% simple fix for exam class features of questions and solution
\usepackage{enumitem}

\newlist{questions}{enumerate}{3}
\setlist[questions]{label=\arabic*.}
\newcommand{\question}{\item}

\newenvironment{solution}{ {\bfseries Solution}:}{}
\usepackage{fancyhdr}
\usepackage[]{natbib}
\bibliographystyle{apalike}

\title{Comprehensive Crap Guide}
\usepackage{etoolbox}
\makeatletter
\providecommand{\subtitle}[1]{% add subtitle to \maketitle
  \apptocmd{\@title}{\par {\large #1 \par}}{}{}
}
\makeatother
\subtitle{Foreign Affairs}
\author{Deependra Dhakal, Samita Paudel}
\date{November, 2019}

\begin{document}
\maketitle

{
\setcounter{tocdepth}{1}
\tableofcontents
}
\hypertarget{introduction-to-international-relations-and-diplomacy}{%
\chapter{Introduction to International Relations and Diplomacy}\label{introduction-to-international-relations-and-diplomacy}}

\hypertarget{diplomacy}{%
\section{Diplomacy}\label{diplomacy}}

Diplomacy is the application of intelligence and tact to the conduct of official relations between the government of independent states, extending sometimes also to their relations with dependent territories, and between govenrment and international institutions; or more briefly, the conduct of business between states by peaceful means.

While diplomacy is properly the conduct or execution of foreign policy, it is sometimes confused with foreign policy itself. But foreign policy is formulated by government, not by diplomats. In order to carry out its policy, a government manages its international relations by applying not only persuasion but also different forms of pressure. How successful these pressures prove will depend to a great extent on the real power, often now referred to as \textbf{hard power}, behind them. The power must be real, but rather than exercise it explicitly, the government may prefer to keep it in reserve with the implication that in certain circumstances it could be used. Nevertheless, in normal circumstances it will conduct its international intercourse by negotiation, a form of \textbf{soft power}. This is diplomacy. Persuasive argument, if applied skilfully and sensitively at the right time, may achieve a better result than persuasion too obviously backed by the threat of force. The latter may provoke resistance and ultimately lead to war.

The etymology of diplomacy takes us to ancient Greece. The diploma is understood to be a document by which a privilege is conferred: a state paper, official document, a charter. The earliest English instance of the use of this word is in the year 1645.

Diplomacy is in fact, as the Duc de Broglie remarked, the best means devised by civilization for preventing international relations from being governed by force alone. The field in which it operates lies somewhere between power politics and civilized usage, and its methods have varied with the political conventions of each age.

\hypertarget{foreign-policy}{%
\section{Foreign policy}\label{foreign-policy}}

Foreign policy is the system of activities evolved by communities for changing the behavior of other states and for adjusting their own activities to the international environment -- George Modelski

Foreign policy is the substance of nation's efforts to promote its interest's vis-a-vis other nations -- Normal Hill

Foreign policy is the key element in the process by which a state translates its broadly conceived goals and interests into concrete courses of action and to attain these objectives and preserve its interests -- Padelford and Loncoln.

Foreign policy is based upon general conception of national requirements. A country's foreign policy is a set of goals that seek to outline how that particular country with interact with other countires of world and to a less extent non-state actors.

Forein policies are designed to help and protect a country's national interests, national security, ideological goals and economic prosperity

Foreign policy is an extension of domestic policy so one must recognize the fact that no nation state can preserve its independence, soverignity and territorial integrity without a policy to build its internal strength that comes from peace and prosperity enjoyed by the people.

Foreign policy of state is the ratifed legal policy aimed at establishing good mutual relationship with other states and international organizations.

Foreign policy of each nation contains:

\begin{itemize}
\tightlist
\item
  A set of principles, policies and decisions adopted and followed by the nation in international relations.
\item
  Objectives, goals or aims of national interest which are to be secured.
\item
  Means to be used for achieving the goals of national interest.
\item
  Broad policy principles and decisions for conducting international relations.
\item
  Assessement of the gains and failures of the nation in respect of its goals of national interest.
\item
  Policies, decisions and action-programmes for maintaining continuity or change or both in international relations.
\end{itemize}

A student of foreign policy analyses the action of a state towards external environment (i.e.~other states) and the conditions, usually domestic, under which those actions are formulated. -- Hlstei

The study of foreign policy includes both the study of national objectives to be achived and the means used for securing those. -- Ceeil V. Crabb

\hypertarget{foreign-policy-of-nepal}{%
\subsection{Foreign policy of Nepal}\label{foreign-policy-of-nepal}}

\begin{questions}

\question What is foreign policy ? Write about foreign policy of Nepal.

\begin{solution}

A country's foreign policy, also called foreign relations or foreign affairs policy, consists of self-interest strategies chosen by the state to safeguard its national interests and to achieve goals within its international relations milieu. The approaches are strategically employed to interact with other countries. In recent decades, due to the deepening level of globalization and transnational activities, states also must interact with non-state actors. These interactions are evaluated and monitored in seeking the benefits of bilateral and multilateral international cooperation. 

Foreign policies of countries have varying rates of change and scopes of intent, which can be affected by factors that change the perceived national interests or even affect the stability of the country itself. The foreign policy of a country can have a profound and lasting impact on other countries and on the course of international relations as a whole, such as the Monroe Doctrine, which began in December 1823, conflicting with the mercantilism policies of 19th-century European countries and the goals of independence of newly formed Central American and South American countries.

The fundamental objective of Nepal's foreign policy is to enhance the dignity of the nation by safeguarding sovereignty, territorial integrity, independence, and promoting economic wellbeing and prosperity of Nepal. It is also aimed at contributing to global peace, harmony and security.

Nepal's foreign policy is guided by the following basic principles:

\begin{enumerate}
\item Mutual respect for each other’s territorial integrity and sovereignty;
\item Non-interference in each other’s internal affairs;
\item Respect for mutual equality;
\item Non-aggression and the peaceful settlement of disputes;
\item Cooperation for mutual benefit;
\item Abiding faith in the Charter of the United Nations;
\item Value of world peace.
\end{enumerate}

Provisions of the Constitution of Nepal on National Interest and Foreign Policy

National Interest (Article 5.1)

Safeguarding of the freedom, sovereignty, territorial integrity, nationality, independence and dignity of Nepal, the rights of the Nepalese people, border security, economic wellbeing and prosperity shall be the basic elements of the national interest of Nepal.

Directive Principles (Article 50.4)

The State shall direct its international relations towards enhancing the dignity of the nation in the world community by maintaining international relations on the basis of sovereign equality, while safeguarding the freedom, sovereignty, territorial integrity and independence and national interest of Nepal.

State Policy (Article 51)

\begin{enumerate}
\item To conduct an independent foreign policy based on the Charter of the United Nations, non-alignment, principles of Panchasheel, international law and the norms of world peace, taking into consideration of the overall interest of the nation, while remaining active in safeguarding the sovereignty, territorial integrity, independence and national interest of Nepal,
\item To review treaties concluded in the past, and make treaties, agreements based on equality and mutual interest.
\end{enumerate}

Put simply, foreign policy is an activity whereby state actors act, react and interact in their bilateral and multilateral dealings with external actors. Its architects straddle two separate environments-internal or domestic environment and external or global environment. Internal political dynamics forms the background context against which a policy is crafted. Factors such as the state's resource base, its geographical position in relation to others, the level of development of its economy, its demographic structure and the political ideology it pursues form the domestic milieu. The international environment is where a nation's foreign policy is actually implemented. Regional perspectives, too, play an equally vital role in approaching the international environment because geopolitics sets the parameters in all aspects of foreign policy. 

Another important consideration is that political regimes may change at regular intervals, but a country’s independence, sovereignty and territorial integrity cannot be compromised at any cost. Nepal needs to maintain a judicious balance in its foreign policy posture vis-à-vis the neighbours and donor countries who practically wield a vital key to Nepal’s economic reconstruction and infrastructural development.

source: \url{https://mofa.gov.np/foreign-policy/}; \url{https://kathmandupost.com/opinion/2019/03/26/foreign-policy-dilemma}
\end{solution}
\end{questions}

\hypertarget{elements-of-foreign-policy}{%
\subsection{Elements of foreign policy}\label{elements-of-foreign-policy}}

\begin{enumerate}
\def\labelenumi{\arabic{enumi}.}
\tightlist
\item
  Size of state territory
\item
  Geographical factors
\item
  Level and nature of economic development
\item
  Cultural and historical factors
\item
  Social structure
\item
  Governement structure
\item
  Internal situation
\item
  Values, talents, experiences and personalities of leaders
\item
  POlitical accountability
\item
  Ideology
\item
  Diplomacy
\item
  International power structure (Global strategic environment)
\item
  Public opinion
\item
  Technology
\item
  External environment
\item
  Alliances and international treaties (Bilateral and multilateral)
\end{enumerate}

Domestic factors

\begin{enumerate}
\def\labelenumi{\arabic{enumi}.}
\tightlist
\item
  Size and population composition
\item
  Geography
\item
  Culture and history
\item
  Economic development
\item
  Technology
\item
  National capacity
\item
  Public mood and opinion
\item
  Political organization
\item
  Role of press
\end{enumerate}

External determinants

\begin{enumerate}
\def\labelenumi{\arabic{enumi}.}
\tightlist
\item
  International law
\item
  International organizations
\end{enumerate}

\hypertarget{foreign-policy-2077}{%
\section{Foreign policy, 2077}\label{foreign-policy-2077}}

\hypertarget{challenges}{%
\subsection{Challenges}\label{challenges}}

\begin{enumerate}
\def\labelenumi{\arabic{enumi}.}
\tightlist
\item
  Protect nation and promote national interest with the mindset of continuation and dynamism in a rapidly changing and volatile regional and international context.
\item
  Protect international border of Nepal
\item
  Construct an integrated vision for conduction of foreign affairs with effective consultation, collaboration and help from all possible sectors.
\item
  Accept and mobilize foreign aid taking into consideration the national leadership and ownership in ensuring promotion of national priorities and interest.
\item
  Ensure international assistance in achieving sustainable development periodic planning goals.
\item
  Mobilize and foreign investment based on national requirement for socio-economic development through effective implementation of economic diplomacy.
\item
  Protect sovereignty and national interest ill impacts of competitive warfare among regional and international superpowers.
\item
  Improvement of multilateralism in order to protect welfare of small and developing nations.
\item
  Generate consensus and assistance from international community to promote welfare on behalf of the countries with special situation.
\item
  Increase competitiveness and continue to provide trade related facilities even after graduating from least developed country.
\item
  Incorporate agendas of mountainous country into international climate change policy formulation and seek for financial and technical assistance for mitigation as well as resilience building against climate change.
\item
  Develop specialized institutional capacity, human resource and professional expertise in the dynamic and competitive global environment for conduction and management of foreign affairs.
\end{enumerate}

\hypertarget{some-terminologies}{%
\section{Some terminologies}\label{some-terminologies}}

\hypertarget{diplomatic-immunity}{%
\subsection{Diplomatic immunity}\label{diplomatic-immunity}}

Diplomatic immunity is a kind of legal immunity and a policy held between governments that ensure that diplomats are given safe passage and are considered not susceptible to lawsuit or prosecution under the host country's laws, but they can still be expelled.

Diplomatic Immunity finds its origin from as international law in the Vienna Convention on Diplomatic Relations (1961), though the concept and custom have a much longer history. Diplomatic immunity as an institution developed to allow for

\hypertarget{diplomatic-asylum}{%
\subsection{Diplomatic asylum}\label{diplomatic-asylum}}

Diplomatic asylum is not established in any international law. It derives its existence from Article 14 of the Universal Declaration of Human Rights, which states: ``Everyone has the right to seek and to enjoy in other countries asylum from persecution.''

The European Convention on Human Rights and the International Covenant on Civil and Political Rights also enshrine this law.

The International Court of Justice has emphasised that in the absence of treaty or customary rules to the contrary, a decision by a mission to grant asylum involves a derogation from the soverignty of the receiving state.

The Organization of American States agreed a convention in 1954.

In a broad sense, according to the UN, it is protection which is granted by a country outside its own borders, and particularly through its diplomatic missions.

A case of Julian Assange

Julian Assange, founder of whistleblowing website Wikileaks, is facing extradition from the US to Sweden over rape and sexual assault allegations. Recently he had spent a night in Ecuadorian embassy in London after claiming diplomatic asylum. The Government authorities said that by spending the night at the embassy he has breached his bail conditions and faces arrest, but Ecuadorian authorities said they were ``studying and analysing'' his request.

As per the customs, local police and security forces are not permitted to enter an embassy unless they have the express permission of the ambassador -- even though the embassy remains the territory of the host nation. This rule was set our in 1961 Vienna Convention on Diplomatic Relations when it codified a custom in place for centuries by establishing the ``rule of inviolability''. Thus, by being at the embassy, Assange was on diplomatic territory and beyond the reach of the police. Assange fears if he is sent to Sweden it may then lead to him being sent to the US to face charges over Wikileaks, for which he could face the death penalty.

\hypertarget{soft-power-and-hard-power}{%
\subsection{Soft power and hard power}\label{soft-power-and-hard-power}}

\textbf{Soft power} is the ability of one country to attract and co-opt, rather than coerce, the other country in order to achieve what it wants. It involves shaping the preferences of others through appeal and attraction. The elements of soft power includes culture, political values, mass media, and foreign policies. The success of soft power heavilty depends on the actor's reputation within the international community, as well as the flow of information between actors.

A country may obtain the outcomes it wants in world politics because other countries -- admiring its values, emulating its examples, aspiring to its level of prosperity and openness -- want to follow it. In this sense, it is also importantn to set the agenda and attract others in world politics, and not only to force tham to change by threatening military force or economic sanctions.

In the current context of pandemic outbreak, scientific innovations such as Covid-19 vaccines have been a part of a country's soft power\footnote{\url{https://tkpo.st/38xwIDB}}. Small countries like Nepal need to pursue proactive diplomacy to ensure that their vaccines needs are fulfilled on time.

\textbf{Hard power} refers to the use of military and economic means to influence the behavior or interests of other political bodies. This form of political power is often aggressive (using coercion), and it is most immediately effective when imposed by a country/political body as a means to obtain the outcome of it's interest upon a lesser military and/or economic power.

According to Joseph Nye, hard power involves ``the ability to use the carrots and sticks of economic and military might to make others follow your will''. Here, ``carrots'' stand for inducements such as reduction of trade barriers, the offer of an alliance or the promise of military protection. On the other hand, ``sticks'' represent threats-including the use of coercive diplomacy, the threat of military intervention, or the implementation of economic sanctions.

Recent event that involves a brandishing of hard-power involves, India imposing an undeclared blockade against Nepal. The events unfolded starting September 2015 owing to India's concern about changes to Nepal's constitution, violent ethnic conflict, and Nepal's increasing cooperation with China.

\hypertarget{ambassador-extraordinary-and-plenipotentiary}{%
\subsection{Ambassador Extraordinary and Plenipotentiary}\label{ambassador-extraordinary-and-plenipotentiary}}

According to the Vienna Convention on Diplomatic Relations of 1961, ambassadors are diplomats of the highest rank, formally representing their head of state with plenipotentiary powers (i.e.~full authority to represent the government). In modern usage, most ambassadors on foreign postings as head of mission carry the full title of Ambassador Extraordinary and Plenipotentiary.

The distinction between extraordinary and ordinary ambassadors was common when not all ambassadors resided in the country to which they are assigned, often service only for a specific purpose or mission.

\hypertarget{charges-d-affaires}{%
\subsection{Charges d' affaires}\label{charges-d-affaires}}

A distinction must be drawn between chargés d'affaires accredited to ministers of foreign affairs, forming the third class of heads of mission under Article 14 of the Vienna Convention on Diplomatic Relations, and chargés d'affaires ad interim who are appointed to act provisionally as head of a mission. The former were sometimes known as chargés d'affaires en pied or as chargés d'affaires en titre and, as indicated above, they have almost vanished from diplomatic practice as the appointment of ambassadors has become entirely general. An exceptional example of this generally obsolete practice however took place in the Federal Republic of Yugoslavia in 1992 where as a sign of displeasure at that State's role in the break-up of Yugoslavia, many States withdrew their ambassadors from Belgrade and replaced them with chargés d'affaires en titre.

Chargés d'affaires ad interim, who are not formally accredited either to heads of State or to ministers of foreign affaires, are by contrast frequent appointments. Article 19 of the Vienna Convention specifies that they should be appointed when the post of head of mission is vacant or the head is unable to perform his functions. It is usual diplomatic practice for an ambassador to take his leave and for there to be a gap before the arrival of his successor during which a chargé d'affaires ad interim will act as head of mission, and a chargé may also be appointed when the ambassador is recalled home for consultations or is abroad on leave, seriously ill, or even held hostage.

The appointment of a chargé d'affaires ad interim must be notified to the ministry of foreign affairs---usually by the departing head of mission but, if he is unable to do so, by the ministry of foreign affairs of the sending State.

\hypertarget{diplomatic-stance-of-nepal}{%
\section{Diplomatic stance of Nepal}\label{diplomatic-stance-of-nepal}}

\hypertarget{history-and-evolution-of-non-aligned-movement}{%
\subsection{History and Evolution of Non-Aligned Movement}\label{history-and-evolution-of-non-aligned-movement}}

August 22, 2012

History

The Non-Aligned Movement (NAM) was created and founded during the collapse of the colonial system and the independence struggles of the peoples of Africa, Asia, Latin America and other regions of the world and at the height of the Cold War. During the early days of the Movement, its actions were a key factor in the decolonization process, which led later to the attainment of freedom and independence by many countries and peoples and to the founding of tens of new sovereign States. Throughout its history, the Movement of Non-Aligned Countries has played a fundamental role in the preservation of world peace and security.

While some meetings with a third-world perspective were held before 1955, historians consider that the Bandung Asian-African Conference is the most immediate antecedent to the creation of the Non-Aligned Movement. This Conference was held in Bandung on April 18-24, 1955 and gathered 29 Heads of States belonging to the first post-colonial generation of leaders from the two continents with the aim of identifying and assessing world issues at the time and pursuing out joint policies in international relations.

The principles that would govern relations among large and small nations, known as the ``Ten Principles of Bandung'', were proclaimed at that Conference. Such principles were adopted later as the main goals and objectives of the policy of non-alignment. The fulfillment of those principles became the essential criterion for Non-Aligned Movement membership; it is what was known as the ``quintessence of the Movement'' until the early 1990s.

In 1960, in the light of the results achieved in Bandung, the creation of the Movement of Non-Aligned Countries was given a decisive boost during the Fifteenth Ordinary Session of the United Nations General Assembly, during which 17 new African and Asian countries were admitted. A key role was played in this process by the then Heads of State and Government Gamal Abdel Nasser of Egypt, Kwame Nkrumah of Ghana, Shri Jawaharlal Nehru of India, Ahmed Sukarno of Indonesia and Josip Broz Tito of Yugoslavia, who later became the founding fathers of the movement and its emblematic leaders.

Six years after Bandung, the Movement of Non-Aligned Countries was founded on a wider geographical basis at the First Summit Conference of Belgrade, which was held on September 1-6, 1961. The Conference was attended by 25 countries: Afghanistan, Algeria, Yemen, Myanmar, Cambodia, Srilanka, Congo, Cuba, Cyprus, Egypt, Ethiopia, Ghana, Guinea, India, Indonesia, Iraq, Lebanon, Mali, Morocco, Nepal, Saudi Arabia, Somalia, Sudan, Syria, Tunisia, Yugoslavia.

The Founders of NAM have preferred to declare it as a movement but not an organization in order to avoid bureaucratic implications of the latter.

The membership criteria formulated during the Preparatory Conference to the Belgrade Summit (Cairo, 1961) show that the Movement was not conceived to play a passive role in international politics but to formulate its own positions in an independent manner so as to reflect the interests of its members.

Thus, the primary of objectives of the non-aligned countries focused on the support of self-determination, national independence and the sovereignty and territorial integrity of States; opposition to apartheid; non-adherence to multilateral military pacts and the independence of non-aligned countries from great power or block influences and rivalries; the struggle against imperialism in all its forms and manifestations; the struggle against colonialism, neocolonialism, racism, foreign occupation and domination; disarmament; non-interference into the internal affairs of States and peaceful coexistence among all nations; rejection of the use or threat of use of force in international relations; the strengthening of the United Nations; the democratization of international relations; socioeconomic development and the restructuring of the international economic system; as well as international cooperation on an equal footing.

Since its inception, the Movement of Non-Aligned Countries has waged a ceaseless battle to ensure that peoples being oppressed by foreign occupation and domination can exercise their inalienable right to self-determination and independence.

During the 1970s and 1980s, the Movement of Non-Aligned Countries played a key role in the struggle for the establishment of a new international economic order that allowed all the peoples of the world to make use of their wealth and natural resources and provided a wide platform for a fundamental change in international economic relations and the economic emancipation of the countries of the South.

During its nearly 50 years of existence, the Movement of Non-Aligned Countries has gathered a growing number of States and liberation movements which, in spite of their ideological, political, economic, social and cultural diversity, have accepted its founding principles and primary objectives and shown their readiness to realize them. Historically, the non-aligned countries have shown their ability to overcome their differences and found a common ground for action that leads to mutual cooperation and the upholding of their shared values.

The ten principles of Bandung

\begin{itemize}
\tightlist
\item
  Respect of fundamental human rights and of the objectives and principles of the Charter of the United Nations.
\item
  Respect of the sovereignty and territorial integrity of all nations.
\item
  Recognition of the equality among all races and of the equality among all nations, both large and small.
\item
  Non-intervention or non-interference into the internal affairs of another -country.
\item
  Respect of the right of every nation to defend itself, either individually or collectively, in conformity with the Charter of the United Nations.
\end{itemize}

A. Non-use of collective defense pacts to benefit the specific interests of any of the great powers.

B. Non-use of pressures by any country against other countries.

\begin{itemize}
\tightlist
\item
  Refraining from carrying out or threatening to carry out aggression, or from using force against the territorial integrity or political independence of any country.
\item
  Peaceful solution of all international conflicts in conformity with the Charter of the United Nations.
\item
  Promotion of mutual interests and of cooperation.
\item
  Respect of justice and of international obligations.
\item
  Evolution
\end{itemize}

The creation and strengthening of the socialist block after the defeat of fascism in World War II, the collapse of colonial empires, the emergence of a bipolar world and the formation of two military blocks (NATO and the Warsaw Pact) brought about a new international context that led to the necessity of multilateral coordination fora between the countries of the South .

In this context, the underdeveloped countries, most of them in Asia and Africa, felt the need to join efforts for the common defense of their interests, the strengthening of their independence and sovereignty and the cultural and economic revival or salvation of their peoples, and also to express a strong commitment with peace by declaring themselves as ``non-aligned'' from either of the two nascent military blocks.

In order to fulfill the aims of debating on and advancing a strategy designed to achieve such objectives, the Bandung Asian-African Conference was held in Indonesia in April 1955. It was attended by 29 Heads of State and Government of the first postcolonial generation of leaders and its expressed goal was to identify and assess world issues at the time and coordinate policies to deal with them.

Although the Asian and African leaders who gathered in Bandung might have had differing political and ideological views or different approaches toward the societies they aspired to build or rebuild, there was a common project that united them and gave sense to a closer coordination of positions. Their shared program included the political decolonization of Asia and Africa. Moreover, they all agreed that the recently attained political independence was just a means to attain the goal of economic, social and cultural independence.

The Bandung meeting has been considered as the most immediate antecedent of the founding of the Movement of Non-Aligned Countries, which finally came into being six years later on a wider geographical basis when the First Summit Conference was held in Belgrade on September 1-6, 1961. This gathering was attended by the Heads of State and Government of 25 countries and observers from another three nations.

This First Summit of the Movement of Non-Aligned Countries was convened by the leaders of India, Indonesia, Egypt, Syria and Yugoslavia. On April 26, 1961, the Presidents of the Arab Republic of Egypt (Nasser) and Yugoslavia (Tito) addressed the Heads of State and Government of 21 ``non-Aligned'' countries and suggested that, taking recent world events and the rise of international tensions into account, a Conference should be held to promote an improvement in international relations, a resistance to policies of force and a constructive settlement of conflicts and other issues of concern in the world.

The Movement played an important role in the support of nations which were struggling then for their independence in the Third World and showed great solidarity with the most just aspirations of humanity. It contributed indisputably to the triumph in the struggle for national independence and decolonization, thus gaining considerable diplomatic prestige.

As one Summit after another was held in the 1960s and 1970s, ``non alignment'', turned already into the ``Movement of Non-Aligned Countries'' that included nearly all Asian and African countries, was becoming a forum of coordination to struggle for the respect of the economic and political rights of the developing world. After the attainment of independence, the Conferences expressed a growing concern over economic and social issues as well as over strictly political matters.

Something that attested to that was the launching at the Algiers Conference in 1973 of the concept of a ``new international economic order.''

By the end of the 1980s, the Movement was facing the great challenge brought about by the collapse of the socialist block. The end of the clash between the two antagonistic blocks that was the reason for its existence, name and essence was seen by some as the beginning of the end for the Movement of Non-Aligned Countries.

The Movement of Non-Aligned Countries could not spare itself difficulties to act effectively in an adverse international political situation marked by hegemonic positions and unipolarity as well as by internal difficulties and conflicts given the heterogeneity of its membership and, thus, its diverse interests.

Nevertheless, and in spite of such setbacks,the principles and objectives of non-alignment retain their full validity and force at the present international juncture. The primary condition that led to the emergence of the Movement of Non-Aligned Countries, that is, non-alignment from antagonistic blocks, has not lost its validity with the end of the Cold War. The demise of one of the blocks has not done away with the pressing problems of the world. On the contrary, renewed strategic interests bent on domination grow stronger and, even, acquire new and more dangerous dimensions for underdeveloped countries.

During the 14th Summit of the Non-Aligned Movement in Havana, Cuba in September 2006, the Heads of States and Governments of the member countries reaffirmed their commitment to the ideals, principles and purposes upon which the movement was founded and with the principles and purposes enshrined in the United Nations Charter.

The Heads of States and Governments stated their firm belief that the absence of two conflicting blocs in no way reduces the need to strengthen the movement as a mechanism for the political coordination of developing countries. In this regard they acknowledged that it remains imperative to strengthen and revitalize the movement. To do so, they agreed to strengthen concrete action, unity and solidarity between all its members, based on respect for diversity, factors which are essential for the reaffirmation of the identity and capacity of the movement to influence International relations.

They also stressed the need to promote actively a leading role for the movement in the coordination of efforts among member states in tackling global threats.

Inspired by the principles and purposes which were brought to the Non-Aligned Movement by the Bandung principles and during the First NAM Summit in Belgrade in 1961, the Heads of States and Governments of the member countries of the Non-Aligned Movement adopted in their 14th Summit in Havana the following purposes and principles of the movement in the present International juncture:

I. Purposes:

\begin{enumerate}
\def\labelenumi{\alph{enumi}.}
\item
  To promote and reinforce multilateralism and, in this regard, strengthen the central role that the United Nations must play.
\item
  To serve as a forum of political coordination of the developing countries to promote and defend their common interests in the system of international relations
\item
  To promote unity, solidarity and cooperation between developing countries based on shared values and priorities agreed upon by consensus.
\item
  To defend international peace and security and settle all international disputes by peaceful means in accordance with the principles and the purposes of the UN Charter and International Law.
\item
  To encourage relations of friendship and cooperation between all nations based on the principles of International Law, particularly those enshrined in the Charter of the United Nations.
\item
  To promote and encourage sustainable development through international cooperation and, to that end, jointly coordinate the implementation of political strategies which strengthen and ensure the full participation of all countries, rich and poor, in the international economic relations, under equal conditions and opportunities but with differentiated responsibilities.
\item
  To encourage the respect, enjoyment and protection of all human rights and fundamental freedoms for all, on the basis of the principles of universality, objectivity, impartiality and non-selectivity, avoiding politicization of human rights issues, thus ensuring that all human rights of individuals and peoples, including the right to development, are promoted and protected in a balanced manner.
\item
  To promote peaceful coexistence between nations, regardless of their political, social or economic systems.
\item
  To condemn all manifestations of unilateralism and attempts to exercise hegemonic domination in international relations.
\item
  To coordinate actions and strategies in order to confront jointly the threats to international peace and security, including the threats of use of force and the acts of aggression, colonialism and foreign occupation, and other breaches of peace caused by any country or group of countries.
\item
  To promote the strengthening and democratization of the UN, giving the General Assembly the role granted to it in accordance with the functions and powers outlined in the Charter and to promote the comprehensive reform of the United Nations Security Council so that it may fulfill the role granted to it by the Charter, in a transparent and equitable manner, as the body primarily responsible for maintaining international peace and security.
\item
  To continue pursuing universal and non-discriminatory nuclear disarmament, as well as a general and complete disarmament under strict and effective international control and in this context, to work towards the objective of arriving at an agreement on a phased program for the complete elimination of nuclear weapons within a specified framework of time to eliminate nuclear weapons, to prohibit their development, production, acquisition, testing, stockpiling, transfer, use or threat of use and to provide for their destruction.
\item
  To oppose and condemn the categorization of countries as good or evil based on unilateral and unjustified criteria, and the adoption of a doctrine of pre-emptive attack, including attack by nuclear weapons, which is inconsistent with international law, in particular, the international legally-binding instruments concerning nuclear disarmament and to further condemn and oppose unilateral military actions, or use of force or threat of use of force against the sovereignty, territorial integrity and independence of Non-Aligned countries.
\item
  To encourage States to conclude agreements freely arrived at, among the States of the regions concerned, to establish new Nuclear Weapons-Free Zones in regions where these do not exist, in accordance with the provisions of the Final Document of the First Special Session of the General Assembly devoted to disarmament (SSOD.1) and the principles adopted by the 1999 UN Disarmament Commission, including the establishment of a Nuclear Weapons Free Zone in the Middle East. The establishment of Nuclear Weapons-Free Zones is a positive step and important measure towards strengthening global nuclear disarmament and non-proliferation.
\item
  To promote international cooperation in the peaceful uses of nuclear energy and to facilitate access to nuclear technology, equipment and material for peaceful purposes required by developing countries.
\item
  To promote concrete initiatives of South-South cooperation and strengthen the role of NAM, in coordination with G.77, in the re-launching of North-South cooperation, ensuring the fulfillment of the right to development of our peoples, through the enhancement of international solidarity.
\item
  To respond to the challenges and to take advantage of the opportunities arising from globalization and interdependence with creativity and a sense of identity in order to ensure its benefits to all countries, particularly those most affected by underdevelopment and poverty, with a view to gradually reducing the abysmal gap between the developed and developing countries.
\item
  To enhance the role that civil society, including NGO´s, can play at the regional and international levels in order to promote the purposes, principles and objectives of the Movement.
\end{enumerate}

\begin{enumerate}
\def\labelenumi{\Roman{enumi}.}
\setcounter{enumi}{1}
\tightlist
\item
  Principles:
\end{enumerate}

\begin{enumerate}
\def\labelenumi{\alph{enumi}.}
\item
  Respect for the principles enshrined in the Charter of the United Nations and International Law.
\item
  Respect for sovereignty, sovereign equality and territorial integrity of all States.
\item
  Recognition of the equality of all races, religions, cultures and all nations, both big and small.
\item
  Promotion of a dialogue among peoples, civilizations, cultures and religions based on the respect of religions, their symbols and values, the promotion and the consolidation of tolerance and freedom of belief.
\item
  Respect for and promotion of all human rights and fundamental freedoms for all, including the effective implementation of the right of peoples to peace and development.
\item
  Respect for the equality of rights of States, including the inalienable right of each State to determine freely its political, social, economic and cultural system, without any kind of interference whatsoever from any other State.
\item
  Reaffirmation of the validity and relevance of the Movement's principled positions concerning the right to self-determination of peoples under foreign occupation and colonial or alien domination.
\item
  Non-interference in the internal affairs of States. No State or group of States has the right to intervene either directly or indirectly, whatever the motive, in the internal affairs of any other State.
\item
  Rejection of unconstitutional change of Governments.
\item
  Rejection of attempts at regime change
\item
  Condemnation of the use of mercenaries in all situations, especially in conflict situations.
\item
  Refraining by all countries from exerting pressure or coercion on other countries, including resorting to aggression or other acts involving the use of direct or indirect force, and the application and/or promotion of any coercive unilateral measure that goes against International Law or is in any way incompatible with it, for the purpose of coercing any other State to subordinate its sovereign rights, or to gain any benefit whatsoever.
\item
  Total rejection of aggression as a dangerous and serious breach of International Law, which entails international responsibility for the aggressor.
\item
  Respect for the inherent right of individual or collective self-defense, in accordance with the Charter of the United Nations.
\item
  Condemnation of genocide, war crimes, crimes against humanity and systematic and gross violations of human rights, in accordance with the UN Charter and International Law.
\item
  Rejection of and opposition to terrorism in all its forms and manifestations, committed by whomever, wherever and for whatever purposes, as it constitutes one of the most serious threats to international peace and security. In this context, terrorism should not be equated with the legitimate struggle of peoples under colonial or alien domination and foreign occupation for self-determination and national liberation.
\item
  Promotion of pacific settlement of disputes and abjuring, under any circumstances, from taking part in coalitions, agreements or any other kind of unilateral coercive initiative in violation of the principles of International Law and the Charter of the United Nations.
\item
  Defense and consolidation of democracy, reaffirming that democracy is a universal value based on the freely expressed will of people to determine their own political, economic, social, and cultural systems and their full participation in all aspects of their life.
\item
  Promotion and defense of multilateralism and multilateral organizations as the appropriate frameworks to resolve, through dialogue and cooperation, the problems affecting humankind.
\item
  Support to efforts by countries suffering internal conflicts to achieve peace, justice, equality and development.
\item
  The duty of each State to fully and in good faith comply with the international treaties to which it is a party, as well as to honor the commitments made in the framework of international organizations, and to live in peace with other States. v. Peaceful settlement of all international conflicts in accordance with the Charter of the United Nations.
\item
  Defense and promotion of shared interests, justice and cooperation, regardless of the differences existing in the political, economic and social systems of the States, on the basis of mutual respect and the equality of rights.
\item
  Solidarity as a fundamental component of relations among nations in all circumstances.
\item
  Respect for the political, economic, social and cultural diversity of countries and peoples.
\end{enumerate}

The movement has succeeded to create a strong front on the International level, representing countries of the third world in the International organizations on top of which the United Nations.

Current Challenges facing the NAM include the necessity of protecting the principles of International law, eliminating weapons of mass destruction , combating terrorism, defending human rights, working toward making the United Nations more effective in meeting the needs of all its member states in order to preserve International Peace , Security and Stability, as well as realizing justice in the international economic system.

On the other hand, the long-standing goals of the Movement remain to be realized. Peace, development, economic cooperation and the democratization of international relations, to mention just a few, are old goals of the non-aligned countries.

In conclusion, The Non-Aligned Movement, faced with the goals yet to be reached and the many new challenges that are arising, is called upon to maintain a prominent and leading role in the current International relations in defense of the interests and priorities of its member states and for achievement of peace and security for mankind.

source: \url{https://mea.gov.in/in-focus-article.htm?20349/History+and+Evolution+of+NonAligned+Movement}

\hypertarget{non-aligned-movement}{%
\subsection{Non aligned movement}\label{non-aligned-movement}}

Yuba Nath Lamsal

As the 17th summit of the Non-Aligned Movement (NAM) was underway in Margarita Island of Venezuela on September 13-18, a real debate started outside on the role and relevance of the NAM in the changed global scenario. The NAM was created 55 years ago at the height of the Cold War marked by a stiff superpower rivalry and division of the world into two rival camps each led by a super power. The NAM was necessary at that time as many countries of the `Third World' could not afford to side with any of the two rival blocs but chose to remain neutral. The NAM, therefore, became an appropriate forum for the countries wishing to have equal partnership and friendship with all countries irrespective of their ideological orientation and strategic alignment. But the international situation and scenario are markedly different at present. Now a question has arisen in the international forums and debates: Is the NAM necessary in the present situation or is it just a waste of resources, energy and time?

Common platform

The founding principles of the NAM were anti-colonialism and anti-imperialism, which appealed to many countries in the world that had either recently been liberated or were still waging national liberation movements to free themselves from the yoke of colonialism and imperialism. The NAM, thus, became a common platform for them to push forward their common agendas as specter of imperialism, colonialism and neo-colonialism continued to hang around and afflict the countries around the world, more particularly the developing nations.

Immediately after the World War II, the global power scenario changed. Until the World War II, the United Kingdom was the center of international power as it used to boast that the sun never set in its empire. The fundamental bases of British power and wealth were its colonies. But after the war, the national liberation movement across the globe intensified so rapidly that the erstwhile colonies were liberated one after another, heavily weakening British power. As British power diminished, the United States emerged to fill the vacuum in the international power politics and became the dominant global power, while Soviet Union suddenly came into the international scene as a rival power challenging the domination of the Western powers.

In the juggling for influence and power in the international arena, two distinct rival camps emerged with the capitalist United States leading the North Atlantic Treaty Organisation (NATO) and the communist Soviet Union commanding another rival block called the Warsaw Pact. The rivalry between these two blocs was so intense that the proxy wars between the two rival blocs were more dangerous than the traditional wars fought ever in the history of mankind. This was a period called as the `Cold War' during which many inter-state and intra-state wars were fought killing more people than the number of people killed in five years during the World War II.

This new scenario caused dilemma to many developing countries. It was more dangerous situation than that of the past. Against this background, the non-aligned movement came into existence. An international conference of 25 developing countries in Belgrade of erstwhile Yugoslavia in 1961 formally gave birth to the NAM, and this gathering was dubbed as the first summit of the NAM. But the NAM did not come so easily and overnight.

There had been quite a lot hidden and otherwise exercises before than that. The Bandung Conference of Indonesia in 1955 was, in fact, a beginning of the NAM as 19 participating countries felt the necessity of an organisation of the neutral countries. Gamal Abdel Nasser of Egypt, Kwame Nkrumah of Ghana, Jawaharlal Nehru of India, Sukarno of Indonesia and Josip Broz Tito of Yugoslavia were the mastermind behind this movement in which Nepal later joined as one of the founding members.

The Bandung Conference not only drew an outline of the NAM but also set some fundamental principles governing the new international organisation, which were later called as the ``Ten Principles of Bandung''. However, the `Ten Bandung Principles' were later modified in the first NAM Summit in Belgrade and shortened to five points, which are famously known as the `Five Principles of Peaceful Co-existence' or ' Panchasheel' as the fundamental basis of international relations.

The non-aligned movement was initiated at a time when the colonial system was in decline and independence struggles raged across Africa, Asia, Latin America and other regions of the world, which provided a hope of a better international world order for hitherto oppressed and exploited countries. Started humbly with 25 countries, NAM has now 120 members, although the organisation appears to be more in name rather than in action especially after the end of the Cold War.

Nepal is a founder member of the NAM. It participated in several rounds of formal and informal discussions on the need and modality of the NAM and its formal announcement in the Belgrade summit in 1961. The participating countries in the first NAM Summit were: Nepal, Afghanistan, Algeria, Yemen, Myanmar, Cambodia, Sri Lanka, Congo, Cuba, Cyprus, Egypt, Ethiopia, Ghana, Guinea, India, Indonesia, Iraq, Lebanon, Mali, Morocco, Saudi Arabia, Somalia, Sudan, Syria, Tunisia and Yugoslavia.

Nepal strictly adheres to the principles of the non-aligned movement in the conduct of its foreign policy, international relations and diplomacy. The Constitution of the Federal Democratic Republic of Nepal has incorporated these principles as the basic guiding principles of Nepal's foreign policy. This, in itself, is the testament of Nepal's unflinching faith and commitment to the NAM and its principles. Moreover, the fundamental principles of NAM or `Panchaseela' are more important for Nepal as these principles have their original roots in Nepal. Lord Buddha, who was born in Nepal, propounded the `Panchasheela' consisting of five codes of human conduct and international relations some 2500 year ago.

Lost charm

The NAM is, perhaps, the first international organisation that advocated the rights and interests of the oppressed and exploited countries of the South and raised voice against the countries of the north. After the collapse of the Soviet bloc and the end of the Cold War, there is no dearth of analysts both in the developed and developing countries alike who raise the question on the role and relevance of the non-aligned movement. The NAM has, of course, lost its original charm and but not its relevance.

The world and international balance of power definitely changed over the years and decades. The world is no longer bipolar nor will the present state of unipolarity stay forever. The world scenario has changed and is bound to keep on changing, continuously and steadily, in future, too, which is the law of nature. Several poles are slowly emerging challenging the US-led unipolar state, thus, requiring even stronger international movements to bring the developing countries together into a common forum for collectively safeguarding their common and shared interests refraining from siding with any of the poles, groups and blocs.

The NAM may appear irrelevant at present given the state of global order, but situation will not always remain as it is now and different world order is sure to emerge in which the NAM may be more relevant. But Non-Aligned Movement (NAM) requires reforms in itself with newer strategies and concepts to work better in coping with the newly emerged international situation and in achieving the shared objectives of the member countries.

source: \url{http://therisingnepal.org.np/news/14500}

\hypertarget{foreign-policy-national-defense-guiding-documents}{%
\chapter{Foreign policy, national defense guiding documents}\label{foreign-policy-national-defense-guiding-documents}}

Minister for Ministry of Foregin Affairs, Mr.~Pradeep Kuman Gyawali, unveiled Nepal's Foreign Policy document on Dec 6, 2019. The policy aims to secure the country's national interests by deepening mutual cooperation with all friendly countries. The policy paper is also a proper culmination to efforts by various task forces that were formed to upbringing an unified foreign policy at different times.

Although matters relating to the foreign policy conduct were expressed in one way or the other in country's various policy documents, including constitution, national security policy, annual budget and prime ministerial speeches, there had been longstanding lack of a integrated foreign policy document.

The policy also reiterates Nepal's long-standing policy of not allowing its soil to be used against neighboring countries and other countries, while expecting similar cooperation and commitment from them.

\hypertarget{features-of-foreign-policy-2077}{%
\section{Features of Foreign Policy, 2077}\label{features-of-foreign-policy-2077}}

\begin{itemize}
\tightlist
\item
  The country will conduct its independent and balanced ties with neighboring countries based on the principles of the UN Charter, Panchsheel, and good neighborliness.
\item
  Nepal will expand and enhance its ties with neighboring and other friendly countries based on sovereignty equality, mutual respect and benefit, read the policy.
\item
  Nepal's border will be kept intact by protecting international border points.
\item
  Boundary issues will be resolved through peaceful diplomatic means and dialogue based on treaties, agreements, historic documents, evidence and facts.
\item
\end{itemize}

\hypertarget{state-head-of-the-state-ministry-and-state-obligations}{%
\chapter{State, head of the state, ministry and state obligations}\label{state-head-of-the-state-ministry-and-state-obligations}}

\hypertarget{state}{%
\section{State}\label{state}}

Under international law, a State is a legal person with legal rights and obligations. As such a legal person the State is to be distinguished from other corporate entities, whether established by internal law such as a commercial multinational company, or by international law such as an international organization.

the status of a State is not conferred by national laws, although the constitutional law of its own internal national legal system may allocate and determine the manner of the executive, judicial, and legislative powers which it exercises. Its status is based on the exercise of effective control over a population within a defined territory which has been recognized and given effect in international law; and it is international law which determines its capacity as a legal person, its competence, and the nature and the extent of certain rights and duties. These attributes of States distinguish them from international organizations which, although also enjoying legal personality under international law, have no population or territory, and derive their legal personality and power in international law from treaty and the function which the particular organization is created to perform.

\hypertarget{privileges-and-immunities-of-the-state}{%
\subsection{Privileges and immunities of the state}\label{privileges-and-immunities-of-the-state}}

In 1972 the Council of Europe adopted the 1972 European Convention on State Immunity setting out a list of exceptions (including the State's waiver) to state immunity and this was followed by the enactment in the United States (1976), UK (1978), and in a number of other States of national legislation introducing a range of exceptions to state immunity for commercial transactions and proceedings for personal injuries caused in the territory of the State where the claim was brought. In 2004 the United Nations General Assembly adopted the Convention on the Jurisdictional Immunities of States and their Property. The Convention requires 30 ratifications to bring it into force but, based largely on state practice, as determined by the work of the International Law Commission and 10 years' discussions in the UN General Assembly's Sixth (Legal) Committee, it consolidates the restrictive approach to state immunity. By this approach no immunity is afforded in respect of proceedings brought against a State in national courts in respect of commercial transactions or claims for personal injury or tangible loss caused by act of the State within the territory of the State where proceedings are brought, and, as regards measures of execution, limited enforcement of judgments against the State is permitted in respect of state property in commercial use.

\hypertarget{ministry-of-foreign-affairs}{%
\section{Ministry of Foreign Affairs}\label{ministry-of-foreign-affairs}}

The miniser of foreign affairs is the regular but far from exclusive intermediary between the State and foreign countries; his functions are regulated by domestic legislation and traditions, and his powers vary according to the political organizations of different States. In international law the minister for foreign affairs is deemed to have the capacity to represent, to make treaties, and generally to act for the State. Ministers for foreign affairs, together with heads of state and heads of government, are recognized by the Vienna Convention on Treaties 1969, `in virtue of their functions and without having to produce Full Powers considered as representing their State for the purposes of performing all acts relating to a treaty' (Article 7(2)).

In contemporary Britain the mode of appointment of Her Majesty's Secretary of State for Foreign and Commonwealth Affairs is by the delivery to him by the Sovereign of the seals of office. There are three seals, viz.~a greater and lesser signet and a small seal called the cachet; all of these are engraved with the Royal Arms. The two former now differ only in point of size. In the Foreign and Commonwealth Office, diplomatic and consular commissions signed by the Sovereign pass under the greater signet; the lesser is used in the case of royal exequaturs granted to foreign consular officers, and for royal warrants (such as instruments authorizing the affixing of the Great Seal to Full Powers and to ratifications of treaties); the cachet, now very rarely deployed, is used to seal the envelopes of letters containing communications of a personal character made by the Queen to foreign sovereigns.

Patents were issued from the fifteenth century onwards till 1852. From that time the practice was intermittent till 1868, but since the latter date patents have not been issued, nor in any case would they affect the powers of the Secretary of State for these follow the seals. The British Foreign and Commonwealth Secretary holds a full power from the Queen, authorizing him to negotiate and conclude, subject if necessary to Her Majesty's ratification, any treaty in respect of Great Britain and Northern Ireland.

Today governments of other countries address themselves to the minister for foreign affairs either through their own accredited diplomatic agent, or through the diplomatic agent who represents his sovereign or government at their own capital. The former is the normal channel of communication and is generally preferred, since by instructing their own representative a government can be confident that the wording and manner of delivery of their message conforms as exactly as possible to their intentions. Moreover, he will report immediately, if necessary, on the reception it obtained. However, there are occasions when a minister for foreign affairs, wishing to communicate his government's message with the greatest force, will summon a foreign ambassador to receive it personally. In such cases the government's own representative will of course be informed and may be required to take complementary action in the capital where he is accredited. These procedures are flexible, and the choice will depend upon circumstances and upon the personalities involved.

As a general rule Notes and other formal communications concerning relations with other countries are signed by the minister for foreign affairs or, more frequently, on his behalf. Under his general supervision are drawn up documents connected with foreign relations, drafts of treaties and conventions, statements of fact and law, manifestos, and declarations. The negotiation of treaties rests with him and he watches over their execution. Ratifications of treaties are exchanged by him or his agents. He proposes to the head of state the nomination of diplomatic agents, he is responsible for drawing up their credentials and Full Powers for signature by the head of state, and for the issue of their instructions. He advises the head of state as to the acceptance of persons who have been proposed to be accredited to him, and also as regards the issue of exequaturs to foreign consular officers. Such officers receive their orders from him. Foreign representatives address themselves to him in order to obtain an audience of the head of state.

At the present day the duties and responsibilities of the minister who is entrusted with the conduct of the foreign relations of his country range over a far wider field. The birth of new States, the advancement of others, constitutional changes which may occur in their methods of government, the growth of organizations designed to foster a better understanding between the nations of the world, the ever-increasing complexity of international relationships, and the many questions to which all these give rise, have largely extended the area within which diplomacy finds its proper scope, and call for close and unremitting attention.

On taking office the minister for foreign affairs informs the diplomatic representatives of foreign States of his appointment, and customarily receives them as soon as possible thereafter at his official residence to exchange greetings with them. He also informs the diplomatic agents of his own country accredited abroad.

In every country the foreign minister is assisted by a trained staff that, under his guidance, constitute the foreign office or ministry for foreign affairs. In Nepal, the permanent staff of the Ministry of Foreign Affairs has at its head the Foreign Secretary (currently served by Mr.~Bharat Paudyal as of Jan, 2021), who has the rank of ambassador.

\hypertarget{roles-responsibility-and-functions}{%
\subsection{Roles, responsibility, and functions}\label{roles-responsibility-and-functions}}

According to Government of Nepal (Allocation of Business) Rules, 2069 (2012) Ministry of Foreign Affairs has the following roles, responsibility, and function:

\begin{itemize}
\tightlist
\item
  Formulation, implementation, monitoring and evaluation of foreign policy, plan and programs of Nepal
\item
  Relation with foreign nations
\item
  Representation of Nepal in foreign countries
\item
  Publicity of Nepal in foreign countries
\item
  Passport and visa to be issued in abroad
\item
  Hospitality Management
\item
  Protocol
\item
  Claim over a person of a Nepali or foreign citizen by the respective governments.
\item
  Diplomatic protection (immunities) and privileges
\item
  Record of Nepali citizens who are in abroad and their right, interest and protection.
\item
  Non-resident Nepalese
\item
  Economic diplomacy
\item
  Development and promotion of public and non-governmental organizations at international level
\item
  Consular practice
\item
  United Nations, South Asian Association of Regional Cooperation and other international and regional organization
\item
  Foreign diplomatic mission in Nepal
\item
  Negotiation and agreement at diplomatic level (on the matters which do not fall under any other ministry)
\item
  Operation of Nepal foreign service
\end{itemize}

\hypertarget{organizational-structure}{%
\subsection{Organizational structure}\label{organizational-structure}}

There are two departments under the Ministry of Foreign Affairs:

\begin{itemize}
\tightlist
\item
  Department of Passport (Nepal), Narayanhiti, Kathmandu
\item
  Department of Consular Services, Tripureshwor, Kathmandu
\end{itemize}

The Ministry has operated a Liaison Office in the border town of Birgunj since 2005.

The Ministry operates the Institute of Foreign Affairs (IFA) in Tripureshwor, Kathmandu.

\hypertarget{international-organizations-treaties-conventions-and-committments}{%
\chapter{International organizations, treaties, conventions and committments}\label{international-organizations-treaties-conventions-and-committments}}

\hypertarget{united-nations}{%
\section{United Nations}\label{united-nations}}

\hypertarget{purposes-and-principels}{%
\subsection{Purposes and principels}\label{purposes-and-principels}}

The UN has been aptly described as a Standing Diplomatic Conference: it is a worldwide association of states which, on signing the Charter of the United Nations, subscribe to its purposes and agree to act in accordance with its principles.

\hypertarget{purposes}{%
\subsubsection{Purposes}\label{purposes}}

\begin{enumerate}
\def\labelenumi{\arabic{enumi}.}
\tightlist
\item
  To maintain international peace and security, and to that end: to take effective collective measures for the prevention and removal of threats to the peace, and for the suppression of acts of aggression or other breaches of the peace, and to bring about by peaceful means, and in conformity with the principles of justice and international law, adjustment or settlement of international disputes or situations which might lead to a breach of the peace.
\item
  To develop friendly relations among nations based on respect for the principle of equal rights and self-determination of peoples, and to take other appropriate measures to strengthen universal peace.
\item
  To achieve international cooperation in solving international problems of an economic, social, cultural, or humanitarian character, and in promoting and encouraging respect for human rights and for fundamental freedoms for all without distinction as to race, sex, language, or religion; and
\item
  To be a center for harmonising the actions of nations in the attainment of these common ends.
\end{enumerate}

\hypertarget{principles}{%
\subsubsection{Principles}\label{principles}}

\begin{enumerate}
\def\labelenumi{\arabic{enumi}.}
\tightlist
\item
  The UN is based on the principle of the soverign equality of all its members.
\item
  All members, in order to ensure to all of them the rights and benefits resulting from membership, shall fulfil in good faith the obligations assumed by them in accordance with the charter.
\item
  All members shall settle their international disputes by peaceful means in such a manner that international peace and security, and justice, are not endangered.
\item
  All members shall refrain in their international relations from the threat or use of force against the territorial integrity or political indepedence of any state, or in any other manner inconsistent with the Purposes of the United Nations.
\item
  All members shall give the UN every assistance in any action it takes in accordance with the Charter, and shall refrain from giving assistance to any state against which the UN is taking a preventive or enforcement action.
\item
  The United Nations shall ensure that states which are not members of the Organisation act in accordance with these Principles so far as may be necessary for the maintenance of international peace and security.
\item
  Nothing contained in the Charter shall authorise the United Nations to intervene in matters which are essentially within the domestic jurisdiction of any state or shall require the members to submit such matters to settlement under the Charter; but this principle shall not prejudice the application of enforcement measures under chapter VII.
\end{enumerate}

\hypertarget{the-un-charter}{%
\subsection{The UN Charter}\label{the-un-charter}}

The Charter is divided into 111 articles grouped in 19 chapters:

\begingroup\fontsize{8}{10}\selectfont

\begin{longtable}[t]{l>{\raggedright\arraybackslash}p{20em}>{\raggedright\arraybackslash}p{10em}}
\caption{\label{tab:unnamed-chunk-2}Chapter summary of the UN Charter}\\
\toprule
Chapter & Title & Articles\\
\midrule
\rowcolor{gray!6}  I & Purposes and principles & Articles 1 and 2\\
\cmidrule{1-3}
II & Membership & Articles 3-6\\
\cmidrule{1-3}
\rowcolor{gray!6}  III & Organs & Articles 7 and 8\\
\cmidrule{1-3}
 & The General Assembly & \\
\cmidrule{2-3}
\rowcolor{gray!6}   & Composition & Article 9\\
\cmidrule{2-3}
 & Functions and powers & Articles 10-17\\
\cmidrule{2-3}
\rowcolor{gray!6}   & Voting & Articles 18 and 19\\
\cmidrule{2-3}
\multirow{-5}{*}{\raggedright\arraybackslash IV} & Procedure & Articles 20-22\\
\cmidrule{1-3}
\rowcolor{gray!6}   & The security Council & \\
\cmidrule{2-3}
 & Composition & Article 23\\
\cmidrule{2-3}
\rowcolor{gray!6}   & Functions and powers & Articles 24-26\\
\cmidrule{2-3}
 & Voting & Article 27\\
\cmidrule{2-3}
\rowcolor{gray!6}  \multirow{-5}{*}{\raggedright\arraybackslash V} & Procedure & Articles 28-32\\
\cmidrule{1-3}
VI & Pacific settlement of disputes & Articles 33-38\\
\cmidrule{1-3}
\rowcolor{gray!6}  VII & Action with respect to threats to the peace, breaches of the peace, and acts of aggression & Articles 39-51\\
\cmidrule{1-3}
VIII & Regional arrangements & Articles 52-54\\
\cmidrule{1-3}
\rowcolor{gray!6}  IX & International economic and social cooperation & Articles 55-60\\
\cmidrule{1-3}
 & The Economic and Social Council & \\
\cmidrule{2-3}
\rowcolor{gray!6}   & Composition & Article 61\\
\cmidrule{2-3}
 & Functions and powers & Articles 62-66\\
\cmidrule{2-3}
\rowcolor{gray!6}   & Voting & Article 67\\
\cmidrule{2-3}
\multirow{-5}{*}{\raggedright\arraybackslash X} & Procedure & Articles 68-72\\
\cmidrule{1-3}
\rowcolor{gray!6}  XI & Declaration regarding non-self governing territories & Articles 73-74\\
\cmidrule{1-3}
XII & International Trusteeship system & Articles 75-85\\
\cmidrule{1-3}
\rowcolor{gray!6}   & The Trusteeship Council & \\
\cmidrule{2-3}
 & Composition & Article 86\\
\cmidrule{2-3}
\rowcolor{gray!6}   & Functions and powers & Articles 87 and 88\\
\cmidrule{2-3}
 & Voting & Article 89\\
\cmidrule{2-3}
\rowcolor{gray!6}  \multirow{-5}{*}{\raggedright\arraybackslash XIII} & Procedure & Articles 90 and 91\\
\cmidrule{1-3}
XIV & The International Court of Justice & Articles 92-96\\
\cmidrule{1-3}
\rowcolor{gray!6}  XV & The Secretariat & Articles 97-101\\
\cmidrule{1-3}
XVI & Miscellaneous provisions & Articles 102-105\\
\cmidrule{1-3}
\rowcolor{gray!6}  XVII & Transitional security arrangements & Articles 106-107\\
\cmidrule{1-3}
XVIII & Amendments & Articles 108-109\\
\cmidrule{1-3}
\rowcolor{gray!6}  XIX & Ratification and signature & Articles 110-111\\
\bottomrule
\end{longtable}
\endgroup{}

Membership of the UN consists of the `original members' (those states which signed the Washington Declaration in 1942 or took part in the UN Conference on International Organization in San Francisco in 1945, and signed and ratified the Charter in accordance with the prescribed procedure); and those states subsequently accepted as members in accordance with the provisions of the Charter.

Membership is further open to all other `peace-loving' states which accept the obligations contained in the Charter and, in the judgment of the UN, are able and willing to carry them out. The admission of new members is dependent on the approval of the General Assembly on the is recommendation of the Security Council.

A member which has persistently violated the principles of the Charter may be expelled from the United Nations by the General Assembly on the recommendation of the Security Council; or may have its rights and privileges of membership suspended by the General Assembly on the recommendation of the Security Council if it has been the object of preventive or enforcement action taken by the Security Council. These rights and privileges, however, may be restored by the Security Council.

Each state is entitled to one vote in the General Assembly and in its dependent committees and councils.

Provision is made in chapter XVIII for amendments to the Charter, and these come into force when they have been

\begin{enumerate}
\def\labelenumi{(\alph{enumi})}
\tightlist
\item
  adopted by a vote of two-thirds of the members of the General Assembly, and
\item
  ratified in accordance with their respective constitutional processes by two-thirds of the members of the Security Council.
\end{enumerate}

The procedure is the same if a General Conference of the members of the United Nations is convened in terms of article 109 for the purpose of reviewing the Charter, except that the requirement for the initial vote (prior to ratification) is a two-thirds majority of those present at the Conference.

The official languages of the UN are Arabic, Chinese, English, French, Russian and Spanish (?? more).

The UN, in terms of its Charter, is based on six principal organs: the General Assembly, the Security Council, the Economic and Social Council, the Trusteeship Council, the International Court of Justice, and the Secretariat. Generally speaking the Assembly and Security Council are the political and legislative bodies, ECOSOC and the Trusteeship Council are specialist bodies dependent on the General Assembly, and the International Court of Justice is an independent body.

The reform of the administration and decision-making process of the United Nations is currently under consideration, and is centered on the composition of the Security Council and the creation of a Development Operations Group, amalgamating such related organizations as UNDP, UNFPA and UNICEF.

\hypertarget{the-general-assembly}{%
\subsection{The General Assembly}\label{the-general-assembly}}

The General Assembly consists of all members of the United Nations. In terms of the Charter, the United Nations may not intervene in matters which are essentially within the jurisdiction of a state, except in respect of the application of enforcement measures in accordance with chapter VII (threats to the peace, breaches of the peace and acts of aggression). Subject to this proviso, the functions of the Assembly are:

\emph{To consider and discuss} (i) any matter within the scope of the Charter or relating to the powers and functions of any of the organs provided for in the Charter, (ii) general principles of cooperation in the maintenance of international peace and security, including principles governing disarmament and the regulation of armaments, and (iii) any question relating to the maintenance of international peace and security.

\emph{To call the attention} of the Security Council to situations which are likely to endanger international peace and security.

\emph{To make recommendations} to the Security Council or to member states (or both) or to a non-member state involved in questions relating to the maintenance of international peace and security in respect of (ii) and (iii) above; for the purpose of promoting international cooperation in the political field and encouraging the progressive development of international law and its codification; and for promoting international cooperation in the economic, social, cultural, educational and health fields, and assisting in the realization of human rights and fundamental freedoms for all without distinction as to race, sex, language or religion.

\emph{To make recommendations} subject to the proviso that if the Security Council is already exercising its prescribed function in respect of such matters, recommendations will only be made if asked for relating to:

\begin{enumerate}
\def\labelenumi{\arabic{enumi}.}
\tightlist
\item
  any matter within the scope of the Charter or relating to the powers and functions of any of the organs provided for in the Charter;
\item
  the peaceful adjustment of any situation, regardless of origin, which it deems likely to impair the general welfare or friendly relations among nations, including situations resulting from a violation of the provisions of the Charter setting forth the Purposes and Principles of the United Nations.
\end{enumerate}

\emph{To receive and consider annual and special reports} from the Security Council (including an account of the measures that it has decided upon or taken to maintain international peace and security), and from the other organs of the United Nations.

\emph{To perform such functions} with respect to the international trusteeship system as are assigned to it under chapters XII and XIII, including the approval of the trusteeship agreements for areas not designated as strategic.

\emph{To consider and approve} the United Nations budget and any financial and budgetary arrangements with the Specialised Agencies and to examine the administrative budgets of such Specialised Agencies with a view to making recommendations to the Agencies concerned.

In the event of a non-member state being involved in a question relating to maintenance of international peace and security, the Assembly may make recommendations to the state concerned; and any question on which action is necessary in such circumstances shall be referred to the Security Council by the Assembly either before or after discussion.

The regular sessions of the Assembly as a rule begin in New York on the third Tuesday in September each year and continue until mid-December, but special sessions may be called by the Secretary-General at the request of the Security Council or of a majority of the members of the UN. At the start of each regular session, the Assembly elects a new President, 21 Vice-Presidents and the Chairmen of the Assembly's seven Main Committees. To ensure equitable geographical representation, the Presidency of the Assembly rotates each year among group of States who select their own candidate.

The work of the General Assembly is coordinated and to a considerable extent organized by two procedural committees: the General Committee, which is made up of the President and Vice-Presidents of the General Assembly and the heads of the seven Main Committees and the Credentials Committee.

The Main Committees, which consider in advance matters placed on the agenda of the General Assembly and (when so requested) make recommendations for consideration by the Assembly in plenary session, are:

\begin{itemize}
\tightlist
\item
  Firs Committee: Disarmament and related international security matters
\item
  Special Political Committee
\item
  Second Committee: Economic and Financial
\item
  Third Committee: Social, Humanitarian and Cultural
\item
  Fourth Committee: Trusteeship and non-self-governing territories
\item
  Fifth Committee: Administrative and Budgetary
\item
  Sixth Committee: Legal
\end{itemize}

The Special Political Committee was created primarily to relieve the First Committee; and there are two Standing Committees: the Committee on Contributions, and the Advisory Committee on Administrative and Budgetary Questions. Subsidiary and ad hoc committees are set up from time to time to deal with specific problems.

Voting in the Assembly is by simple majority of the members present and voting, except in the following circumstances, when the necessary majority is the affirmative vote of two-thrids of the members present and voting:

\begin{itemize}
\tightlist
\item
  recommendations with respect to the maintenance of international peace and security,
\item
  the election of non-permanent members of the Security Council,
\item
  the election of members of the Economic and Social Council,
\item
  the election of members of the Trusteeship Council,
\item
  the admission of new members to the United Nations,
\item
  the suspension of the rights and privileges of members,
\item
  the expulsion of members,
\item
  matters relating to the operation of the trusteeship system,
\item
  budgetary questions,
\item
  any other matters considered by the Assembly (by a simple majority of the members present and voting) to be sufficiently important to require a two-thirds majority.
\end{itemize}

A member in arrears in the payment of its financial contributions to the UN is not entitled to vote if its arrears equal or exceed the amount of its contribution due for preceding two full years, unless the General Assembly is satisfied that failure to pay is due to circumstances beyond the members' control.

Much of the work of the United Nations is conducted on the basis of regional groups: e.g.~African States, Asian States, Latin American States and Western European and Other States. For election purposes the USA falls within Western European and Other States.

\hypertarget{the-security-council}{%
\subsection{The Security Council}\label{the-security-council}}

The security council consists of 15 members: China, France, Russia, the United Kingdom and the USA, who constitute the five permanent members; and ten non-permanent members elected by the General Assembly for a term of two years (provided that no member is elected for two consecutive periods).

(Continue from\ldots Diplomatic Handbook; Feltham R.G.)

\hypertarget{the-economic-and-social-council}{%
\subsection{The Economic and Social Council}\label{the-economic-and-social-council}}

\hypertarget{the-trusteeship-council}{%
\subsection{The Trusteeship Council}\label{the-trusteeship-council}}

\hypertarget{the-international-court-of-justice}{%
\subsection{The International Court of Justice}\label{the-international-court-of-justice}}

\hypertarget{the-secretary-general-and-the-secretariat}{%
\subsection{The Secretary-General and the Secretariat}\label{the-secretary-general-and-the-secretariat}}

\hypertarget{un-peace-keeping-operations}{%
\subsection{UN peace-keeping operations}\label{un-peace-keeping-operations}}

\hypertarget{intergovernmental-agencies-related-to-the-un-including-specialised-agencies}{%
\subsection{Intergovernmental agencies related to the UN (including specialised agencies)}\label{intergovernmental-agencies-related-to-the-un-including-specialised-agencies}}

\hypertarget{food-and-agriculture-organization}{%
\subsubsection{Food and Agriculture Organization}\label{food-and-agriculture-organization}}

\hypertarget{international-atomic-energy-agency}{%
\subsubsection{International Atomic Energy Agency}\label{international-atomic-energy-agency}}

\hypertarget{international-civil-aviation-organization}{%
\subsubsection{International Civil Aviation Organization}\label{international-civil-aviation-organization}}

\hypertarget{international-fund-for-agricultural-development}{%
\subsubsection{International Fund for Agricultural Development}\label{international-fund-for-agricultural-development}}

\hypertarget{international-labor-organization}{%
\subsubsection{International Labor Organization}\label{international-labor-organization}}

\hypertarget{international-maritime-organization}{%
\subsubsection{International Maritime Organization}\label{international-maritime-organization}}

\hypertarget{international-telecommunications-union}{%
\subsubsection{International Telecommunications Union}\label{international-telecommunications-union}}

\hypertarget{the-international-monetary-fund}{%
\subsubsection{The International Monetary Fund}\label{the-international-monetary-fund}}

\hypertarget{the-world-bank}{%
\subsubsection{The World Bank}\label{the-world-bank}}

\hypertarget{united-nations-educational-scientific-and-cultural-organization}{%
\subsubsection{United Nations Educational Scientific and Cultural Organization}\label{united-nations-educational-scientific-and-cultural-organization}}

\hypertarget{united-nations-development-programme}{%
\subsubsection{United Nations Development Programme}\label{united-nations-development-programme}}

\hypertarget{universal-postal-union}{%
\subsubsection{Universal Postal Union}\label{universal-postal-union}}

\hypertarget{world-health-organization}{%
\subsubsection{World Health Organization}\label{world-health-organization}}

\hypertarget{world-intellectual-property-organization}{%
\subsubsection{World Intellectual Property Organization}\label{world-intellectual-property-organization}}

\hypertarget{world-meteorological-organization}{%
\subsubsection{World Meteorological Organization}\label{world-meteorological-organization}}

\hypertarget{subsidiary-organizations}{%
\subsubsection{Subsidiary organizations}\label{subsidiary-organizations}}

\begin{enumerate}
\def\labelenumi{\arabic{enumi}.}
\tightlist
\item
  Human rights
\item
  International law commission
\item
  International research and training institute for the advancement of women
\item
  Other consultative bodies
\end{enumerate}

\begin{itemize}
\tightlist
\item
  Commission for the Unification and Rehabilitation of Korea
\item
  Committee on the Peaceful Uses of Outer Space
\item
  Conciliation Commission for Palestine
\item
  Scientific Committee on the Effects of Radiation
\item
  Working Group on Direct Broadcasting Satellites
\item
  United Nations University (Tokyo)
\item
  World Food Council
\end{itemize}

\begin{enumerate}
\def\labelenumi{\arabic{enumi}.}
\setcounter{enumi}{3}
\tightlist
\item
  UN Conference on Disarmament and the Disarmament Commission
\item
  UN Children's Emergency Fund
\item
  UN Conference on Trade and Development
\item
  UN Environment Programme
\item
  UN Fund for Population Activities
\item
  UN High Commission for Refugees
\item
  UN Industrial Development Organization
\item
  UN Institute for Training and Research
\item
  UN Relief and Workd Agency for Palestine Refugees and the Near East
\end{enumerate}

\hypertarget{un-women}{%
\subsection{UN Women}\label{un-women}}

\begin{questions}

\question Describe about United Nations Commission on Status of Women.

\begin{solution}

The Commission on the Status of Women (CSW or UNCSW) is a functional commission of the United Nations Economic and Social Council (ECOSOC), one of the main UN organs within the United Nations. CSW has been described as the UN organ promoting gender equality and the empowerment of women. Every year, representatives of Member States gather at United Nations Headquarters in New York to evaluate progress on gender equality, identify challenges, set global standards and formulate concrete policies to promote gender equality and advancement of women worldwide. In April 2017, ECOSOC elected 13 new members to CSW for a four-year term 2018–2022. One of the new members is Saudi Arabia, which has been criticised for its treatment of women.

The commission was formed on 21 June 1946 and is headquartered in NY, USA.

\end{solution}

\end{questions}

\hypertarget{un-agencies-in-nepal}{%
\subsection{UN agencies in Nepal}\label{un-agencies-in-nepal}}

\begin{itemize}
\tightlist
\item
  Food and Agriculture Organization (FAO)
\item
  International Fund for Agricultural Development (IFAD)
\item
  International Labour Organization (ILO)
\item
  International Organization for Migration (IOM)
\item
  Office for the Coordination of Humanitarian Affairs, Humanitarian Support Unit (OCHA HSU)
\item
  UN HABITAT
\item
  UN Women
\item
  Joint United Nations Programme on HIV/AIDS (UNAIDS)
\item
  United Nations Capital Development Fund (UNCDF)
\item
  United Nations Development Programme (UNDP)
\item
  UNDSS
\item
  United Nations Educational, Scientific and Cultural Organisation (UNESCO)
\item
  The United Nations Population Fund (UNFPA)
\item
  United Nations High Commissioner for Refugees (UNHCR)
\item
  UN Information Centre (UNIC)
\item
  United Nations Children Fund (UNICEF)
\item
  Regional Centre for Peace and Disarmament (UNODA-RCPD)
\item
  United Nations Office for Drugs and Crime (UNODC)
\item
  United Nations Volunteers (UNV)
\item
  World Food Programme (WFP)
\item
  World Health Organization Office (WHO)
\end{itemize}

\hypertarget{world-trade-organization-wto}{%
\section{World Trade Organization (WTO)}\label{world-trade-organization-wto}}

The WTO agreements cover a wide range of activities such as agriculture, textiles and clothing, banking, telecommunications, government purchases, industrial standards and product safety, food and sanitation regulations and intellectual property. GATT is now the WTO's principal rule-book for trade in goods. The Uruguay Round also created new rules for dealing with trade in services, relevant aspects of intellectual property, dispute settlement, and trade policy reviews. The complete set runs to some 30,000 pages consisting of about 30 agreements and separate commitments (called schedules) made by individual members in specific areas such as lower customs duty rates and services market-opening (WTO, 2008).

\hypertarget{wto-agreements}{%
\subsection{WTO agreements}\label{wto-agreements}}

\hypertarget{agreement-related-to-agriculture}{%
\subsubsection{Agreement related to Agriculture}\label{agreement-related-to-agriculture}}

Agreement on Agriculture (AOA)

The AOA was the outcome of the Uruguay Round (UR) negotiations that started in 1986 and concluded in 1994. The long-term objectives set by the AOA are to establish a fair and market oriented agricultural trading system through substantial reductions in agricultural support and protection.

AOA deals all the matters of tariff, domestic support and export subsidies. It is rightly identified that the root cause of distortion of international trade in agriculture is the massive domestic subsidies given by industrialized countries over the decades (WTO, 1995). In order to minimize such dumped exports and to keep their markets open for efficient agricultural producers of the world, the starting point has to be the reduction of the domestic production subsidies given by the industrialized countries, followed by reduction of export subsidies and the volume of subsidized exports, and minimum market access opportunity for foreign agricultural producers".

Domestic Support

Domestic support provides commitments to reduce agricultural subsidies and other programmes including those raise or guarantee farm gate prices and farmers' incomes. These supports are divided by the AOA into different boxes in tricky way.

Implications of AOA Domestic measures to Nepal

The agricultural businesses are linked to trade and hence to WTO in several ways. The agribusiness needs imported inputs, machineries and technology from abroad. The major inputs imported by Nepal are chemical fertilizers, pesticides, hormones, veterinary medicines, seeds and packaging materials. As the market is liberalized and imports are not restricted, the agribusiness also needs to compete in domestic markets with imported goods. Some agribusinesses generate exportable goods. The export can be done either under bilateral agreements (such as with India), regional agreements (like with Bangladesh and Pakistan) and multilateral agreements (like with the countries not involved in bilateral and regional agreements such as USA, European Union, Japan). All such activities of agribusiness like import of inputs, export of outputs and competition in domestic markets are affected Agreement on Agriculture (AOA), Application of Sanitary and Phytosanitary (SPS) measures, Agreement on Technical Barriers to Trade (TBT) and Agreement on Trade Related Aspects of Intellectual Property Rights (TRIPS) primarily.

Market Access

Market access is one of the three main pillars of the AOA -- the other two being domestic support measures and export competition. It deals with rules and commitments related to import of goods. Its purpose is to expand trade by preventing various non-tariff barriers and by binding and reducing tariffs. Besides tariffs, other trade policy instruments covered by the market access pillar include Tariff Rate Quotas (TRQs) and Special Safeguard (SSG) as a trade remedy measure (Sharma and Karki, 2004).

In the WTO context, market access is about both obligations and rights. Nepal's obligation is to provide market access to other Members in return for her right of access to others' markets for Nepalese goods on multilaterally agreed terms. Thus, a balanced analysis of market access provisions would cover both obligations and rights.

The AoA provisions on market access:

\begin{itemize}
\tightlist
\item
  Prohibition of quantitative restrictions on imports
\item
  Tariff binding and reduction
\item
  Bound versus applied tariffs
\item
  Tariff Rate Quota
\item
  Special Safeguard Measures
\item
  Export subsidies and export restrictions
\end{itemize}

In common with all least developed countries (LDCs) and a majority of the developing countries, Nepal does not subsidize exports. At the time of the WTO accession, it committed not to subsidize exports in future also. One of the conclusions of the study is that this commitment is unlikely to have any negative implications for the Nepalese agriculture for two reasons. First, export subsidization is not a sound economic policy. Second, Nepal could not afford export subsidies at its low level of economic development. On the other hand, there were several instances in the past when export subsidization by other countries had some negative effects on the Nepalese agriculture. Therefore, it is in Nepal's interest to tighten WTO rules on export subsidy (Tiwari et al, 2004).

Implication on Nepalese Agriculture:

\begin{itemize}
\tightlist
\item
  Nepal can promote exports through various incentive measures of the WTO Subsidies Agreement.
\item
  Limited implications of the WTO rules on export restriction and taxation policies.
\item
  Nepal is occasionally affected negatively by export subsidization by others
\item
  Domestic policy issues and analytical needs
\end{itemize}

Sanitory and Phyto Sanitory (SPS)

Food quality and safety issues have entered into a new era of evolution as it involves integrated effort linking production to consumption in the entire food chain. The traditional domain of inspecting and analysing the end product does not necessarily meet the requirement of emerging trade regime of WTO and related agreements such as the SPS.

The main objectives of the SPS Agreement are the following (Karki et al, 2004).

\begin{itemize}
\tightlist
\item
  Protect and improve the current human health, animal health, and phytosanitary situation of all Member countries; and
\item
  The entry, establishment or spread of pests, disease, disease-carrying organisms or disease-causing organisms;
\item
  Additives, contaminants, toxins or disease-causing organisms in foods, beverages or feedstuffs;
\item
  Carried by animals, plants or products thereof, or from the entry, establishment or spread of pests; or
\item
  Prevent or limit other damage within the territory of the Member from the entry, establishment or spread of pests.
\end{itemize}

The following are the main elements of the SPS Agreement (Karki et al, 2004).

\begin{enumerate}
\def\labelenumi{\arabic{enumi}.}
\tightlist
\item
  Harmonization
\item
  Equivalence
\item
  Risk assessment
\item
  Transparency
\item
  Consultation and dispute settlement:
\item
  Technical cooperation and Special and Differential Treatment
\end{enumerate}

The SPS Agreement and Trade In Live Animals and Animal Products

The SPS Agreement recognizes the International Office of Epizootics (OIE) as the relevant international organisation responsible for the development and promotion of international animal health standards, guidelines, and recommendations affecting trade in live animals and animal products. Similarly, the official (Public) Veterinary Services of a country are recognized as the relevant authority with ultimate responsibility for animal health matters involving international trade in live animals and animal products.

The OIE provides detailed Guidelines for the Evaluation of Veterinary Services (Chapter 1.3.4. of the Terrestrial Animal Health Code). According to these guidelines, the national Veterinary Services should be able to demonstrate capacity, supported by appropriate legislation, in the following areas (Mahato et al, 2004).

\begin{itemize}
\tightlist
\item
  Exercise control over all animal health matters
\item
  Prescribe methods for control and to exercise systematic control over the import and export processes of animals and animal products in so far as this control relates to sanitary and zoo-sanitary matters.
\item
  Control imports and transit of animals, animal products and other materials that may introduce animal diseases.
\item
  Present a functional animal disease reporting system which covers all regions of the country
\item
  Provide accurate and valid certification for exports of animals and animal products.
\end{itemize}

The SPS Agreement: Trade in Plants and Plant Products

With increasing trade in plant and plant products, the risk of the spread of harmful pests and diseases has also increased. The negative impact on plant health and plant products could be substantial, e.g.~an imported harmful pest could destroy entire orange production in a country or a region, or could result into reduced yield, quality deterioration and environmental pollution.

In the case of Nepal, export and import of agricultural and forest-based products through the long and porous borders had been taking place almost without any phytosanitary considerations until recently. To a large extent the practice continues even now. As the potential negative effects are being increasingly recognized and as WTO Members started to implement the SPS Agreement since 1995 the situation is changing. Nepal also had its share of the deleterious effects of the harmful pests that came with imported plants and the difficulties in exporting plants and products, particularly to India in recent years. In view of this, and her commitment to implement provisions of the SPS agreement by 1 January 2007 timely action in this direction has become necessary ( KC et al, 2004).

Agreement on Technical Barriers to Trade

Agreement on Technical Barriers to Trade (TBT) establishes disciplines on technical regulations, standards and conformity assessment procedures for agricultural and well as non-agricultural products. Technical regulations and standards deal with product characteristics, processes and production methods related to a product, and may also bear upon terminology, symbols, packaging, marking or labelling. Conformity assessment procedures, such as testing, inspection, evaluation and approval, are employed to determine compliance with technical regulations and standards.

The TBT agreement encourages members to give positive consideration to accepting as equivalent technical regulations of other members, even if those regulations differ from their own, provided they are satisfied that these regulations adequately fulfil the objectives of their own regulations. Under this provision, members retain the discretion to determine whether equivalence will adequately satisfy their legitimate regulatory objectives. This will lead to smoother trade of goods produced by the agribusiness as well (Pant, 2004).

Trade-related Aspects of Intellectual Property Rights

Agreement on Trade-related Aspects of Intellectual Property Rights (TRIPS) protects and enforces the rights of inventors. Its main objectives are promoting, transferring and disseminating the technological innovations for the advantages of both producers and users. In agriculture sector, it includes definition of specific genes and modifications of genes. Such genes modified organisms are commonly called genetically modified organisms (GMO). The major concern in TRIPS concerning to agriculture is the exploitation of plant genetic resources and claiming rights for gene spliced plants and animals. Therefore, seed related the TRIPS would affect businesses and seed using businesses in one or the other way (Pant, 2007).

Lastly, the TRIPS agreement provides that geographical indications, hitherto used for alcoholic products, can be used in agribusiness as well. This will be helpful to create and protect agricultural products having geographical reputations like Ilam tea, Sindhuli junar, Marpha apple, Nepal honey, etc. Such reputations of the names of the place will be protected by the law and produce of other geographic areas and foreign countries cannot use such indications.

General Agreement on Trade in services (GAT)

This Agreement opens the door for foreign investment in different sectors while accessing the membership of WTO by Nepal, had agreed on 74 different sectors and sub-sectors for trade out of which only 2 sectors are related to agriculture. They include

\begin{enumerate}
\def\labelenumi{\arabic{enumi}.}
\tightlist
\item
  Animal medical services
\item
  Technical Experiment and investigation services
\end{enumerate}

\hypertarget{wto-and-emerging-issues-in-nepalese-agriculture}{%
\subsection{WTO and Emerging Issues in Nepalese Agriculture}\label{wto-and-emerging-issues-in-nepalese-agriculture}}

Keeping eyes on the benefits that can arise from the agreements of WTO, we need to address major issues in agricultural sector. The issues at hand include scale sensitivity, trade restriction at disguise, resource mobility and technology transfer (Pant, 2004 \& 2007).

\begin{enumerate}
\def\labelenumi{\arabic{enumi}.}
\tightlist
\item
  Scale Sensitivity of WTO Provisions
\end{enumerate}

It is usually believed that the WTO Agreement on Agriculture (AOA) does not focus on addressing the development needs and concerns of the small-scale subsistence farmers in developing countries. It rather tends to emphasize on commercial agriculture and trade. For example, rules under WTO say that the fees and charges on the services from the public sector should match the amount of the service provided and not be based on the volume of transactions.

\begin{enumerate}
\def\labelenumi{\arabic{enumi}.}
\setcounter{enumi}{1}
\tightlist
\item
  Disguised Protection in the Name of Quality Standard
\end{enumerate}

The AOA commits reductions in protection measures and trade distorting subsidies. The experience shows that the agricultural trade has become more protected in the developed countries after the establishment of WTO. In Nepal the agriculture is exposed to market forces in the name of economic liberalization.

\begin{enumerate}
\def\labelenumi{\arabic{enumi}.}
\setcounter{enumi}{2}
\tightlist
\item
  Competitiveness
\end{enumerate}

The major problem in international competition is our small scale and high cost of production. Low technology production makes our product less competitive in terms of quality. In addition, our trading partners may prevent entry of our products in the name of protecting animal, plant and human health and life in their country. We have competitiveness in the products with low capital and high labour inputs. Some of such products are, for example, hybrid vegetable seeds, flower seeds, medicinal herbs, silk, honey, dry fruits and cottage cheese.

\begin{enumerate}
\def\labelenumi{\arabic{enumi}.}
\setcounter{enumi}{3}
\tightlist
\item
  Labour Mobility
\end{enumerate}

Nepal's agriculture sector is labour surplus. As the surplus labour comes out of it, farms will experience labour shortage. This will have two effects. First, wage rate in the farm will increase. That will not only make agriculture workers better of but also make technological interventions like mechanization financially viable to adopt. Second, some farms will not be able to afford for higher wage or mechanization. Such financially non-viable farms need to go for enterprise transformation for increasing productivity. This will increase the efficiency of the farm sector.

\begin{enumerate}
\def\labelenumi{\arabic{enumi}.}
\setcounter{enumi}{4}
\tightlist
\item
  Land Mobility
\end{enumerate}

There are two issues in land mobility. First, moving land from low productive enterprises to high productive enterprises is experienced for centuries. The speed is gaining in recent years. Some examples of such movements of land (generally referred as change in cropping pattern) are as follows.
- maize/millet farming upland is getting shifted to summer vegetables, ginger and cardamom
- fallow lands during winter is moved to winter vegetables
- pasture lands in hills are shifter to tea, coffee and fruits

\begin{enumerate}
\def\labelenumi{\arabic{enumi}.}
\setcounter{enumi}{5}
\tightlist
\item
  Technology Transfer in Agribusiness
\end{enumerate}

The TRIPS measure is expected to accelerate the pace of generation of agricultural technology. However, the royalty rights of the inventors will make the technology costlier to the users. On the other hand, the agribusiness can make better choice of technology for adoption and it will increase the productive efficiency and net gain of the agribusiness.

Increase in the price of agricultural products due to reduction in the farm subsidies in developed countries will increase the demand of agricultural products from developing countries and is expected to facilitate investment in agriculture in LDCs. It ultimately will help in the modernization of agriculture sector.

\begin{enumerate}
\def\labelenumi{\arabic{enumi}.}
\setcounter{enumi}{6}
\tightlist
\item
  Service Sector Openings
\end{enumerate}

Nepal has opened some services to foreign investment with some conditions like compulsory incorporation of certain portion of equity to domestic investors, and employment of local staff except a certain fraction of high level experts and managers. Among the services opened, those directly concerned with the agribusiness are (a) veterinary services, (b) research and development services and (c) technical testing and analysis services. Opening of these services is expected to attract foreign direct investment in such agribusiness. In addition, availability of the better services will help domestic agribusinesses to expand. It will also help promote the product quality for export.

\begin{enumerate}
\def\labelenumi{\arabic{enumi}.}
\setcounter{enumi}{7}
\tightlist
\item
  Effects of Policy Changes Abroad
\end{enumerate}

Reductions in the domestic and export subsidies on agriculture in developed and developing counties is expected to increase the international prices of food items. In fact the increase in food prices will encourage the private investment in agricultural sector to increase its productivity.

Conclusions

The major provisions of WTO relating to the agribusiness are in the Agreement on Agriculture, SPS/TBT agreements and TRIPS agreements. Other agreements of WTO affect the agribusiness indirectly. Several provisions of these agreements can be used to support agricultural production, marketing and trade. For this purpose, we need to revise our domestic policies in standards setting and certification, increasing the production scale and making our resources mobile.

Nepal is concerned with the food security and safeguarding rural employment, for which we need some flexibility under the provisions for domestic support. We are also questioning the extremely high subsidies and tariff walls even now being maintained by the developed countries, although they are committed to reduction of both under the Uruguay round. We are seeking better market access for our agricultural products for integration to multilateral trading system.

Because of the large dependency on agriculture for employment, even minor changes in agricultural employment opportunities, commonly prices or trade conditions can have major socio-economic ramifications in Nepal. To comply with the principles of WTO without compromising our development needs and livelihood of the farmers we need some additional facilities like
- Maximum improvement of opportunities and terms of access (like duty free quota free access to the markets in developed countries) for agricultural products of our production potential
- Meaningful and practical special and differential treatment to LDCs to enable them to take appropriate domestic policy measures to address growth and development needs in agriculture; and
- In the meantime, it is absolutely necessary to protect the resource poor subsistence farmers from surges of cheap subsidized imports.

\hypertarget{treaties-and-conventions}{%
\section{Treaties and conventions}\label{treaties-and-conventions}}

\hypertarget{rio-convention-un-conference-on-environment-and-development-unced}{%
\subsection{Rio convention (UN Conference on Environment and Development, UNCED)}\label{rio-convention-un-conference-on-environment-and-development-unced}}

\begin{itemize}
\tightlist
\item
  Relates to three conventions, which are results of the Earth Summit held in Rio de Janeiro in 1992 (June 3rd to 14th).
\item
  The convention documented the following:

  \begin{itemize}
  \tightlist
  \item
    Agenda 21 (a non-binding action plan of the UN promoting sustainable development),
  \item
    The statement of Forest Principles,
  \item
    The Rio Declaration on Environment and Development
  \end{itemize}

  \begin{enumerate}
  \def\labelenumi{\arabic{enumi}.}
  \setcounter{enumi}{61}
  \tightlist
  \item
    Following conventions were formed:
  \end{enumerate}

  \begin{itemize}
  \tightlist
  \item
    UNFCCC, UN Framework Convention on Climate Change
  \item
    CBD, Convention on Biological Diversity
  \item
    UNCCD, UN Convention to Combat Desertification
  \end{itemize}
\end{itemize}

\hypertarget{the-cbd}{%
\subsection{The CBD}\label{the-cbd}}

With 196 ratified parties, the CBD aims to conserve and protect biodiversity, biological resources and safeguard life on Earth, as an integral part of economic and social development. Considering biological diversity as a global asset to current and future generations and populations across the planet, the Convention works to prevent species extinction and maintain protected habitats. As well, the CBD promotes the sustainable use of components of biological diversity, and works to maintain the environmental and sustainable process of access and benefit sharing, derived from genetic resource use.

The convention was established on December 29th, 1993. It has following objectives:
1. The conservation of biological diversity
2. The sustainable use of components of biological diversity
3. The fair and equitable sharing of benefits arising out of the utilization of genetic resources.

The CBD currently follows the Strategic Plan for Biodiversity 2011-2020 and its Aichi Biodiversity Targets, used as a vehicle to maintain synergies at National levels. Its mission is to "take effective and urgent action to halt the loss of biodiversity to ensure that by 2020, ecosystems are resilient and continue to provide essential services, thereby securing the planet's variety of life, and contributing to human well-being, and poverty eradication.

\hypertarget{the-unfccc}{%
\subsection{The UNFCCC}\label{the-unfccc}}

With 197 ratified parties, the United Nations Framework Convention on Climate Change is committed to the objective of ``{[}stabilizing{]} greenhouse gas concentrations in the atmosphere at a level that would prevent dangerous anthropogenic interference with the climate system. Such a level should be achieved within a time frame sufficient to allow ecosystems to adapt naturally to climate change, to ensure that food production is not threatened and to enable economic development to proceed in a sustainable manner.'' Following the adoption of the Paris Agreement in 2015 and previously the Kyoto Protocol in 1997, the UNFCCC Secretariat works to maintain the goals and objectives of the Convention, as the primary United Nations body whose role functions to address the threat of climate change.

\begin{itemize}
\tightlist
\item
  The Paris Agreement, 2015:
\end{itemize}

\hypertarget{the-unccd}{%
\subsection{The UNCCD}\label{the-unccd}}

An international agreement that ties the sustainability of land management and the issues of land degradation to the environment. Among the areas of consideration, the Convention focuses on restoring degraded ecosystems found in dryland areas. The United Nations Convention to Combat Desertification (UNCCD), consisting of 197 parties works towards creating `a future that avoids, minimizes, and reverses desertification/land degradation and mitigates the effects of drought in affected areas at all levels.'

Legislatively, the UNCCD is committed to achieving Land Degradation Neutrality (LDN) and combat pressing environmental issues of Desertification, land degradation and drought (DLDD) through a newly created 2018-2030 Strategic Framework, consistent with the 2030 Agenda for Sustainable Development. This framework follows the 10-year strategic plan and framework for 2008-2018 that aimed to establish global partnerships in working toward the reversal and prevention of desertification and land degradation. The UNCCD aims to restore the productivity of degraded land, while improving livelihoods and aiding populations that are vulnerable because of environmental destruction. ``The Convention's 197 parties work together to improve the living conditions for people in drylands, to maintain and restore land and soil productivity, and to mitigate the effects of drought.''

\hypertarget{rio-20-summit-rio-earth-summit-2012-june-13th-june-22nd}{%
\subsection{Rio +20 Summit (Rio Earth Summit, 2012; June 13th-June 22nd)}\label{rio-20-summit-rio-earth-summit-2012-june-13th-june-22nd}}

The issues addressed included:

\begin{itemize}
\tightlist
\item
  systematic scrutiny of patterns of production -- particularly the production of toxic components, such as lead in gasoline, or poisonous waste including radioactive chemicals
\item
  alternative sources of energy to replace the use of fossil fuels which delegates linked to global climate change
\item
  new reliance on public transportation systems in order to reduce vehicle emissions, congestion in cities and the health problems caused by polluted air and smoke
\item
  the growing usage and limited supply of water
\item
  UNFCCC led to Kyoto Protocol and the Paris Agreement.
\end{itemize}

\hypertarget{nagoya-protocol}{%
\subsection{Nagoya Protocol}\label{nagoya-protocol}}

The Nagoya Protocol on Access to Genetic Resources and the Fair and Equitable Sharing of Benefits Arising from their Utilization to the Convention on Biological Diversity, also known as the Nagoya Protocol on Access and Benefit Sharing (ABS) is a 2010 supplementary agreement to the 1992 Convention on Biological Diversity (CBD). Its aim is the implementation of one of the three objectives of the CBD: the fair and equitable sharing of benefits arising out of the utilization of genetic resources, thereby contributing to the conservation and sustainable use of biodiversity. However, there are concerns that the added bureaucracy and legislation will, overall, be damaging to the monitoring and collection of biodiversity, to conservation, to the international response to infectious diseases, and to research.

The protocol was adopted on 29 October 2010 in Nagoya, Japan, and entered into force on 12 October 2014. It has been ratified by 114 parties, which includes 113 UN member states and the European Union. It is the second protocol to the CBD; the first is the 2000 Cartagena Protocol on Biosafety.

The Nagoya protocol applies to genetic resources that are covered by the CBD, and to the benefits arising from their utilization. The Nagoya Protocol sets out obligations for its contracting parties to take measures in relation to access to genetic resources, benefit-sharing and compliance. Implementation of the protocol deals with follwing major points:

\begin{itemize}
\tightlist
\item
  The Nagoya Protocol's success will require effective implementation at the domestic level. A range of tools and mechanisms provided by the Nagoya Protocol will assist contracting parties including:
\end{itemize}

\begin{enumerate}
\def\labelenumi{\arabic{enumi}.}
\tightlist
\item
  Establishing national focal points (NFPs) and competent national authorities (CNAs) to serve as contact points for information, grant access, or cooperate on issues of compliance
\item
  An Access and Benefit-sharing Clearing-House to share information, such as domestic regulatory ABS requirements or information on NFPs and CNAs
\item
  Capacity-building to support key aspects of implementation.
\end{enumerate}

\hypertarget{the-ramsar-convention}{%
\subsection{The Ramsar Convention}\label{the-ramsar-convention}}

Abbreviated form for ``The Ramsar Convention on Wetlands of International Importance especially as Waterfowl Habitat''; a.k.a. Convention on Wetlands)

It is an international treaty for conservation and sustainable use of wetlands. It is named after the city of Ramsar, Iran where convention was signed in 1971.

Every three years, representatives of the Contracting Parties meet as the conference of the Contracting Parties (COP), the policy-making organ of the Convention which adopts decisions (Resolutions and Recommendations) to administer the work of the Convention and improve the way in which the Parties are able to implement its objectives. COP12 was held in Punta del Este, Uruguay, in 2015. COP13 was held in Dubai, United Arab Emirates, in October 2018.

List of Wetlands of International Importance include 2331 Ramsar sites in May 2018 covering over 2.1 million sqkm. The country with highest number of Sites is the UK with 170, and the country with the greatest area of listed wetlands is Bolivia, with over 140, 000 sqkm.

The Ramsar Convention works closely with six other organisations known as International Organization Partners (IOPs). These are:
- Birdlife International
- International Union for Conservation of Nature (IUCN)
- International Water Management Institute (IWMI)
- Wetlands International
- WWF International
- Wildfowl \& Wetlands Trust (WWT)

These organizations support the work of the Convention by providing expert technical advice, helping implement field studies, and providing financial support. The IOPs also participate regularly as observers in all meetings of the Conference of the Parties and as full members of the Scientific and Technical Review Panel.

Following bodies are established by the Convention: 1. Conference of contracting parties (COP), 2. The Standing Committee, 3. The Scientific and Technical Review Panel (STRP), 4. The Secretariat.

The secretariat is based at the headquarters of the IUCN, Gland, Switzerland. Martha Rojas Urrego is the sixth secretary of the Ramsar Convention on wetlands.

The 2nd of February each year is World Wetlands Day, marking the date of the adoption of the Convention on Wetlands on 2 February 1971. WWD was celebrated for the first time in 1997 and has grown remarkably since then. In 2015 World Wetlands Day was celebrated in 59 countries.

\hypertarget{cartagena-protocol-on-biosafety-to-the-convention-on-biological-diversity}{%
\subsection{Cartagena Protocol on Biosafety to the Convention on Biological Diversity}\label{cartagena-protocol-on-biosafety-to-the-convention-on-biological-diversity}}

Drafted: 29 Jan, 2000; Signed: 15 May, 2000; Location: Montreal, Quebec; Effective: 11 September, 2003; Signatories 103; Parties: 171; Depositary: Secretary-General of the United Nations -- Record retrieved: September, 2019

It is supplement to the Convention on Biological Diversity effective since 2003. The Biosafety Protocol seeks to protect biological diversity from the potential risks posed by genetically modified organisms resulting from modern biotechnology.

The Biosafety Protocol makes clear that products from new technologies must be based on the precautionary principle and allow developing nations to balance public health against economic benefits. It will for example let countries ban imports of genetically modified organisms if they feel there is not enough scientific evidence that the product is safe and requires exporters to label shipments containing genetically altered commodities such as corn or cotton.

The required number of 50 instruments of ratification/accession/approval/acceptance by countries was reached in May 2003. In accordance with the provisions of its Article 37, the Protocol entered into force on 11 September 2003. As of February 2018, the Protocol had 171 parties, which includes 168 United Nations member states, the State of Palestine, Niue, and the European Union.
The precautionary approach is contained in Principle 15 of the Rio Declaration on Environment and Development.

The main features of the Cartagena Protocol on Biosafety are:
1. Promotes biosafety by establishing rules and procedures for the safe transfer, handling, and use of LMOs, with specific focus on transboundary movements of LMOs.
2. It features a set of procedures including one for LMOs that are to be intentionally introduced into the environment called the advance informed agreement procedure (AIA), and one for LMOs that are intended to be used directly as food or feed or for processing.
3. Parties to the Protocol must ensure that LMOs are handled, packaged and transported under conditions of safety.
4. The shipment of LMOs subject to transboundary movement must be accompanied by appropriate documentation specifying, among other things, identity of LMOs and contact point for further information.
These procedures and requirements are designed to provide importing Parties with the necessary information needed for making informed decisions about whether or not to accept LMO imports and for handling them in a safe manner.

The Party of import makes its decisions in accordance with scientifically sound risk assessments. The Protocol sets out principles and methodologies on how to conduct a risk assessment. In case of insufficient relevant scientific information and knowledge, the Party of import may use precaution in making their decisions on import. Parties may also take into account, consistent with their international obligations, socio-economic considerations in reaching decisions on import of LMOs.

Parties must also adopt measures for managing any risks identified by the risk assessment, and they must take necessary steps in the event of accidental release of LMOs.

To facilitate its implementation, the Protocol establishes a Biosafety Clearing-House for Parties to exchange information, and contains a number of important provisions, including capacity-building, a financial mechanism, compliance procedures, and requirements for public awareness and participation. (Article 20 of the Protocol, SCBD 2000). The Biosafety Clearing-House was established in a phased manner, and the first meeting of the Parties approved the transition from the pilot phase to the fully operational phase, and adopted modalities for its operations (Decision BS-I/3, SCBD 2004).

\hypertarget{the-kyoto-protocol}{%
\subsection{The Kyoto Protocol}\label{the-kyoto-protocol}}

Signed: 11 December, 1997; Location: Kyoto, Japan; Effective: 16 February, 2005, Condition: Ratificaiton by at least 55 states to the convention; Expiration: In force (first commitment period expired 31 December, 2012; Signatories: 84; Parties: 192)

This Protocol extends the 1992 UNFCCC that commits state parties to reduce GHG emissions, based on the scientific consensus that global warming is occurring and it extremely likely that human-made CO2 emissions have predominantly caused it.

The Kyoto Protocol applies to the six greenhouse gases listed in Annex A: Carbon dioxide (CO2), Methane (CH4), Nitrous oxide (N2O), Hydrofluorocarbons (HFCs), Perfluorocarbons (PFCs), and Sulphur hexafluoride (SF6).

Canada withdrew from the Protocol in December, 2012. The protocol's second commitment period was agreed in 2012, known as Doha Amendment to the Kyoto Protocol, in which 37 countries have binding targets. As of September 2019, 132 states have accepted the Doha Amendment, while entry into force requires the acceptance of 144 states. Of the 37 countries with binding commitments, 7 have ratified.

Negotiations were held in the framework of the yearly UNFCCC Climate Change Conferences on measures to be taken after the second commitment period ends in 2020. This resulted in the 2015 adoption of the Paris Agreement, which is a separate instrument under the UNFCCC rather than an amendment of the Kyoto Protocol.

On 8 December 2012, at the end of the 2012 United Nations Climate Change Conference, an agreement was reached to extend the Protocol to 2020 and to set a date of 2015 for the development of a successor document, to be implemented from 2020. The outcome of the Doha talks has received a mixed response, with small island states critical of the overall package. The Kyoto second commitment period applies to about 11\% of annual global emissions of greenhouse gases. Other results of the conference include a timetable for a global agreement to be adopted by 2015 which includes all countries. At the Doha meeting of the parties to the UNFCCC on 8 December 2012, the European Union chief climate negotiator, Artur Runge-Metzger, pledged to extend the treaty, binding on the 27 European Member States, up to the year 2020 pending an internal ratification procedure.

Ban Ki Moon, Secretary General of the United Nations, called on world leaders to come to an agreement on halting global warming during the 69th Session of the UN General Assembly on 23 September 2014 in New York. UN member states have been negotiating a future climate deal over the last five years. A preliminary calendar was adopted to confirm ``national contributions'' to the reduction of CO2 emissions by 2015 before the UN climate summit which was held in Paris at the 2015 United Nations Climate Change Conference.

\hypertarget{world-conference-on-women-1975}{%
\subsection{World conference on Women, 1975}\label{world-conference-on-women-1975}}

The conference was held between 19 June and 2 July 1975 in Mexico City, Mexico. It was the first international conference held by the United Nations to focus solely on women's issues and marked a turning point in policy directives. After this meeting, women were viewed as part of the process to develop and implement policy, rather than recipients of assistance. The conference was one of the events established for International Women's Year and led to the creation of both the United Nations Decade for Women and follow-up conferences to evaluate the progress that had been made in eliminating discrimination against women and their equality. Two documents were adopted from the conference proceedings, the World Plan of Action which had specific targets for nations to implement for women's improvement and the Declaration of Mexico on the Equality of Women and Their Contribution to Development and Peace, which discussed how nations foreign policy actions impacted women. It also led to the establishment of the International Research and Training Institute for the Advancement of Women to track improvements and continuing issues and the United Nations Development Fund for Women to provide funding for developmental programs. The conference marked the first time that the parallel Tribune meeting was successful in submitting input to the official meeting and became a catalyst for women's groups to form throughout the world.

\hypertarget{trade-agreements}{%
\section{Trade agreements}\label{trade-agreements}}

The Nagoya protocol (refer to Section on Treaties and Conventions) relates to Preferential Trade Agreements (PTAs), which include provisions relating to access to genetic resources or to the sharing of the benefits that arise out of their utilization.

\begin{itemize}
\tightlist
\item
  PTA is a trading bloc that gives preferential access to certain products from the participating countries. This is done by reducing tariff but not by abolishing them completely. A PTA can be established through a trade pact. It is the first stage of economic integration. The line between a PTA and a Free Trade Area (FTA) may be blurred, as almost any PTA has a main goal of becoming a FTA in accordance with the General Agreement on Tariffs and Trade.
\item
  These tariff preference have created numerous departures from the normal trade relations principle, that WTO members should apply the same tariff to imports from other WTO members.
\item
  With the recent multiplication of bilateral PTAs and the emergence of Mega-PTAs (wide regional trade agreements such as the Transatlantic Trade and Investment Partnership; TTIP) or Trans Pacific Partnership (TPP)), a global trade system exclusively managed within the framework of the WTO now seems unrealistic and the interactions between trade systems have to be taken into account. The increased complexity of the international trade system generated by the multiplication of PTAs should be taken into account in the study of the choice of for a used by countries or regions to promote their trade relations and environmental agenda.
\end{itemize}

A free trade area is basically a preferential trade area with increased depth and scope of tariffs reduction. All free trade areas, customs unions, common markets, economic unions, customs and monetary unions and economic and monetary unions are considered advanced forms of a PTA, but these are not listed below.

Multilateral

\begin{itemize}
\tightlist
\item
  Economic Cooperation Organization (ECO) (1992)
\item
  Generalized System of Preferences
\item
  Global System of Trade Preferences among Developing Countries (GSTP) (1989)
\item
  Latin American Integration Association (LAIA/ALADI) (1981){[}3{]}
\item
  Melanesian Spearhead Group (MSG) (1994)
\item
  Protocol on Trade Negotiations (PTN) (1973)
\item
  South Pacific Regional Trade and Economic Cooperation Agreement (SPARTECA) (1981){[}4{]}
\end{itemize}

Bilateral

Several hundred bilateral PTAs have been signed since the early 20th century. The TREND project of the Canada Research Chair in International Political Economy lists around 700 trade agreements, the vast majority of which are bilateral.

\begin{itemize}
\tightlist
\item
  European Union - ACP countries, formerly via the trade aspects of the Cotonou Agreement, later via Everything But Arms (EBA) agreements
\item
  India -- Afghanistan (2003)
\item
  India -- Mauritius
\item
  India -- Nepal (2009)
\item
  India -- Chile (2007)
\item
  India -- MERCOSUR (2009)
\item
  ASEAN -- PR China (2005)
\item
  Laos -- Thailand (1991)
\end{itemize}

\hypertarget{most-favoured-nation-mfn}{%
\section{Most favoured nation (MFN)}\label{most-favoured-nation-mfn}}

It is a status or level of treatment accorded by one state to another in international trade. The term means the country which is the recipient of this treatment must nominally receive equal trade advantages as the ``most favoured nation'' by the country granting such treatment (trade advantages include low tariffs or high import quotas). In effect, a country that has been accorded MFN status may not be treated less advantageously than any other country with MFN status by the promising country. There is a debate in legal circles whether MFN clauses in bilateral investment treaties include only substantive rules or also procedural protections. The members of the World Trade Organization (WTO) agree to accord MFN status to each other. Exceptions allow for preferential treatment of developing countries, regional free trade areas and customs unions.

Trade experts consider MFN clauses to have the following benefits:

\begin{itemize}
\tightlist
\item
  Increases trade creation and decreases trade diversion. A country that grants MFN on imports will have its imports provided by the most efficient supplier if the most efficient supplier is within the group of MFN. Otherwise, that is, if the most efficient producer is outside the group of MFN and additionally, is charged higher rates of tariffs, then it is possible that trade would merely be diverted from this most efficient producer to a less efficient producer within the group of MFN (or with a tariff rate of 0). This leads to economic costs for the importing country, which can outweigh the gains from free trade.
\item
  MFN allows smaller countries, in particular, to participate in the advantages that larger countries often grant to each other, whereas on their own, smaller countries would often not be powerful enough to negotiate such advantages by themselves.
\item
  Granting MFN has domestic benefits: having one set of tariffs for all countries simplifies the rules and makes them more transparent. Theoretically, if all countries in the world confer MFN status to each other, there will be no need to establish complex and administratively costly rules of origin to determine which country a product (that may contain parts from all over the world) must be attributed to for customs purposes. However, if at least one nation lies outside the MFN alliance, then customs cannot be done away with.
\item
  MFN restrains domestic special interests from obtaining protectionist measures. For example, butter producers in country A may not be able to lobby for high tariffs on butter to prevent cheap imports from developing country B, because, as the higher tariffs would apply to every country, the interests of A's principal ally C might get impaired.
\item
  As MFN clauses promote non-discrimination among countries, they also tend to promote the objective of free trade in general.
  MFN: A case of India
\end{itemize}

As per the obligation under the World Trade Organization (WTO), the member countries of WTO shall extend Most Favored Nation (MFN) status to each other automatically, unless otherwise specified in the agreement or schedule notified to the WTO by the member country. Pursuant to this Provision, in case of goods, India has extended MFN status to member countries of WTO. As regards SAARC countries, Bangladesh, Maldives, Nepal, Pakistan and Sri Lanka are members of WTO and except the Islamic Republic of Pakistan, these countries have extended MFN status to India. India has extended MFN status to all these SAARC countries including Pakistan. So far as exception to MFN status, if any, in services is concerned, each member country has indicated the same in the schedule of commitments in services notified to WTO. The mfn status was finally withdrawn for pakistan on 15 February 2019 in response to the Pulwama attack.

\hypertarget{bimstec}{%
\section{BIMSTEC}\label{bimstec}}

\hypertarget{millennium-development-goals-mdg}{%
\section{Millennium development goals (MDG)}\label{millennium-development-goals-mdg}}

Millennium Development Goals (MDG) Progress Report (NPC, 2010) stated that the poverty rate has decreased, a reduction in the population suffering chronic food insecurity, and reduced unemployment rates, and there is a positive trend towards gender equality as indicated by a balanced enrollment of girls and boys in primary schools. The Report notes there has been increased allocation of public resources in favor of marginalized groups in remote areas, and greater attention has been given to environmental conservation and adaptation to climate change. The policy environment for achieving most of the MDG targets seems favorable with the Interim Constitution of Nepal 2063 (2007) and its subsequent laws emphasizing inclusive, participatory and decentralized governance.

\begin{longtable}{l}
\toprule
Goals and targets from the millennium declaration\\
\midrule
Goal 1: Eradicate extreme poverty and hunger\\
Goal 2:  Achieve universal primary education\\
Goal 3:  Promote gender equality and empower women\\
Goal 4:  Reduce child mortality\\
Goal 5:  Improve maternal health\\
\addlinespace
Goal 6:  Combat HIV/AIDS, malaria and other diseases\\
Goal 7:  Ensure environmental sustainability\\
Goal 8:  Develop a global partnership for development\\
\bottomrule
\end{longtable}

\begin{longtable}{>{\raggedright\arraybackslash}p{33em}l}
\toprule
Goals and targets from the millennium declaration & Goal num\\
\midrule
Target 1:   Halve, between 1990 and 2015, the proportion of people whose income is less than one dollar a day & Goal 1\\
Target 2: Halve, between 1990 and 2015, the proportion of people who suffer from hunger & Goal 1\\
Target 3:   Ensure that, by 2015, children everywhere, boys and girls alike, will be able to complete a full course of primary schooling & Goal 2\\
Target 4:   Eliminate gender disparity in primary and secondary education preferably by 2005 and to all levels of education no later than 2015 & Goal 3\\
Target 5:   Reduce by two-thirds, between 1990 and 2015, the under-five mortality rate & Goal 4\\
\addlinespace
Target 6:   Reduce by three-quarters, between 1990 and 2015, the maternal mortality ratio & Goal 5\\
Target 7:   Have halted by 2015 and begun to reverse the spread of HIV/AIDS & Goal 6\\
Target 8:   Have halted by 2015 and begun to reverse the incidence of malaria and other major diseases & Goal 6\\
Target 9:   Integrate the principles of sustainable development into country policies and programmes and reverse the loss of environmental resources & Goal 7\\
Target 10: Halve, by 2015, the proportion of people without sustainable access to safe drinking water & Goal 7\\
\addlinespace
Target 11 By 2020, to have achieved a significant improvement in the lives of at least 100 million slum dwellers & Goal 7\\
Target 12: Develop further an open, rule-based, predictable, non-discriminatory trading and financial system
Includes a commitment to good governance, development, and poverty reduction – both nationally and internationally & Goal 8\\
Target 13: Address the special needs of the least developed countries
Includes: tariff and quota free access for least developed countries' exports; enhanced programme of debt relief for HIPC and cancellation of official bilateral debt; and more generous ODA for countries committed to poverty reduction & Goal 8\\
Target 14: Address the special needs of landlocked countries and small island developing States
(through the Programme of Action for the Sustainable Development of Small Island Developing States and the outcome of the twenty-second special session of the General Assembly) & Goal 8\\
Target 15: Deal comprehensively with the debt problems of developing countries through national and international measures in order to make debt sustainable in the long term & Goal 8\\
\addlinespace
Target 16: In co-operation with developing countries, develop and implement strategies for decent and productive work for youth & Goal 8\\
Target 17: In co-operation with pharmaceutical companies, provide access to affordable, essential drugs in developing countries & Goal 8\\
Target 18: In co-operation with the private sector, make available the benefits of new technologies, especially information and communications & Goal 8\\
\bottomrule
\end{longtable}

\begin{longtable}{>{\raggedright\arraybackslash}p{33em}l}
\toprule
Indicators for monitoring progress & Goal num\\
\midrule
1.    Proportion of population below \$1 (PPP) per daya & Goal 1\\
2.    Poverty gap ratio [incidence x depth of poverty] & Goal 1\\
3.    Share of poorest quintile in national consumption & Goal 1\\
4.    Prevalence of underweight children under-five years of age & Goal 1\\
5.    Proportion of population below minimum level of dietary energy consumption & Goal 1\\
\addlinespace
6.    Net enrolment ratio in primary education & Goal 2\\
7.    Proportion of pupils starting grade 1 who reach grade 5 & Goal 2\\
8.    Literacy rate of 15-24 year-olds & Goal 2\\
9.    Ratios of girls to boys in primary, secondary and tertiary education & Goal 3\\
10.  Ratio of literate females to males of 15-24 year-olds & Goal 3\\
\addlinespace
11.  Share of women in wage employment in the non- agricultural sector & Goal 3\\
12.  Proportion of seats held by women in national parliament & Goal 3\\
13.  Under-five mortality rate & Goal 4\\
14.  Infant mortality rate & Goal 4\\
15.  Proportion of 1 year-old children immunised against measles & Goal 4\\
\addlinespace
16.  Maternal mortality ratio & Goal 5\\
17.  Proportion of births attended by skilled health personnel & Goal 5\\
18.  HIV prevalence among 15-24 year old pregnant women & Goal 6\\
19.  Condom use rate of the contraceptive prevalence rateb & Goal 6\\
20.  Number of children orphaned by HIV/AIDSc & Goal 6\\
\addlinespace
21.  Prevalence and death rates associated with malaria & Goal 6\\
22.  Proportion of population in malaria risk areas using effective malaria prevention and treatment measuresd & Goal 6\\
23.  Prevalence and death rates associated with tuberculosis & Goal 6\\
24.  Proportion of tuberculosis cases detected and cured under directly observed treatment short course (DOTS) & Goal 6\\
25.  Proportion of land area covered by forest & Goal 7\\
\addlinespace
26.  Ratio of area protected to maintain biological diversity to surface area & Goal 7\\
27.  Energy use (kg oil equivalent) per \$1 GDP (PPP) & Goal 7\\
28.  Carbon dioxide emissions (per capita) and consumption of ozone-depleting CFCs (ODP tons) & Goal 7\\
29.  Proportion of population using solid fuels & Goal 7\\
30.  Proportion of population with sustainable access to an improved water source, urban and rural & Goal 7\\
\addlinespace
31.  Proportion of urban population with access to improved sanitation & Goal 7\\
32.  Proportion  of  households  with  access  to  secure  tenure (owned or rented) & Goal 7\\
33.  Net ODA, total and to LDCs, as percentage of OECD/DAC donors’ gross national income & Goal 8\\
34.  Proportion of total bilateral, sector-allocable ODA of OECD/DAC donors to basic social services (basic education, primary health care, nutrition, safe water and sanitation) & Goal 8\\
35.  Proportion of bilateral ODA of OECD/DAC donors that is untied & Goal 8\\
\addlinespace
36.  ODA received in landlocked countries as proportion of their GNIs & Goal 8\\
37.  ODA received in small island developing States as proportion of their GNIs
Market access & Goal 8\\
38.  Proportion of total developed country imports (by value and excluding arms) from developing countries and LDCs, admitted free of duties & Goal 8\\
39. Average tariffs imposed by developed countries on agricultural products and textiles and clothing from developing countries & Goal 8\\
40.  Agricultural support estimate for OECD countries as percentage of their GDP & Goal 8\\
\addlinespace
41.  Proportion of ODA provided to help build trade capacitye
Debt sustainability & Goal 8\\
42.  Total number of countries that have reached their HIPC decision points and number that have reached their HIPC completion points (cumulative) & Goal 8\\
43.  Debt relief committed under HIPC initiative, US\$ & Goal 8\\
44.  Debt service as a percentage of exports of goods and services & Goal 8\\
45.  Unemployment rate of 15-24 year-olds, each sex and totalf & Goal 8\\
\addlinespace
46.  Proportion of population with access to affordable essential drugs on a sustainable basis & Goal 8\\
47.  Telephone lines and cellular subscribers per 100 population & Goal 8\\
48.  Personal computers in use per 100 population and Internet users per 100 population & Goal 8\\
\bottomrule
\end{longtable}

\hypertarget{sustainable-development-goals}{%
\section{Sustainable development goals}\label{sustainable-development-goals}}

Chapter I: Introduction

Nepal, a young democracy and the youngest federal democratic republic, has been passing through a protracted political transition for more than a decade. The country, as one of the 48 least developed countries of the world with per capita income of about US\$ 850 in 2017, is recovering from a decade long violent armed conflict, and passing through a stage of peace building, social reconciliation and economic revival.

Nepal has made significant progress in poverty reduction and human development in the last two decades - absolute poverty declined by one percentage point each year and HDI improved by one basis point per year. Still, absolute poverty at 21.6 percent is among the highest in South Asia; and the country is at the bottom of the countries with middle human development status.

Political and social strife, prolonged political transition, and instability of the government has deteriorated investment climate, suppressed agricultural activities, and undermined the expansion of service sectors like tourism and finance resulting in slow economic growth. Besides, the country has been passing through various economic and environmental risks and vulnerabilities. The massive earthquake and trade blockade in the Southern border between April and September 2015 are their manifestations.

This report is an update of the National SDGs Report prepared by NPC in 2015. It revises SDGs indicators to make them consistent with global ones, updates their baseline status, and revises targets set for 2030 wherever necessary. Consultations with NPC Thematic Task Forces for SDGs and various non-government stakeholders including the private sector and civil society organizations were held to discuss the proposed SDG indicators and their 2030 outputs.

Chapter II: Overview of MDG Outcome, and Implementation Status of SDGs

The MDG Final Status Report shows Nepal's commendable progress on reducing extreme poverty and hunger during the period of 2000--2015. Extreme poverty (defined as 1\$ - a - day) dropped from 33.5 percent of the population in 1990 to 16.4 percent in 2013 thereby achieving the target of halving poverty by 2015. Poverty gap narrowed down to 5.43 in 2011 from 7.55 in 2004 implying that poor people, on average,are now closer to rising above the poverty line than in 2004. A noteworthy achievement was made in the reduction of hunger also.

A very good progress was made in achieving universal primary education - enrollment ratio (NER) reached to 96.6 percent, survival rate to 89.4 percent and the literacy rate (15-24 years) to 88.6 percent in 2015. Gender equality was achieved at primary and secondary education levels with gender parity index (GPI) scores of 1.09 in primary and 1.0 in secondary education in 2015.

All the targets under child health - reducing infant mortality rate (IMR), reducing under-five mortality rate (U5MR), and increasing immunization against measles have been met. The IMR dropped from 64 per 1,000 live births in the year 2000 to 33 in 2015; and U5MR declined from 91 per 1000 live births in 2000 to 38 in 2015. Nepal was close to meeting the targets of reducing the maternal mortality ratio (MMR).

The country has been able to reduce the rate of biodiversity loss. Forest cover increased from less than 40 percent in 2000 to 44.7 percent of total land area in 2015. It has achieved the MDG target of halving the proportion of people without sustainable access to basic drinking water and basic sanitation as households with access to an improved source of drinking water increased from 46 percent in 1990 to 83.6 percent in 2015; and households having access to sanitation (toilets) increased from 6 percent in 1990 to 81 percent in 2015.

The MDGs Final Status Report highlights on several unfinished agenda of the MDG with particular focus on absolute poverty, malnutrition, universal school enrollment, maternal mortality, and access to reproductive health facility. Besides, the report also flags on the quality, inclusion and equality aspects of the MDGs achieved in quantitative terms - such as in education, child health, and environment. The SDGs have to meet these gaps even before embarking on making further progress.

The 14th Year Plan starting July 2016 and budget for FY 2016/17 have considered SDGs while prioritizing and allocating resources. The budget allocations in the annual programs are audited against the SDGs, also to check if the allocations are balanced across SDG areas. The budget audit of FY 2016/17 shows almost two-thirds (65 percent) of the spending gone in three infrastructure related SDG areas - namely the Goals 7, 9 and 11; 18 percent on social sectors like education, health, water supply and sanitation; and 13percent to directly address poverty and hunger (Goals 1 and 2).

In order to coordinate SDGs implementation across various sectors, the government has constituted SDGs Steering Committee under the chairmanship of the Prime Minister while SDGs working committee is constituted under VC of NPC to coordinate SDGs implementation at the national as well as sub-national levels. Thematic working groups for SDGs are formed under the convenership of Members of the NPC with secretaries of the relevant Ministries as co-conveners and other concerned agencies as members of the groups.

Chapter III: Targets and Indicators for Sustainable Development Goals

The proposed SDG 1 targets for 2030 are to reduce extreme poverty to less than five percent, reduce poverty gap to 2.8 percent, raise per capita income to US\$ 2,500,and raise social protection budget to 15 percent of total budget.

The targets for SDG 2 include reduction in prevalence of undernourishment (measure of sufficiency of access to food) to 3 percent and prevalence of underweight children under five years of age to 5 percent by 2030.

The proposed SDG 3 targets include reduction of MMR to less than 70 per 100 thousand live births by 2030 which is in line with the global target. The child health targets include reduction of preventable death of newborn and children to less than one percent. However, for overall newborn and U5 mortality rates, the targets are to reduce them from 23 and 38 per thousand live births in 2015 to 10 and 22 respectively by 2030. The other targets include almost elimination of the prevalence of HIV, TB, Malaria and other Tropical Diseases, and water borne diseases.

The major targets for SDG 4 include 99.5 percent net enrollment and completion of primary education, and 99 percent gross enrollment in secondary education by 2030. The other targets are: 95 percent of students enrolled in grade one to reach grade eight; and elimination of gender disparities in tertiary education.

The major targets for SDG 5 are elimination of wage discrimination at similar work, elimination of physical/sexual violence, eliminating all harmful practices, such as child, early and forced marriage and chhaupadi, increasing seats held by women in the national parliament to 40 percent, and increasing women' proportion in public service decision making positions by nearly 4 folds to 7.5 percent of total employees in 2030 from 2 percent in 2015.

The proposed targets for SDG 6 include basic water supply coverage to 99 percent of households and piped water supply and improved sanitation to 90 percent of households. Other targets are to free 99 percent of the communities from open defecation, to reach 95 percent of the households with improved sanitation facilities which are not shared, and to ensure 98 percent of the population using latrine.

The proposed specific targets for SDG 7 include accessibility of 99 percent households to electricity, reduction to 10 percent-from nearly 75 percent now - the households who resort to firewood for cooking, increase per capita electricity to 1500 kwh and decrease the commercial energy use per unit of GDP from 3.20 ToE/million Rs in 2015 to 3.14 ToE/million Rs in 2030.

The SDG 8 target for 2030 is to achieve per capita GDP growth of 7 percent, to maintain the growth of agriculture to about 5 percent and that of construction to 15 percent; reduce material intensity in manufacturing to 60 percent; lower underemployment to less than 10 percent; and to eliminate the worst form of child labor.

The 2030 targets for road in SDG 9 are to increase road density to 1.5 km/sq km and paved road density to 0.25 km/sq km and to connect all the districts, municipalities and village councils by road. For industries, the targets for 2030 are to increase the share of industry to 25 percent so as to promote labor intensive activities and to raise employment in manufacturing to 13percent of the total employment.

The key targets for SDG 10 are to reduce consumption inequality (index) from 0.33 in 2015 to 0.16by 2030 and to reduce income inequality from 0.46 to 0.23 at the same time period. The PALMA index is targeted to improve from 1.3 to 1 in the same period.

The 2030 target for SDG 11 is to make at least 50 percent of the urban road to be safe from global standard. Besides, the proposed specific targets for 2030 include doubling the proportion of households living in safe houses; substantially reducing air pollution, preventing the deaths and injuries due to disaster, repairing and reconstructing, by 2020, all cultural heritages destroyed by earthquake.

The SDG 12 target for land to be available for cereal production by 2030 is set at 75 percent of all cultivated land. Consumption of wood is proposed to be not more than 0.05 cubic m per capita per year by 2030 from to 0.11 cu m per capita per year in 2015. Use of plastic per capita which was 2.7 gram per day in 2015 is proposed to be almost zero by 2030. Similarly, the post-harvest loss of food is targeted to be less than 1 percent by 2030 from 15 percent in 2015.

The key targets for SDG 13 are making half the existing CO2 emission level including from transportation, industrial, and commercial sectors. Consumption of ozone depleting substance is targeted to reduce to one third of the existing level. In the meantime, climate smart villages are proposed to be 170 and climate smart farming to be 500 units by 2030 from zero at present. Almost all the schools will be covered by climate change education and the number of trained persons (local planners) for climate change planning would reach 3 thousand.

The key 2030 targets for SDG 15 are to maintain conservation area at 23.3 of the total land area, increase forest under community management from 39 percent to 42 per cent of the forest area, halt forest loss and degradation, increase mountain ecosystem covered by the protected area to 70 percent in 2030 from 68 percent in 2015, and undertake additional plantation of 5000 ha per annum.

The major targets for SDG 16 include ending death from violent conflict, violence against women, and violence against children, improve transparency and accountability score from a scale of 3 at present to 5, and good governance scale from (-) 0.78 to 2.0 in a scale of -2.5 to 2.5. The 2030 targets also include eliminating marriage before the age of 18 years, 100 percent birth registration, 80 percent voter turnout in elections, and access to justice for all.

SDG 17 is for all the stakeholders to adhere to - ranging from resource mobilization and capacity development and accountability to shared responsibility. The target for revenue collection for 2030 is set at 30 percent of GDP. Domestic expenditure financed by domestic revenue is targeted to reach 80 per cent from 76 percent in 2015. For meeting the private sector investment financing gap, foreign direct investment (inward stock) is targeted to increase to 20 percent of GDP in 2030 from less than 3 percent in 2015.

Chapter IV: Issues and Challenges in SDGs Implementation

The SDGs are comprehensive, ambitious and challenging goals and require huge resources as well as capacity enhancement to achieve them in 15 years' time. As SDGs are not stand-alone goals, achievement of one goal has implications to the achievement of several other goals.

Once the SDGs are tailored in the periodic plans and annual budgets, there is a need for annual budget audit from the SDGs perspectives. While developing a framework for such an audit is critical, a designated agency for the task has to be put in place. A SDGs dashboard could be created to provide open source information on the state of SDGs implementation and their progress.

Although SDGs are equally important, indivisible, and common for all countries, their priorities are country specific, depending upon the level, gaps and structure of development. Priority, being a relative and not an absolute term, implies that prioritization should be done in ranks such as priority one, priority two and priority three. There is no defined way to prioritize them by goals or sectors. Prioritization can better be done within the targets and indicators. Prioritization will also be guided by financing and other resource availability and donor support to specific SDGs.

As the localization of SDGs at the Provincial and Local Government levels is critical for the universal, equitable and inclusive outcome of sustainable development efforts, it is equally important to have a political set up at those levels willing and capable of handling the development agenda in an effective manner.

Containing inequality through market based policy instruments would be a formidable task; and unless pro poor growth policies and interventions along with strong distributive measures are put in place, it is hard to increase the share of bottom 40 percent in national income. Currently, the bottom 40 percent of the population holds only 12 percent of the national income and it has to be raised to 18 percent of total income if some tangible achievement is to be made in this area.

As the SDGs encompass diverse set of outputs and activities, they can be financed by multiple sources. The government can finance them through public investment programs at all the levels. Households, who can afford, can finance them in the form of out-of-pocket expenses. The non-government organizations, cooperatives and community sectors can finance some of the SDGs through resource mobilization at the community level. The private sector can mobilize equity and debt from the domestic market as well as through Foreign Direct Investment for financing the SDGs which could be implemented in business model.

The SDG 17 has clearly mentioned about enhancing international support for implementing effective and targeted capacity-building in developing countries to support national plans to implement all the sustainable development goals, including through North-South, South-South and triangular cooperation. But international support must be led and reinforced by the national government so that capacity constraints do not impede the achievement of SDGs despite commitments, resource availability and willingness to carry forward the agenda.

Strong monitoring system with credible data base is crucial to the success of SDGs. Only an effective government with a strong statistical systems can measure and incentive progress across the goals. There is also huge difference between government recorded data and those compiled by non-government organizations. This is particularly the case with violence, crime, human trafficking, violation of human rights, etc.

The existing statistical system is marred with various problems such as data sources spread over multiple windows andother than census and economic surveys, social and governance related surveys being outside the purview of CBS. Data revolution taking place due to technological changes, evolution of big data and growing trend of open data system is yet to evolve.

Chapter V: Conclusion and Way Forward for Achieving the SDGs

Eradicating poverty requires commencing two-pronged strategyputting credible macroeconomic policies in place for higher economic growth and affecting distribution of income through policy interventions in the labor market along with expanding social protection. Unlocking the economic growth potential requires that enabling fiscal, monetary, external trade, investment and labour market policies are in place, and are providing environment for private investment --domestic and foreign.

Income inequality has reduced the impact of economic growth on poverty reduction in Nepal also. Such inequality is built in the disparity in the ownership of productive assets like land, capital and technology, level of education and skill, and remoteness of human settlements. For progressively achieving and sustaining income growth of the bottom 40 per cent of the population at a rate higher than the national average, interventions have to focus on those very population and areas along with following progressive taxation and subsidy policies.

Inclusion of peace, justice, and responsible institutions in SDGs agenda is a welcome step. However, the goal, especially in the components of good governance, rule of law and access to justice, is quite vague. The country needs to prioritize particular indicators in the agenda as appropriate to the context.

The SDGs being shared responsibility, both the national government and international development partners have to be on board and should be working together to address the resource, governance, and capacity gaps at the country level. But if the experience of MDGs was any indication, it would be extremely difficult to bring all the global development actors on one platform. This applies even among the UN agencies.

The SDGs are very much resource intensive; and developing countries like Nepal can face large financing, technological and implementation capacity gaps. Such gaps have to be measured in concrete terms with proper needs assessment tools so that funding and raising implementation capacity could be objectively worked out. Pledging international aid for the SDGs also requires that the funding gap is credibly worked out. The government will have to identify the interventions needed for achieving the targets and their indicators, and assess the constraints in their implementation.

Obviously SDG priorities have to be set against fiscal, financial, managerial, institutional and other capacity constraints. But they should not be rigidly bound by the constraints; and, efforts must be made to unlock the binding resource and other capacity constraints.In essence, prioritization of projects and programs should be based, along with direct or indirect contribution to the achievement of any SDGs, on synergy that they can bring in the development process, having larger social impact along with economic.

The SDGs cut across all sectors and actors including the civil society organizations and the private sector. A large portion of activities under SDGs are to be held at the private sector and other non-government development actors. At the current planning process, more than 55 per cent of the investment required to achieve the outputs and outcomes of the plan are not monitored. This should be corrected, and all non-government actors including the private sector should be brought under the monitoring purview of the plan and the SDGs.

SDGs are very much data intensive so far as their monitoring and tracking the progress is concerned. The aspiration of `leaving no one behind' implies that the progress must be tracked at a highly disaggregated level, also given the disparity in the initial condition of development across age, sex, location, ethnicity, disability and income groups. A large number of surveys have to be done in the next few years to meet the data gap and create base line data for the targets having no database so far. Existing surveys will have to be tailored to the SDG monitoring needs particularly when it comes to gender, social class and geographical location based target setting, analysis, and monitoring of the SDG outcomes.

source: Nepal's Sustainable Development Goals: Baseline Report, June 2017 (NPC, GoN)

\hypertarget{policy-background-challanges-and-progress-in-nepal}{%
\subsection{Policy background, challanges and progress in Nepal}\label{policy-background-challanges-and-progress-in-nepal}}

Nepal's social and political progress has been highly progressive. Economic growth remained sluggish but picking up now. Nepal's natural vulnerabilities, from earthquakes to climate change, lie unmitigated. Building on the gains so far, the challenge for Nepal is to swiftly complete the unfinished agenda of the Millennium Development Goals (MDGs), and embrace a much more ambitious aspiration of fulfilling the SDGs. Furthermore, Nepal expects to become a vibrant middle income country by 2030. However, the country is resource constrained, and it needs to forge a diverse alliance for SDGs.

Policy and Institutional Context

The goal of leaving no one behind fits well with the inclusive political order that Nepal has been building. The new Constitution (2015) aspires to create a prosperous, egalitarian and pluralistic society, and serves as the overarching guide to all development policies, plans and programs. The current (14th) periodic plan (2016/17-2018/19), and other sectoral plans, policies and their targets are being aligned with SDGs. Specific SDGs codes are assigned for all national programs in the national budget. Three high-level committees have been formed to help implement SDGs. A steering committee is chaired by the Prime Minister; a coordination committee is chaired by the Vice Chairman of the National Planning Commission (NPC) and nine thematic committees are headed by NPC Members. The membership of the coordination and working committees is broadly representative of the public and private sectors, as well as civil society and development partners.

National Targets and Progress of Selected SDGs

Nepal was probably one of the first countries to produce a SDG baseline study in 2015, before the formal adoption of the SDGs. Nepal has halved extreme poverty (SDG 1) in the past 15 years, and is on track to bring it d0wn to less than 5 percent by 2030. SDG 2 targets include the reduction in the prevalence of undernourishment to 3 percent and prevalence of underweight children under five years of age to 5 percent by 2030. Similarly, SDG 3 targets include reducing the maternal mortality rate to less than 70 per 100,000 live births by 2030. Other targets include the virtual elimination of the prevalence of HIV, TB, Malaria, other tropical diseases, and water borne diseases. In Nepal today, there is gender parity at all levels of education. The targets for SDG 5 includes the elimination of wage discrimination, physical/sexual violence, and all harmful social practices, such as child marriages. Nepal expects women to fill 40 percent of all elected seats in local governments, and at least one-third of the seats in the national parliament. In the civil service, women in public decision-making positions will have increased four-fold of total employees by 2030.

The targets for SDG 9 are to increase road density to 1.5 km per square km and paved road density to 0.25 km per square km, and to connect all districts, municipalities and village councils to the national road network. In industries, the target by 2030 is to increase the share of employment to 25 percent; within the subset of manufacturing, employment is to reach 13 percent. SDG 17, on the means of implementation, expects adherence by all stakeholders, from resource mobilization and capacity development to shared responsibility and accountability. Nepal's progress in revenue mobilization is impressive, but also vulnerable to likely swings in the large volumes of inward remittances which bolster import-based taxation. The aim is to increase the share of revenue from about 22 percent today to 30 percent of Gross Domestic Product (GDP) by 2030. Domestic expenditures financed by revenue is estimated to reach 80 percent. For meeting the private sector investment financing gap, foreign direct investment (inward stock) is expected to increase to 20 percent of GDP in 2030 from less than 3 percent in 2015.

Challenges in SDGs Implementation

SDGs are interlinked, indivisible, and ambitious posing major implementation challenges in a low-income country like Nepal, which has limited resources. As the country embarks on implementing a new federal structure of governance, a prominent challenge will be to quickly mainstream SDGs into the provincial and local level planning and budgeting systems. Weak database and lack of availability of disaggregated data by sex, age, social groups, disability status, geography, income and sub-national level will hinder monitoring of progress. In addition to the realignment of policies, financing challenges will loom large, particularly to trigger and sustain job-creating economic growth, enhance the quality of social service provisioning and to invest adequately to reduce risks from disasters.

Way Forward

Positive lessons learned from the MDGs era will need to be scaled up. For example, in health, education, water and sanitation, the Sectorwide Approach yielded better results because of coordinated resource mobilization. Similarly, the triangular partnership between the government, private sector and development partners proved quite effective in some areas, but will need to be augmented significantly especially to spur reforms that attract substantial private capital and entrepreneurship. Nepal will need to constantly update its targets and indicators contained in sectoral master plans, medium term plans and strategies. Many SDG goals and indicators do not yet have a quantitative baseline. This needs an urgent redress, and the data that do exist need further disaggregation, particularly based on new political jurisdictions. Monitoring SDGs progress within the existing institutional framework of data generation and management needs an overhaul. Above all, SDGs are interconnected and the achievement of one goal has a synergetic effect on others. These integrated challenge needs a matching response in terms of fiscal, managerial and institutional capacities.

Source:

\url{https://sustainabledevelopment.un.org/content/documents/15896Nepal_Main_Message_HLPF_2017.pdf}

For main document, refer to: \url{https://www.npc.gov.np/images/category/Sustainable_Devlopment_Agenda_Eng.pdf}

\hypertarget{diplomatic-missions-and-diplomatic-relations-of-nepal}{%
\chapter{Diplomatic missions and Diplomatic relations of Nepal}\label{diplomatic-missions-and-diplomatic-relations-of-nepal}}

\hypertarget{diplomatic-missions}{%
\section{Diplomatic missions}\label{diplomatic-missions}}

Under long established principles of international law now codified in Article 2 of the Vienna Convention on Diplomatic Relations, the establishment of diplomatic relations between States and the establishment of permanent diplomatic missions take place by mutual consent. The right to send and receive diplomatic agents flows from recognition as a sovereign State and was formerly known as the right of legation (ius legationis).

It is in modern practice highly exceptional for two States to recognize each other without formally establishing diplomatic relations---and such a situation usually indicates extreme tension or coolness between them. The United Kingdom for example recognized North Korea as a State some years before the establishment of diplomatic relations between the two in December 2000.

It is common, though not universal practice, for the government of a State to issue a formal statement on recognizing another-usually newly established-State and such a statement may offer to establish diplomatic relations with the new State, or be followed very shortly by such an offer.

Where the birth of the new State takes place without the consent of the State from whose territory it is being formed, premature recognition and establishment of diplomatic relations by other States will be regarded by the parent State as unlawful and in some circumstances as an intervention in its own internal affairs. Thus in February 1991 the Soviet Union protested at a decision by Iceland to recognize Lithuania-then still regarded by Moscow as a constituent Republic of the USSR-and to establish diplomatic relations with it.

The disappearance of a sovereign State - usually on fusion with another State - is on the same principles followed by the ending of its separate diplomatic relations with other States as they recognize the new situation. On the reunification of Germany in 1989, the separate East German embassies were taken over by the newly reunited Germany. Other States, if they retain former embassy premises in a city which is no longer the seat of government, may transform them into consulates - or perhaps seek the permission of the government of the newly unified State for their transformation into offices forming part of the mission to that State.

Where a change of government within a State takes place, continuance or otherwise of diplomatic relations between that State and others also depends on whether the new government has been recognized. When the change of government is constitutional, for example as a result of free elections, diplomatic relations with other States will continue as a matter of course. Fresh credentials may be required by ambassadors from other States where the new government results from or entails a different head of state. Where a change of government results from conflict or revolution, diplomatic relations may continue or there may be a hiatus, and the appointment of new ambassadors is likely.

Where a government - even one apparently in control of a State - has not received recognition from other States it will be unable to establish or to continue diplomatic relations with these other States and its envoys will not be received abroad, accorded any diplomatic status, or given control of the financial assets or property such as embassy premises belonging to the State.

A State establishes a permanent diplomatic mission in another State with which it is in diplomatic relations only where it decides, first, that a permanent mission is necessary for the conduct on its behalf of some or all of the diplomatic functions described below, and secondly, that conditions in the receiving State would permit its representatives to exercise such functions safely and effectively. Embassies are financed from the resources of the sending State which are in most cases subject to stringent controls and in some cases to public and parliamentary scrutiny of expenditure. So a sending State which has limited political or commercial interests in another State, and few of its own nationals resident in or visiting that State, may well decide that it does not require a permanent embassy there. Where the receiving State is in the midst of armed conflict, civil disorder, or a high level of terrorist threat, other States will in general not set up permanent missions and may withdraw missions already in existence.

In such situations there is a range of more limited options available to the two states for the conduct of their relations:

\begin{enumerate}
\def\labelenumi{\arabic{enumi}.}
\tightlist
\item
  diplomatic contacts in the capital of a third State or in the margins of international organizations - in particular the United Nations;
\item
  occasional special missions sent to discuss specific issues of mutual interest;
\item
  multiple accreditation - more usually the sending of a single ambassador to two or more receiving States but less frequently the sending by two or more States of a single ambassador to one receiving State.
\item
  protection of the interests of the sending State by a third State which is represented in the receiving State.
\item
  consular relations may be continued or established between the two States in the absence of a permanent diplomatic mission and even in the absence of recognition of the receiving government or of diplomatic relations.
\end{enumerate}

\hypertarget{functions}{%
\subsection{Functions}\label{functions}}

The functions of a diplomatic mission, which are set out in Article 3 of the Vienna Convention on Diplomatic Relations, are to represent the sending State, to protect its interests and those of its nationals, to negotiate with the government of the receiving State, to report to the government of the sending State on all matters of importance to it, and to promote friendly relations in general between the two States.

The mission must seek to develop relations between the two countries in economic, financial, labour, cultural, scientific, and defence matters. Although there have been important changes in the ways in which diplomatic relations are conducted - many due to developments in communications and travel - the basic functions themselves have hardly altered over the past three hundred years. The list in Article 3 is however not exclusive, so that novel forms of cooperation, such as police or judicial liaison over extradition, possible drug trafficking, or child abduction, may be accepted as properly diplomatic in character.

In carrying out all of its functions the mission acts on the instructions received from the government of the sending State and on its behalf. The function of `representing the sending State' is listed first because it describes not merely ceremonial appearances and acts by an ambassador but embraces all of the subsequently named functions.

A diplomatic mission is required to carry out all of its functions in accordance with international law and also (except where it benefits from a specific exemption) in accordance with the laws and regulations of the receiving State, as dictated by Article 3 of the Vienna Convention. The balancing of the right of an embassy to protect the interests of its own nationals and, more generally, human rights, and its duty not to interfere in the internal affairs of the host State has in recent years given rise to numerous diplomatic disputes, particularly, though not exclusively, in countries which strictly limit freedom of speech and assembly.

\hypertarget{diplomatic-privilage-and-immunites}{%
\section{Diplomatic privilage and immunites}\label{diplomatic-privilage-and-immunites}}

The traditional norm of privileged positions of diplomats is necessary because the representatives of a state can only carry out their diplomatic functions satisfactorily if they are utterly free from pressures, whether legal, physical or moral, that the state in which they are serving may be able to impose on them.

Privileges and immunities are applicable both to the diplomatic mission and its functions, and to the individual. Vienna Convention on Diplomatic Relations is widely accepted with reference to provisioning of diplomatic privileges.

\textbf{Privileges and immunities in respect of the mission and its function}

\begin{enumerate}
\def\labelenumi{\arabic{enumi}.}
\tightlist
\item
  Inviolability and immunity for premises and property
\item
  Inviolability of records, documents, correspondence and archieves
\item
  Freedom and inviolability of communications
\item
  The diplomatic bag, pouch or valise (sealed bag or container clearly marked as such, containing only official documents and articles for official use)
\item
  Exemption from taxation
\item
  Right to import, and exemption from customs duties
\end{enumerate}

\textbf{Personal privileges and immunities}

General provisions for personal privileges and immunities are described into two heads based on the extent of coverage:

A. Full diplomatic privileges and immunities applicable to diplomats and member of their families forming part of their household;

\begin{enumerate}
\def\labelenumi{\arabic{enumi}.}
\item
  Immunity from criminal jurisdiction
\item
  Immunity from civil and administrative jurisdiction
\item
  Provision of waiver of diplomatic immunity only by their government
\item
  Counterclaims
\item
  Inviolability of correspondence
\item
  Inviolability of property
\item
  Exemption from liability for public service
\item
  Exemption from liability to serve as witness
\item
  Exemption from national and local taxation
\item
  Exemption from custom duties
\item
  Exemption from social security provisions
\item
  Exemption from inspection of personal luggage
\item
  Freedom of travel
\item
  Limited diplomatic privileges and immunities.
\end{enumerate}

These privilate are granted to:

\begin{enumerate}
\def\labelenumi{\arabic{enumi}.}
\tightlist
\item
  Non-diplomatic staff of a mission who are not nationals or permanent residents of the state in which they are serving. This class includes members of the administrative and technical staff of a mission; members of their families forming part of their household and members of the domestic (`service') staff of a mission who are not nationals or permanent residents of the state in which they are serving; and private servants of members of the diplomatic staff who are not nationals or permanent residents of the state in which they are serving.
\item
  Diplomats and other members of the staff of a mission who are nationals or permanent residents of the host state
\end{enumerate}

\hypertarget{diplomatic-mission-of-nepal}{%
\section{Diplomatic mission of Nepal}\label{diplomatic-mission-of-nepal}}

\begingroup\fontsize{6}{8}\selectfont

\begin{longtable}[t]{>{\raggedleft\arraybackslash}p{0.8em}>{\raggedleft\arraybackslash}p{6.5em}>{\raggedleft\arraybackslash}p{6em}>{\raggedleft\arraybackslash}p{0.8em}>{\raggedleft\arraybackslash}p{6.5em}>{\raggedleft\arraybackslash}p{6em}>{\raggedleft\arraybackslash}p{0.8em}>{\raggedleft\arraybackslash}p{6.5em}>{\raggedleft\arraybackslash}p{6em}>{\raggedleft\arraybackslash}p{0.8em}>{\raggedleft\arraybackslash}p{6.5em}>{\raggedleft\arraybackslash}p{6em}}
\caption{\label{tab:diplomatic-ties-nepal}Countries to which Nepal formed diplomatic relation, and the date of first making of such arrangement.}\\
\toprule
SN & Country & Date since & SN & Country & Date since & SN & Country & Date since & SN & Country & Date since\\
\midrule
\rowcolor{gray!6}  1 & Afghanistan & 1-Jul-61 & 35 & Equator & 21-Jun-06 & 68 & Libya & 30-Dec-75 & 101 & Romania & 20-Apr-68\\
2 & Albania & 23-May-72 & 36 & Estonia & 20-Apr-92 & 69 & Lithuania & 8-Feb-05 & 102 & Russian Federation & 20-Jul-56\\
\rowcolor{gray!6}  3 & Algeria & 29-Apr-68 & 37 & Ethiopia & 15-Apr-71 & 70 & Luxembourg & 27-Nov-75 & 103 & Saint Vincent & 27-Sep-07\\
4 & Andorra & 22-Sep-06 & 38 & Fiji & 12-Jun-86 & 71 & Macedonia & 6-Jan-98 & 104 & Sanmarino & 10-Aug-05\\
\rowcolor{gray!6}  5 & Argentina & 1-Jan-62 & 39 & Finland & 21-Sep-74 & 72 & Malaysia & 1-Jan-60 & 105 & Saudi Arabia & 15-Mar-77\\
\addlinespace
6 & Armenia & 26-Mar-93 & 40 & FR of Germany & 4-Apr-58 & 73 & Maldives & 1-Aug-80 & 106 & Seychelles & 10-Oct-96\\
\rowcolor{gray!6}  7 & Australia & 15-Feb-60 & 41 & France & 20-Apr-49 & 74 & Mali & 19-Nov-09 & 107 & Singapore & 25-Mar-69\\
8 & Austria & 15-Aug-59 & 42 & Gabon Republic & 17-Jun-85 & 75 & Malta & 25-Sep-83 & 108 & Slovak Republic & 4-Mar-94\\
\rowcolor{gray!6}  9 & Azerbaijan & 28-Feb-95 & 43 & Georgia & 22-Sep-05 & 76 & Mauritius & 12-Feb-81 & 109 & Slovenia & 2-Dec-97\\
10 & Bahrain & 13-Jan-77 & 44 & Greece & 2-Feb-60 & 77 & Mexico & 25-Nov-75 & 110 & Solomon Island & 15-Dec-11\\
\addlinespace
\rowcolor{gray!6}  11 & Bangladesh & 8-Apr-72 & 45 & Guatemala & 8-Aug-06 & 78 & Moldova & 20-Jul-93 & 111 & Somalia & 24-Oct-84\\
12 & Belarus & 19-Jul-93 & 46 & Guyana & 22-Jun-94 & 79 & Mongolia & 5-Jan-61 & 112 & South Africa & 28-Jul-94\\
\rowcolor{gray!6}  13 & Belgium & 19-Aug-63 & 47 & Haiti & 23-May-07 & 80 & Montenegro & 18-Jul-11 & 113 & Spain & 13-May-68\\
14 & Bhutan & 3-Jun-83 & 48 & Holy See & 10-Sep-83 & 81 & Morocco & 18-Feb-75 & 114 & Sri Lanka & 1-Jul-57\\
\rowcolor{gray!6}  15 & Bolivia & 21-May-87 & 49 & Honduras & 18-Aug-06 & 82 & Mozambique & 30-Sep-86 & 115 & Sudan & 11-Jul-69\\
\addlinespace
16 & Bosnia Herzegovina & 12-Jan-00 & 50 & Hungary & 15-Jan-61 & 83 & Myanmar (Burma) & 19-Mar-60 & 116 & Sweden & 10-Jun-60\\
\rowcolor{gray!6}  17 & Botswana & 8-Jan-09 & 51 & Iceland & 25-May-81 & 84 & Netherlands & 2-Apr-60 & 117 & Switzerland & 10-Nov-59\\
18 & Brazil & 7-Feb-76 & 52 & India & 13-Jun-47 & 85 & New Zealand & May-61 & 118 & Syria & 26-Feb-70\\
\rowcolor{gray!6}  19 & Brunei & 3-Feb-84 & 53 & Indonesia & 25-Dec-60 & 86 & Nicaragua & 5-Oct-86 & 119 & Tanzania & 10-Jan-75\\
20 & Bulgaria & 15-Apr-68 & 54 & Iran & 14-Dec-64 & 87 & Nigeria & 20-Dec-75 & 120 & Thailand & 30-Nov-60\\
\addlinespace
\rowcolor{gray!6}  21 & Cambodia (Kampuchea) & 18-Apr-75 & 55 & Iraq & 30-Oct-68 & 88 & Norway & 26-Jan-73 & 121 & Tunisia & 14-Apr-84\\
22 & Canada & 18-Jan-65 & 56 & Ireland & 19-Aug-99 & 89 & Oman & 21-Jan-77 & 122 & Turkey & 15-Nov-62\\
\rowcolor{gray!6}  23 & Chile & 1962 & 57 & Israel & 1-Jun-60 & 90 & Pakistan & 20-Mar-60 & 123 & Turkmenistan & 17-Oct-05\\
24 & Colombia & 7-May-87 & 58 & Italy & 31-Aug-59 & 91 & Panama & 15-Feb-84 & 124 & Ukraine & 15-Jan-93\\
\rowcolor{gray!6}  25 & Congo & 22-Sep-06 & 59 & Japan & 28-Jul-56 & 92 & Paraguay & 2-Aug-06 & 125 & United Arab Emirates & 22-Jan-77\\
\addlinespace
26 & Costa Rica & 16-Aug-77 & 60 & Jordan & 20-Aug-65 & 93 & Peru & 28-Jan-76 & 126 & United Kingdom & 1816\\
\rowcolor{gray!6}  27 & Croatia & 6-Feb-98 & 61 & Kenya & 3-Jun-75 & 94 & Philippines & 12-Feb-60 & 127 & USA & 25-Apr-47\\
28 & Cuba & 25-Mar-75 & 62 & Kuwait & 25-Feb-72 & 95 & Poland & 25-Nov-59 & 128 & Vanuatu & 19-Sep-06\\
\rowcolor{gray!6}  29 & Cyprus & 18-Aug-80 & 63 & Kyrgyzstan & 26-Mar-93 & 96 & Portugal & 1-Sep-76 & 129 & Venezuela & 28-Apr-87\\
30 & Czech Republic & 2-Mar-94 & 64 & Laos & 20-May-60 & 97 & PR of China & 1-Aug-55 & 130 & Vietnam & 15-May-75\\
\addlinespace
\rowcolor{gray!6}  31 & Denmark & 15-Dec-67 & 65 & Latvia & 20-Apr-92 & 98 & Qatar & 21-Jan-77 & 131 & Yemen & 25-Dec-85\\
32 & Dominican Republic & 28-Sep-07 & 66 & Lebanon & 18-Aug-63 & 99 & Republic of Korea & 15-May-74 & 132 & Zambia & 10-Sep-86\\
\rowcolor{gray!6}  33 & DPR of Korea & 15-May-74 & 67 & Lesotho & 18-May-10 & 100 & Republic of Tajikistan & 13-Sep-05 & 133 & Zimbabwe & 27-Nov-84\\
34 & Egypt & 16-Jul-57 &  &  &  &  &  &  &  &  & \\
\bottomrule
\end{longtable}
\endgroup{}

\hypertarget{india-nepal-relations}{%
\section{India-Nepal relations}\label{india-nepal-relations}}

The Republic of India and the Federal Democratic Republic of Nepal initiated their relationship with the 1950 Indo-Nepal Treaty of Peace and Friendship and accompanying secret letters that defined security relations between the two countries, and an agreement governing both bilateral trade and trade transiting Indian territory. The 1950 treaty and letters exchanged between the Indian government and Rana rulers of Nepal, stated that ``neither government shall tolerate any threat to the security of the other by a foreign aggressor'' and obligated both sides ``to inform each other of any serious friction or misunderstanding with any neighboring state likely to cause any breach in the friendly relations subsisting between the two governments.'' These accords cemented a ``special relationship'' between India and Nepal. The treaty also granted Nepalese, the same economic and educational opportunities as Indian citizens in India, while accounting for preferential treatment to Indian citizens and businesses compared to other nationalities in Nepal. The Indo-Nepal border is open; Nepalese and Indian nationals may move freely across the border without passports or visas and may live and work in either country. However, Indians aren't allowed to own land-properties or work in government institutions in Nepal, while Nepalese nationals in India are allowed to work in Indian government institutions (except in some states and some civil services the IFS, IAS, and IPS). After years of dissatisfaction by the Nepalese government, India in 2014, agreed to revise and adjust the treaty to reflect the current realities. However, the modality of adjustment hasn't been made clear by either side.

Despite the close linguistic, marital, religious, and, cultural ties, at people to people level between Indians and Nepalese, since late 2015, political issues and border disputes have strained relations between the two countries with anti-Indian sentiment growing amongst the government and people of Nepal. Further because of border disputes between the two countries, a boundary agreement hasn't yet been ratified by either government.

Independent political history
1950-1970

This section possibly contains original research. Please improve it by verifying the claims made and adding inline citations. Statements consisting only of original research should be removed. (November 2019) (Learn how and when to remove this template message)
The foundation of relations between India and Nepal was laid with Indo-Nepalese friendship Treaty in 1950. In the 1950s, the Rana rulers of Nepal welcomed close relations with India, fearing a China-backed communist overthrow of their (Rana) autocratic regime.{[}citation needed{]} Rana rule in Nepal however collapsed within 3 months of signing the 1950 Indo-Nepal Treaty of Peace and Friendship, only to be replaced by the only pro-Indian party of the time - Nepali Congress. As the number of Indians living and working in Nepal's Terai region increased and the involvement of India in Nepal's politics deepened in the 1960s and after, so too did Nepal's discomfort with the special relationship.{[}citation needed{]} India's influence over Nepal increased throughout the 1950s. The Nepalese Citizenship Act of 1952 allowed Indians to immigrate to Nepal and acquire Nepalese citizenship with ease-a source of huge resentment in Nepal (This policy was not changed until 1962 when several restrictive clauses were added to the Nepalese constitution). Also in 1952, an Indian military mission was established in Nepal, which consisted of a Major General and 20 other Indian army personnel (later extended to 197 in total). At the same time, Nepal's Royal family's dissatisfaction with India's growing influence began to emerge, and overtures to China were initiated by Nepal as a counterweight to India. Further the Nepalese government, as a deliberate attempt to show pro-USA tilt in Nepalese foreign policy, established diplomatic ties with the state of Israel in June 1, 1960, while the Indian government supported Palestine and remained pro-USSR throughout the cold war.

Following the 1962 Sino-Indian border war, the relationship between Kathmandu and New Delhi thawed significantly. India suspended its support to India-based Nepalese opposition forces (opposing the dissolution of democratic government by King Mahendra). The defeat of Indian forces in 1962 provided Nepal with the breathing space and Nepal extracted several concessions in trade. In exchange, through a secret accord concluded in 1965, similar to an arrangement that had been suspended in 1963, India won a monopoly on arms sales to Nepal.

In 1969 relations again became stressful as Nepal challenged the existing mutual security arrangement and asked that the Indian security checkposts and liaison group be withdrawn. Resentment also was expressed against the 1950s TPF. India withdrew its military check-posts and liaison group consisting of 23 military personnel in 1970 from Nepal, although the treaty was not abrogated.

Tensions came to a head in the mid-1970s, when Nepal pressed for substantial changes in the trade and transit treaty and openly criticised Sikkim's 1975 annexation by India. In 1975 King Birendra Bir Bikram Shah Dev against the backdrop of Indian annexation of Nepal's close neighbor `The Kingdom of Sikkim' proposed Nepal to be recognized internationally as a `Zone of Peace' where military competition would be off limits. Nepal's proposal immediately received support from Pakistan and China, but not from India. In New Delhi's view, if the king's proposal did not contradict the 1950 treaty that the-then Indian government had signed with the Rana rulers of Nepal, it was unnecessary; if it was a repudiation of the special relationship, it represented a possible threat to India's security and could not be endorsed. In 1984 Nepal repeated the proposal, but there was no reaction from India. Nepal continually promoted the proposal in international forums and by 1990 it had won the support of 112 countries including the US, the UK, and France.

1970-1980
In 1978 India agreed to separate trade and transit treaties, satisfying a long-term Nepalese demand. However, much to the annoyance of Nepalese Royal Palace and in continued violation of the 1950s PFT, India consistently allowed the opposition parties of Nepal to use Indian soil to launch agitation against the Nepalese government and refused to endorse Nepal as a Zone of Peace. In 1988, when the two treaties were up for renewal, Nepal refused to accommodate India's wishes for a single trade and transit treaty stating that `it violates the principle of freedom to trade'. Thereafter, both India and Nepal took a hard-line position that led to a serious crisis in India-Nepal relations. Nepalese leaders asserted the position that as per the UN charter, transit privileges were ``a fundamental and a permanent right of a land-locked country'' and thus India's demand for a single treaty was unacceptable. So, after two extensions, the two treaties expired on 23 March 1989, resulting in a virtual Indian economic blockade of Nepal that lasted until late April 1990. As time passed Indian economic sanctions over Nepal steadily widened. For example, preferential customs and transit duties on Nepalese goods entering or passing through India (whether imports or exports) were discontinued. Thereafter India let agreements relating to oil processing and warehouse space in Calcutta for goods destined to Nepal expire. Aside from these sanctions, India cancelled all trade credits it had previously extended to Nepal on a routine basis.

To withstand the renewed Indian pressure, Nepal undertook a major diplomatic initiative to present its case on trade and transit matters to the world community. The relationship with India was further strained in 1989 when Nepal decoupled its rupee from the Indian rupee which previously had circulated freely in Nepal. India retaliated by denying port facilities in Calcutta to Nepal, thereby preventing delivery of oil supplies from Singapore and other sources. In historian Enayetur Rahim's view, ``the economic consequences of the dispute\ldots{} were enormous. Nepal's GDP growth rate plummeted from 9.7\% in 1988 to 1.5\% in 1989. This had a lot to do with the decreased availability of goods. Shortly after the imposition of sanctions, Nepal experienced serious deficiencies of important goods such as coal, fuel, oil, medicine and spare parts. Nepal also suffered economically from higher tariffs, the closure of border points and the tense political atmosphere. From one of the most thriving economies in Asia, Nepal was now quickly finding itself in the league of World's poorest nation.'' Although economic issues were a major factor in the two countries' confrontation, Indian dissatisfaction with Nepal's decision to impose work permits over Indians living in Nepal and Nepal government's attempt to acquire Chinese weaponry in 1988 played an important role. India linked security with economic relations and insisted on reviewing India-Nepal relations as a whole. After failing to receive support from wider international community, Nepalese government backed down from its position to avoid the worsening economic conditions. Indian government, with the help of Nepalese opposition parties operating from India, managed to bring a change in Nepal's political system, in which the king was forced to institute a parliamentary democracy. The new government, led by pro-India parties, sought quick restoration of amicable relations with India.

1990s
The special security relationship between New Delhi and Kathmandu was re-established during the June 1990 New Delhi meeting of Nepal's prime minister Krishna Prasad Bhatarai and Indian prime minister V.P. Singh, after India ended its 13-month-long economic blockade of Nepal. During the December 1991 visit to India by Nepalese prime minister Girija Prasad Koirala, the two countries signed new, separate trade and transit treaties and other economic agreements designed to accord Nepal additional economic benefits.

Indian-Nepali relations appeared to be undergoing still more reassessment when Nepal's prime minister Man Mohan Adhikary visited New Delhi in April 1995 and insisted on a major review of the 1950 peace and friendship treaty which Nepal believed was enabling an ongoing demographic shift in Nepal's Terai region. In the face of benign statements by his Indian hosts relating to the treaty, Adhikary sought greater economic independence for his landlocked nation while simultaneously striving to improve ties with China.

In June 1990, a joint Kathmandu-New Delhi communique was issued pending the finalisation of a comprehensive arrangement covering all aspects of bilateral relations, restoring trade relations, reopening transit routes for Nepal's imports, and formalising respect of each other's security concerns. Essentially, the communiqu? announced the restoration of the status quo ante and the reopening of all border points, and Nepal agreed to various concessions regarding India's commercial privileges. Kathmandu also announced that lower cost was the decisive factor in its purchasing arms and personnel carriers from China and that Nepal was advising China to withhold delivery of the last shipment.

2000s
In 2005, after King Gyanendra took over, Nepalese relations with India soured. However, even after the restoration of democracy, in 2008, Prachanda, the Prime Minister of Nepal, visited India, in September 2008 only after visiting China, breaking the long-held tradition of Nepalese PM making India as their first port-of-call. When in India, he spoke about a new dawn, in the bilateral relations, between the two countries. He said, ``I am going back to Nepal as a satisfied person. I will tell Nepali citizens back home that a new era has dawned. Time has come to effect a revolutionary change in bilateral relations. On behalf of the new government, I assure you that we are committed to make a fresh start.''

In 2006, the newly formed democratic parliament of Nepal passed the controversial citizenship bill that led to distribution of Nepalese citizenship to nearly 4 million stateless immigrants in Nepal's Terai by virtue of naturalisation. While the Indian government welcomed the reformed citizenship law, certain section of Nepalese people expressed deep concerns regarding the new citizenship act and feared that the new citizenship law might be a threat to Nepalese sovereignty. The citizenship bill passed by the Nepalese parliament in 2006 was the same bill that was rejected by Late King Birendra in 2000 before he along with his entire family was massacred. Indian government formally expressed sorrow at the death of Late King Birendra of Nepal.

In 2008, Indo-Nepal ties got a further boost with an agreement to resume water talks after a 4-year hiatus. The Nepalese Water Resources Secretary Shanker Prasad Koirala said the Nepal-India Joint Committee on Water Resources meet decided to start the reconstruction of the breached Koshi embankment after the water level went down. During the Nepal PM's visit to New Delhi in September the two Prime Ministers expressed satisfaction at the age-old close, cordial and extensive relationships between their states and expressed their support and co-operation to further consolidate the relationship.

The two issued a 22-point statement highlighting the need to review, adjust and update the 1950 Treaty of Peace and Friendship, amongst other agreements. India would also provide a credit line of up to \$15 million to Nepal to ensure uninterrupted supplies of petroleum products, as well as lift bans on the export of rice, wheat, maize, sugar and sucrose for quantities agreed to with Nepal. India would also provide \$2 million as immediate flood relief.
In return, Nepal will take measures for the ``promotion of investor friendly, enabling business environment to encourage Indian investments in Nepal.''

2010s
In 2010 India extended a Line of credit worth US\$50 million \& 80,000 tonnes of foodgrains. Furthermore, a three-tier mechanism at the level of ministerial, secretary and technical levels will be built to push forward discussions on the development of water resources between the two sides. Politically, India acknowledged a willingness to promote efforts towards peace in Nepal. Indian External affairs minister Pranab Mukherjee promised the Nepali Prime Minister Prachanda that he would ``extend all possible help for peace and development.''

However, in recent years, the increasing dominance of Maoism in Nepal's domestic politics, along with the strengthening economic and political influence of the People's Republic of China has caused the Nepalese government to gradually distance its ties with India, though Nepal still does support India at the UN. Prime Minister of India Narendra Modi visited Nepal in August 2014, marking the first official visit by an Indian prime minister in 17 years. During his visit, Indian government agreed to provide Nepal with US\$1 billion as concessional line of credit for various development purposes and a HIT formula, but he insisted that Indian immigrants in Nepal don't pose a threat to Nepal's sovereignty and therefore open border between Nepal and India should be a bridge and not a barrier. Nepal and India signed an important deal on 25 November 2014 as per which India will build a 900 MW hydropower plant at a cost of another US\$1 billion. An amount of US\$250 million has been granted to Nepal as a part of the agreements signed on 22 February 2016 for post-earthquake reconstruction.

A perpetual issue for many people of Nepali origin; the birthplace of Gautama Buddha has long been a cultural and social issue devoid from the political landscape of both Nepal and India. However, since the souring of relations between the two countries, the issue has been used to undermine relations between the two countries both politically and socially. The two-day-long International Buddhist conference in Kathmandu which ran from May 19-20, 2016 marked Vesak and the 2,560th birthday of the Buddha was also used to promote the Buddha's birthplace which lies in modern-day Nepal. The decision of the Nepal Culture Ministry to change the theme, ``Preservation and Development of Buddhist Heritage of Nepal'' with the sub-theme ``Lumbini - Birthplace of Buddha'' under the name ``Lumbini - Fountainhead of Buddhism'' was met with criticism from India which subsequently boycotted the conference due to this and on the back of China's supposed monetary involvement in the conference. Nepali Prime Minister, K.P. Oli told the media that the conference, ``should help us make clear to the world that Buddha was born in Nepal and that Buddhist philosophy is the product of Nepal''.

In early March 2017, the fatal shooting of a Nepali man who was protesting Indian-occupation on disputed territory between India and Nepal sparked protests in the capital Kathmandu. Indian troops had previously prevented a group of Nepalese farmers living along the border from completing a culvert in the disputed area which ultimately led to protests. It was considered rare for India to retaliate with gunfire.

Border disputes
Main article: Territorial disputes of India and Nepal
The Territorial disputes of India and Nepal include Kalapani 400 km2 at India-Nepal-China tri-junction in Western Nepal and Susta 140 km2 in Southern Nepal. Nepal claims that the river to the west of Kalapani is the main Kali river hence the area should belong to Nepal. But India claims that the river to the west of Kalapani is not the main Kali river, and, therefore the border there should be based on the ridge lines of the mountains Om Parvat to the east of the river. The river borders the Nepalese province of Sudurpashchim and the Indian state of Uttarakhand. The Sugauli Treaty signed by Nepal and British India on 4 March 1816 locates the Kali River as Nepal's western boundary with India. Subsequent maps drawn by British surveyors show the source of the boundary river at different places. This discrepancy in locating the source of the river led to boundary disputes between India and Nepal, with each country producing maps supporting their own claims. Indian government, however, from 1962 onward, forwarded the argument that border should be based on the ridge lines of the mountain Om Parvat. The Kali River runs through an area that includes a disputed area of about 400 km? around the source of the river although the exact size of the disputed area varies from source to source. The dispute intensified in 1997 as the Nepali parliament considered a treaty on hydro-electric development of the river. India and Nepal differ as to which stream constitutes the source of the river. Nepal has reportedly tabled an 1856 map from the British India Office to support its position. Kalapani has been controlled by India's Indo-Tibetan border security forces since the Sino-Indian War with China in 1962. In 2015, the Nepalese parliament objected an agreement between India and China to trade through Lipulekh Pass, a mountainous pass in the disputed Kalapani area, stating that the agreement between India and China to trade through Kalapani violates Nepal's sovereign rights over the territory. Nepal has called for the withdrawal of the Indian border forces from Kalapani area.

As the first step for demarcating Indo-Nepal border, survey teams from both countries located and identified missing pillars along the border, and, an agreement was reached to construct new pillars in some places. According to the Nepalese government estimates, of the 8000 boundary pillars along the border, 1,240 pillars are missing, 2,500 require restoration, and, 400 more need to be constructed. The survey teams conducted survey of the border pillars based on the strip maps prepared by the Joint Technical Level Nepal-India Boundary Committee (JTLNIBC). The JTLNIBC was set up in 1981 to demarcate the India-Nepal border and after years of surveying, deliberations and extensions, the Committee had delineated 98 per cent of the India-Nepal boundary, excluding Kalapani and Susta, on 182 strip maps which was finally submitted in 2007 for ratification by both the countries. Unfortunately neither country ratified the maps. Nepal maintained that it cannot ratify the maps without the resolution of outstanding boundary disputes, i.e.~Kalapani and Susta. India, on the other hand, awaited Nepal's ratification while at the same time urging it to endorse the maps as a confidence building measure for solving the Kalapani and Susta disputes. In absence of a ratification, the process of completely demarcating the India-Nepal boundary could not be undertaken.

Border crossings
Main articles: Designated border crossings of India and Borders of India
Integrated check posts with immigration and customs facilities are:

Jogbani, Bihar
Sunauli, Uttar Pradesh
Rupaidiha, Uttar Pradesh
Taulihawa-Siddharthnagar (only for India and Nepalese citizens)
Jathi, Bihar

Since 2014 to enhance the collaborative relations between the two nations, Nepal and India started Trans-border bus services from New Delhi to Kathmandu connecting the nation's capital of both countries. The service is in operation by Delhi Bus Corporation (DTC) India and several other private Travel companies. At present(2019), Kathmandu to Delhi bus service, Kathmandu to Siliguri Bus service, Kathmandu to Varanasi, Delhi to Janakpur bus service are in operation.

Trade
India is Nepal's largest trade partner and the largest source of foreign investments, besides providing transit for almost entire third country trade of Nepal. India accounts for over two-third of Nepal's merchandise trade, about one-third of trade in services, one-third of foreign direct investments, almost 100\% of petroleum supplies, and a significant share of inward remittances on account of pensioners, professionals and workers working in India In the year 2017-2018, Nepal's total trade with India was about US\$8.2 billion; Nepal's exports to India were about US\$446.5 million; and imports from India were about US\$7.7 billion.

Nepal's main imports from India are petroleum products (28.6\%), motor vehicles and spare parts (7.8\%), M. S. billet (7\%), medicines (3.7\%), other machinery and spares (3.4\%), coldrolled sheet in coil (3.1\%), electrical equipment (2.7\%), hotrolled sheet in coil (2\%), M. S. wires, roads, coils and bars (1.9\%), cement (1.5\%), agriculture equipment and parts (1.2\%), chemical fertilizer (1.1\%), chemicals (1.1\%) and thread (1\%). Nepal's export basket to India mainly comprises jute goods (9.2\%), zinc sheet (8.9\%), textiles (8.6\%), threads (7.7\%), polyester yarn (6\%), juice (5.4\%), catechue (4.4\%), Cardamom (4.4\%), wire (3.7\%), tooth paste (2.2\%) and M. S. Pipe (2.1\%).

Human trafficking
Human trafficking in Nepal is a serious concern. An estimated 100,000-200,000 Nepalese in India are believed to have been trafficked. Sex trafficking is particularly rampant within Nepal and to India, with as many as 5,000-10,000 women and girls trafficked to India alone each year. The seriousness of trafficking of Nepalese girls to India was highlighted by CNN Freedom Project's documentary: Nepal's Stolen Children. Maiti Nepal has rescued more than 12,000 stolen Nepalese children from sex trafficking since 1993.

2015 Madhesi crisis and Nepal
Main article: 2015 Nepal fuel crisis

\hypertarget{nepal-uk-relation}{%
\section{Nepal-UK relation}\label{nepal-uk-relation}}

\begin{enumerate}
\def\labelenumi{\arabic{enumi}.}
\tightlist
\item
  Nepal and United Kingdom have celebrated 200 years of their relationship. How do you view this relationship and what recommendations, if you have, you wish to offer to take the subsisting relationship to the next level ?
\end{enumerate}

\begin{solution}

Nepal established diplomatic relations with the United Kingdom in 1816. Treaty of Friendship between Great Britain and Nepal was signed in 1923 which further formalized relations between the two countries. Ever since the establishment of their diplomatic relations, friendship, mutual understanding, cordiality, cooperation and respect for each other's national interests and aspirations have characterized relationship between the two countries. The UK is also the first country in the world with which Nepal had established diplomatic relations and also the first to have established its Embassy in Kathmandu, the capital of Nepal. This is the country where Nepal had established its first diplomatic mission (Legation) back in 1934. It was elevated to the Ambassador level in 1947 AD.

The UK remained one of the top development partners of Nepal with the annual British aid on an increasing trend. Tourism, trade, education, and British Gurkha connection remained the key dimensions of the bilateral relations.

Since then, relations between the two countries have continued to grow, with a new Treay of Perpetual Peace and Friendship signed in 1950 which expanded areas of cooperation and an exchange of State Visits. Amicable relations continue today; Nepal continues to be the source of recruitment of Gurkha soldiers into the British army -- a tradition dating back to the nineteeth century but still an essential part of Britain's modern army -- and the UK remains one of the most significant providers of development assistance to Nepal.

As one of Great Britain's allies during the two world wars, Nepalese soldiers in hundreds of thousands fought and sacrificed their lives in many battlefields of the world, thus leaving a heritage of deep and sincere friendship in the history of two countries. The Gurkha soldiers are the most visible bridge between Nepal and the United Kingdom. It is to state that the Gurkha's service in the British army started on April 24, 1815, which is continuing till the date spanning over more than two hundred years. The solid foundation of the relation is built on the history of service, sacrifice and bravery of these Gurkha soldiers.

Nepal and United Kingdom are celebrating the bicentenary of bilateral relations between two countries. To mark the 200th year of the beginning of official contacts between the two countries in 1816, a number of programs and events from both sides have been organized in 2016-2017. There have been intermittent to and fro high level visits between Nepal and United Kindom since as early as 1960.

\textbf{Treaties/Memorandum of Understandings/Agreements between the UK and Nepal}

Bilateral consultation mechanism

With a view to enhancing the friendly cooperation and further promoting the bilateral relations Nepal and the United Kingdom have signed a MoU on 7th January, 2014. On behalf of Nepal and UK, Foreign secretary Mr. Arjun Bahadur Thapa and UK's Permanent Under Secretary Sir Simon Frazer signed the MoU for establishing the Consultation Mechanism at the Foreign and Commonwealth Office, London respectively. As per the MoU, both the parties will hold consultations at jointly decided intervals alternately in the capital of either country at an appropriate level as may be decided upon and hold discussion on all aspects of bilateral relations as well as on international or regional issues of mutual interest.

Finalized Agreements/MoU

\begin{itemize}
\item TIFA Agreement signed in 2011.
\item An agreement on promotion and protection of investment between Nepal and UK signed on March 2, 1993
\item Development Partnership Agreement between Nepal and UK, signed on December 2013. It supersedes the Technical cooperation agreement desiring to strengthen the traditional cooperation and cordial relation between Nepal and the UK signed on 31 May 1994.
\item Trade Agreement signed in 1965.
\end{itemize}

Agreements/MoU in Progress

\begin{itemize}
\item Bilateral Agreement on Avoidance of Double Taxation
\end{itemize}

\textbf{Economic and Technical Cooperation (Development Cooperation)}

The UK aid to Nepal in various fields of activities started in 1961. The United Kingdom has been offering fellowships to the Government of Nepal since 1950s. British volunteers are engaged in Nepal since 1964. These programmes have contributed to Nepal's need for specialised and trained manpower and also have developed important links between the people of the two countries. In order to better understand and address the issues of poverty in Nepal, in April 1999 Department for International Development (DFID) established an in-country office staffed by a multi-disciplinary team of both UK nationals and staff appointed in country. DFID presence in Nepal has greatly assisted in working with Government of Nepal in the development of understanding and establishment of networks that will promote opportunities for change.

Recently Development Partnership Arrangement to provide a transparent and mutual accountability framework between Nepal and the United Kingdom on development assistance was signed on December 2013. The assistance of British Government generally comes through the Umbrella Agreement. The first such agreement of Pound Sterling 12 million was signed on 23 September 1979 for the implementation of various projects. The UK Government’s Department for International Development (DFID) prepared an Operational Plan for Nepal during 2011-2016. This plan commits up to Pound Sterling 413 million of UK official development assistance during the period 2011-2016. All bilateral aid with Nepal is on a grant basis.

As per Development Cooperation Report 2014-2015 by Ministry of Finance, UK remained the top ODA provider (based on disbursement) with a total assistance of US dollars 168.07 million in the Fiscal Year 2014/2015.

DFID Technical Support has been instrumental in promoting foreign direct investment especially in large hydropower project through Investment Board Nepal for the last few years. This very programme has provisioned of supporting IBN to promote private investment in the country. Likewise there are provisions of continuing support to the second and third phases of Financial Sector Reform Programme of the Government initiated few years back. The program will last for six years from 2015 to 2020.

The DFID Nepal's operational Plan is divided into four main areas: governance and security, inclusive wealth creation, human development (basic services including education and health), and climate change/disaster management. The UK has assisted Nepal in the areas of livelihoods, e.g. agriculture, forestry, transport and communications, local development; basic services, e.g. education, health, water supply and sanitation; good governance, human rights and peace building efforts.

\textbf{Earthquake, 2015}

Nepal received generous support and assistance from the Government of the United Kingdom in the aftermath of the earthquake of 25 April 2015. It responded to the Nepal earthquakes immediately with the deployment of Eight Disaster Response Specialists under Rapid Response Facility and with the release of GBP 5 million. The United Kingdom Government pledged USD 110 million for the reconstruction and rebuilding of Nepal. 

\textbf{Trade}

Nepal and UK signed trade agreement in 1965. Despite our extensive engagements and linkages, both at the level of government and, people in the backdrop of our long-standing friendly and cooperative relations, trade and investment relations between the two countries have largely remained way below its actual potential.

With high imbalance and elements of unpredictability, this is, of course, a typical feature of a trade relation between a developed and least developed country. Nepal faced a trade deficit of 142,182,856 Rupees in the first eight month of 2016. The total export to UK stands at 1,256,018,669 Rupees while the total import stands at 1,398,201,525. It is for this reason that Nepal places high importance to directing foreign development cooperation in areas which help enhancing our productive capacity. Aid for trade is, therefore, another preferred area for us in our development cooperation. 

The major items of import from UK are: Motor car, vehicle, parts of aeroplane and helicopter, Whiskies, Malt not roasted, Sweets biscuits, Chocolate in blocks, slab or bar etc. Major Nepalese exports to UK are Pashmina shawls, goatskin, leather goods, Nepalese paper and paper products, woollen carpets, handicrafts, ready-made garments, silverware and jewellery. Likewise, major imports from UK are copper scrap, hard drinks, cosmetics, medicine and medical equipment, textiles, copper wire rod, machinery and parts, aircraft and spare parts, scientific research equipment, office equipment and stationery.

Exchange of visits by trade delegations from Nepal and Britain has added a new dimension to the commercial relations between Nepal and Britain. Nepal-Britain Chamber of Commerce and Industry is active in promoting trade and investment between Nepal and UK.

\textbf{Tourism}

A sizeable number of British tourists come to Nepal every year for trekking and mountaineering and other leisurely activities. Given our historical linkages and increasing people-to-people contacts, the prospects for increasing the tourist arrival from UK in Nepal are quite promising.

UK's support and cooperation in line with our broader policy priorities will be much appreciated, especially in construction and upgradation of tourism infrastructure with focus on tourist safety. DIFID’s recent involvement and interest in tourist security, trekkers tracking, and maintenance of damaged trekking trails, including the Great Himalayan Trail, is most welcome.

Embassy of Nepal in London is planning to organize the many tourism promotional programme to increase the number of tourists from UK.

\textbf{Consular activities}

Embassy of Nepal in London involves in various consular service ranging from passports, visas, Travel Document, NRN Card, Verification, Attestation and Power of Attorney. During Fiscal Year 2075/76, Embassy of Nepal in London generated GBP 542048.72 (NRS 7,73,71,510.41) revenue.

\textbf{Investment}

Foreign investment commitment from UK stands at 1663 million rupees till 2013/14. The major investment is in the areas of banking, tourism, education, technology. An agreement for the promotion and protection of investment between Nepal and UK was signed on March 2, 1993.

There are some British joint ventures in the areas of hotel, travel and trekking, tea production, education, garment, bio-technology and consultancy. These are significant figures, but do not reflect the true potential. There are many British and Non Resident Nepalese (NRN) entrepreneurs who are still making trade and investment successes in Nepal. Standard Chartered Bank and Unilever are two major British companies who have made a huge return from their investment in Nepal in the last 20 years.

An agreement on promotion and protection of investment between Nepal and UK signed on 02 March 1993 provides a framework to further expand cooperation in this field. In this context, the proposal for signing another agreement on avoidance of double taxation will further help boost the flow of FDI between the two countries.

\textbf{Foreign employment and British Gurkha}

Thousands of Nepalese are working at service sectors like hospitals, university, hotels and restaurants in UK. About 200 Nepalese are recruited as Maritime Security Guards in UK. About 150,000 Nepalese are living in UK. Many Nepalese belong to Ex-Gurkha Servicemen and their families.

British Gurkhas is a fully integrated part of the British Armed Forces. It constitutes an important element in Nepal-Britain relations.  Over 160,000 Gurkhas were enlisted in the Gurkha Brigade and other units of the Indian Army during the First and Second World Wars. In recognition of their distinguished service, the British Gorkhas servicemen from Nepal have won 13 Victoria Crosses (VC), the highest British gallantry honour.

\textbf{Cultural Cooperation}

British Council in Nepal is working to extend cultural relations between UK and Nepal. Similarly, Nepalese diaspora and Gurkha soldiers are visible bond who have interlinked the Nepalese culture and traditions in UK.

\textbf{Education cooperation}

UK is supporting US dollars 260,012 for the education of marginalised girl in Kailali district of Nepal. DFID was the core education funding agency in Nepal, supporting the foundations of education values. Active as a School Sector Reform Plan (SSRP) pooling Development Partner. With Education for All overall goals, the SSRP objectives were to 
\begin{enumerate}
\item ensure access and equity in primary education; 
\item improve the efficiency and institutional capacity of primary education; and 
\item enhance the quality and relevance of basic primary education for children and illiterate adults.
\end{enumerate}

In 2013, DFID committed around GBP 2 million to establish 810 additional girls' toilet and water supplies/WASH facilities across 18 districts selected based on low indicators in hygiene and sanitation with an implementation date which was extended to July 2015.

The British Government has been regularly providing scholarships in different areas for the development of human resource in Nepal since 1950s. Britain is offering Chevening Award to Nepalese from many years. A lot of Nepalese students are pursuing higher level studies in the UK. 

\textbf{Security cooperation}

UK has provided support for upgrading of Nepal Army Birendra Peace Operations Training Centre (BPOTC) located at Panchkhal (45 km East of Kathmandu). It has also provided Explosive Ordinance Device (EOD) related Improvised Explosive Device Disposal (IEDD) related Special Training to Nepal Army. UK is continually providing the special training courses to Nepal Army officials.

High level visit from Nepal Army and British Army are regularly taking place in UK and Nepal respectively.

\textbf{Nepalese students in UK}

\end{solution}

\hypertarget{nepal-usa-relation}{%
\section{Nepal-USA relation}\label{nepal-usa-relation}}

\begin{enumerate}
\def\labelenumi{\arabic{enumi}.}
\setcounter{enumi}{1}
\tightlist
\item
  Discuss about the bilateral US relations with Nepal
\end{enumerate}

\textbf{Solution:}

The United States recognized the Kingdom of Nepal on April 27, 1947, when Joseph C. Satterthwaite presented a letter from President Harry S. Truman to then king Tribhubana of Nepal. Satterthwaite was on a special diplomatic mission to Nepal as personal representative of the President. The two countries established diplomatic relations in 1948. Bilateral relations are friendly, and U.S. policy objectives toward Nepal center on helping it build a peaceful, prosperous, resilient and democratic society. Primary U.S. objectives in Nepal include supporting a stable, democratic Nepal that respects the rule of law; promoting investor-friendly economic development; and improving disaster risk management systems.

The United States enjoys a strong and positive relationship with Nepal. Years of diplomatic, development, and military engagement have advanced U.S. interests in transitioning Nepal into a more peaceful, stable democracy with significant economic potential. Since the end of its 10-year civil war in 2006 and the devastating earthquakes of 2015, Nepal has successfully transitioned into a full-fledged constitutional federal republic grounded in a constitution promulgated in 2015. With the recent formation of a new government, Nepal may now be on the cusp of a period of much-needed political stability.

Nepal has sought a bilateral consultation mechanism with the United States and has signed similar agreements with other countries like Australia in summer 2017. Such a forum might include as topics: Millenium Challenge Cooperation (MCC) coordination and implementation; trade (TPS, technical assistance, etc.); security and defense cooperation; and humanitarian assistance and disaster response.

To improve Nepal's economic situation, the MCC signed, in 2017, a \$500 million Compact with Nepal to expand Nepal's electricity transmission infrastructure and improve its road maintenance regime. The Nepali government has committed another \$130 million for a program total of \$630 million. The Compact will build 300 kilometers (km) of high-voltage electric transmission lines, three substations, perform enhanced road maintenance on 305 km of strategic highways, and provide technical assistance to the national electric utility, the new electricity regulator, and the Department of Roads. The Compact is entered into force after the end of FY 2019.

U.S. assistance has been critical to helping Nepal rebuild after the devastating 2015 earthquake and remains committed to building Nepal's resilience in the event of any future disaster. The United States has provided over \$190 million for earthquake relief, recovery, and reconstruction. While these commitments have exceeded our initial pledge of \$130 million made at the International Conference on Nepal's Reconstruction in June 2015, significant recovery needs remain. To date, the United States has built 36 schools and hospitals; has directly helped rebuild over 16,000 homes; trained 260,000 people in safer construction; and developed policies, systems, and controls to ensure that \$8.6 billion in reconstruction results in safer structures for all. In addition to rebuilding a safer Nepal, we are working to help Nepal institutionalize all lessons learned from the 2015 earthquake. We are also helping Nepal to implement its new disaster management law and stand up a new National Disaster Management Authority. We are further supporting Nepal as it introduces federalism by working with newly elected local governments to implement their own disaster management plans---thus helping local authorities meet commitments made to their constituencies. These efforts will help Nepal on its journey to self-reliance.

The United States has also committed security assistance to Nepal, working with the Nepali Army to strengthen their peacekeeping and disaster response capabilities.

The United States and Nepal have signed a trade and investment framework agreement, providing a forum for bilateral talks to enhance trade and investment, discuss specific trade issues, and promote more comprehensive trade agreements between the two countries. In 2016, Nepal became one of few countries in the world with a single-country trade preference program with the United States. Principal U.S. exports to Nepal include agricultural products, aircraft parts, optic and medical instruments and machinery. U.S. imports from Nepal include carpets, apparel and jewelry.

\hypertarget{foreign-investment}{%
\subsection{Foreign Investment}\label{foreign-investment}}

\begin{questions}

\question What are the prospects of Foreign Investment in Nepal ?

\begin{solution}
Nepal has been pursuing a liberal foreign investment policy and been striving to create an investment- friendly environment to attract FDIs into the country. Our tax slabs one of the lowest and our position is fairly good in ease of doing business. Profitable areas of investment include hydropower, industrial manufacturing, services, tourism, construction, agriculture, minerals and energy.

Nepal encourages foreign investment both as joint venture operations with Nepalese investors or as 100 per cent foreign-owned enterprises. The few sectors that are not open to foreign investment are either reserved for national entrepreneurs in order to promote small local enterprises and protect indigenous skills and expertise or are restricted for national security reasons. Approval of the GoN is required for foreign investment in all sectors. No foreign investment is allowed in cottage industries. However, no restriction is placed on transfer of technology in cottage industries.

Most investment gurus believe that agriculture and mining will produce the best returns around the world in the next 20-30 years. Food prices are expected to go up because the growing middle class population in emerging markets will demand more expensive food including meat and more jewelry. Worldwide inflation will be higher than expected; so holding precious metals will be better than holding cash.

Nepal is close to India and China which will have the largest surge in the middle class population in the history of the world. As families become smaller and wealthier, they will start eating well. Meat consumption will rise. It will take more agricultural resources to produce more meat.

These families will also be more interested in traveling. Due to proximity, Nepal is a prime destination for both Chinese and Indian tourists.

Investment opportunities in Nepal

Land, Land and More Fertile Land

Although land prices in the Terai and rural areas in Nepal are starting to go up, they are still cheap. Land can provide dividends in terms of crops while waiting for the value to go up in a couple of decades.

Jeremy Grantham, a famous investor who predicted the 2007 financial meltdown, predicts that state-of-the-art organic farming is the best investment in the world now. In Nepal, we can combine both traditional and modern organic farming techniques to grow food in anticipation of the rapid food price increase.

Tourism Investments

Nepal is the country in the world which is sandwiched between two fastest growing large economies. India and China will have a huge growth in middle class population eager to travel. Nepal can tap the growing tourism market by anticipating where Chinese and Indian tourists may want to spend.

Buying shares in tourism-related stocks such as hotels, airlines or restaurants is a passive way to tap this potential. You can also open a resort or travel agency in anticipation of the boom. If you open up your own venture, the key to success is to be appealing and different from others.

Hydropower

Potential of hydropower in Nepal is huge. While investing in large-scale hydropower plants may need huge sum, it is possible to buy shares in related companies when they go public.

Several hydropower projects are underway in Nepal. In the next 10 years, several of them will become public. This will provide a good opportunity for an ordinary investor without millions of rupees to invest. For large scale investors, projects await in number. Nepal concluded a much awaited Power Trade Agreement with India in 2014 paving the way for trade of electric power just like other marketable commodities. This now ensures predictability of market once electricity is produced.

Outsourcing from US, Europe and Australia

Most foreign companies are looking to outsource programming, research or labor-intensive work to developing countries. These companies will save a substantial amount of money by doing so. Such outsourcing opportunities exist especially in web programming because of a large number of students interested in computer engineering in Nepal.

Medical Tourism

Nepal has produced some of the best doctors around the world. Surgeons from Nepali hospitals also do surgeries in Singapore, UK and USA. It is true that Nepali hospitals lack the infrastructure and equipment found in developed countries, but the quality of doctors is high.

So a good opportunity to invest is in medical tourism in Nepal. Healthcare services in the developed world is expensive. A simple bypass surgery costs around a hundred thousand dollars there, while the same surgery can be done in Nepal at a fraction of the price.

Since Nepal is a country endowed with natural beauty, visiting Nepal for pleasure as well as medical reasons should be an attraction.

Investment Board

With the view to attracting foreign as well as domestic Investment to boost the economy, the Government of Nepal has constituted a high level Investment Board, chaired by the Prime Minister of Nepal. The main objective of this board is to facilitate investors in investing in the potential sectors in Nepal. It provides one window facilities to the investors.

These are a number of investment ideas that can generate significant returns over the years. There will be even more opportunities in Nepal as the middle class population grow in Nepal. Rate of returns on investment in Nepal is high.

source: \url{https://mofa.gov.np/about-nepal/investment-in-nepal/}
\end{solution}

\end{questions}

\hypertarget{terrorism}{%
\subsection{Terrorism}\label{terrorism}}

Terrorism in its broadest sense, describes the use of intentionally indiscriminate violence as a means to create terror, or fear, to achieve political, religious or ideological aim. It is used in this regard primarily to refer to violence against peacetime targets or in war against non-combatants.

Terrorism has been practiced by political organizations with both rightist and leftist objectives, by nationalistic and religious groups, by revolutionarie, and even by state institutions such as armies, intelligence services, and police.

Terrorism is not legally defined in all jurisdictions; the statutes that do exist, however, generally share some common elements. Terrorism involves the use of threat of violence and seeks to create fear, not just within the direct victims but among a wide audience. The degree to which it relies on fear distinguishes terrorism from both conventional and guerilla warfare. Although conventional militiary foreces invariably engage in psychological warfare against the enemy, their principal means of victory is strength of arms. Similarly, guerrilla forces, which often rely on acts of terror and other forms of propaganda, aim at military victory and occassionally succeed (e.g.~the Viet Cong in Vietnam and the Khmer Rouge in Cambodia). Terrorism proper is thus the systematic use of violence to generate fear, and thereby to achieve political goals, when direct military victory is not possible. This has led to some social scientists to refer to guerrilla warfare as the ``weapon of the weak'' and terrorism as the ``weapon of the weakest''.

Terrorism has been described variously as both a tactic and strategy; a crime and a holy duty; a justified reaction to oppression and an inexcusable abomination. Obviously, a lot depends on whose point of view is being represented. Terrorism has often been an effective tactic for the weaker side in a conflict. As an asymmetric form of conflict, it confers coercive power with many of the advantages of military force at a fraction of the cost. Due to secretive nature and small size of terrorist organizations, they often offer opponents no clear organization to defend against or to deter.

This is why pre-emption is being considered to be so important. In some cases, terrorism has been a means to carry out a conflict without the adversary realizing the nature of the threat, mistaking terrorism for criminal activity. Because of these characteristics, terrorism has become increasingly common among those pursuing extreme goals throughout the world. But despite its popularity, terrorism can be a nebulous concept.

The US Department of Defense defines terrorism as ``the calculated use of unlawful violence or threat of unlawful violence to inculcate fear; intended to coerce or to intimidate governments or societies in pursuit of goals that are generally political, religious, or ideological.''

The UN produced the following definition of terrorism in 1992; ``An anxiety inspiring method of repeated violent action, employed by semi-clandestine individual, group or state actors, for idiosyncratic, criminal or political reasons, whereby -- in contrast to assassination -- the direct targets of violence are not the main targets.''

The strategy of terrorists is to commit acts of violence that draws the attention of the local populace, the government, and the world to their cause. The terrorists plan their attack to obtain the greatest publicity, choosing targets that symbolize what they oppose.

The effectiveness of the terrorist act lies not in the act itself, but in the public's or governments reaction to the act. For example, in 1972 at the Munich Olympics, the Black September organization killed 11 Israelis. The Israelis were the immediate victims but the true target was the estimated 1 billion people watching the televised event. Those billion people watching were to be introduced to fear -- which is terrorism's ultimate goal. The introduction of this fear can be from the threat of physical harm/a grizzly death, financial terrorism from the fear of losing money or negative effects on the economy, cyber terrorism harming the critical technological infrastructures of society and psychological terrorism designed to influence people's behavior.

There are three perspectives of terrorism: the terrorist's, the victim's, and the general public's. The phrase ``one man's terrorist is another man's freedom fighter'' is a view terrorists themselves would gladly accept. Terrorists do not see themseleves as evil. They believe they are legitimate combatants, fighting for what they believe in, by whatever means possible to attain their goals. A victim of a terrorist act sees the terrorist as a criminal with no regard for human life. The general public's view though can be the most unstable. the terrorists tke great pains to foster a Robin Hood image in hope of swaying the general public's point of view towards their cause. This sympathetic view of terrorism has become an integral part of their psychological warfare and has been countered vigourously by governments, the media and other organizations.

Technolgical advances, such as automatic weapons and compact electrically detonated explosives, gave terrorists a new mobility and lethality, and the growth of air travel provided new methods and opportunities. Terrorism was virtually an official policy in totalitarian states such as those of Nazi Germany under Adolf Hitler and the Soviet Union under Stalin. In these states arrest, imprisonment, tourture, and execution were carried out without legal guidance or restraints to create a climate of fear and to encourage adherece to the national ideology and the declared economic, social, and political goals of the state.

One of the deadliest terrorist strikes to date were the September 11 attacks of 2001, in which suicide terrorists associated with al-Qaeda hijacked four commercial airplanes, crashing two of them into the twin towers of the World Trade Center complex in the New York city and the third into the Pentagon building near Washington D.C.; the fourth plane crashed near Pittsburgh, Pennsylvania. The crashes destroyed much of the World Trade Center complex and a large portion of one side of the Pentagon and killed more than 3000 people.

Terrorism appears to be an enduring feature of political life. Even prior to the September 11 attacks, there was widespread concern that terrorist might escalate their destructive power to vastly greater proportions by using weapons of mass destruction -- including nuclear, biological, or chemical weapons -- as was done by the Japanese doomsday cult AUM Shirnrikyo, which released nerve gas into a Tokyo subway in 1995.

Factors conducive to terrorism

\begin{enumerate}
\def\labelenumi{\arabic{enumi}.}
\tightlist
\item
  Poverty
\item
  Globalization
\item
  Legitimate grievances and failure of governments.
\item
  Humiliation
\item
  Lack of democracy, and widespread and systematic violation of human rights
\item
  Lopsided foreign policy
\item
  Failed state governance and authority
\end{enumerate}

Goals, strategies, adn weapons of terrorism

\begin{enumerate}
\def\labelenumi{\arabic{enumi}.}
\tightlist
\item
  Social and political justice
\item
  Self determination
\item
  Racial superiority
\item
  Foreign policy
\item
  Publicty
\item
  Demoralized governments
\end{enumerate}

Costs associated with terrorism are so widespread, complex, and intangible that they are virtually impossible to measure. Individuals, families, governments, companies and non-state actors bear the time, money and resource costs of terrorism to varying degrees. Migration, trade, travel and interpersonal relations are affected. It is generally believed that the financial crisis and the global recesssion were caused in by policies adopted by the Bush administration to fight global terrorism, including wars in Afganistan and Iraq.

Kinds of terrorism

Although the types of terrorism tend to overlap, they vary in their implications and affect us in different ways. For example, the indiscriminate nature of global terrorism contrasts sharply with domestic terrorism aimed at specific groups or governments.

\begin{enumerate}
\def\labelenumi{\arabic{enumi}.}
\tightlist
\item
  Domestic terrorism occurs within the borders of a particular country and is associated with extremist groups.
\item
  Nationalist terrorism is closely associated with struggles for political autonomy and independence.
\item
  Religious terrorism grows out of extreme fundamentalist religous groups that believe that God is on their side and that their violence is divinely justified and inspired.
\item
  State terrorism is a cold, calculated, efficient and extremely destructive form of terrorism, partly because of the overwhelming power at the disposal of governments.
\end{enumerate}

Leading terrorist organizations

\begin{itemize}
\tightlist
\item
  ISIS (part of northern Iraq and parts of Western Syria)
\item
  Al-Qaeda (Afganistan)
\item
  Taliban
\item
  Boko Haram (now sworn to ISIS; Nigeria)
\item
  Lashkar-e-toa (Pakstan)
\item
  Hezbollah (Lebanon)
\item
  Al-Shabaab
\item
  Hamas (Palestine)
\item
  Naxal/Naxalites (India)
\item
  Irish Republican Army
\end{itemize}

No cause justified terrorism. The world must repond and fight vigillantly with this evil that is intent on threatening and destroying our basic feedoms and way of life. The complexity of terrorists requires employing a wide variety of instruments to combat it. It addition to military, traditional law enforcement, intelligent responses, and global cooperation has become obvious to counter terrorist attacks.

The most prevalent response to terrorism is the use of force, both domestically and internationally. Besides, reduction in the production of weapons of mass destruction and control over media and technology can be effective.

We must retaliate forcefully and successfully in order to reassure the world of our confidence as a great power, and our ability to retain that status. Lastly, and practically, we must explore the possibilities of terrorism that exist and attempt to grasp the actual threat that these possibilities realistically pose for us.

\hypertarget{globalization}{%
\subsection{Globalization}\label{globalization}}

Globalisation is the process of interaction and integration among people, companies, and governments worldwide. As a complex and multifaceted phenomenon, globalization is considered by some as a form of capitalist expansion which entails the integration of local and national economies into a global, unregulated market economy. Globalization has grown due to advances in transportation and communication technology. With the increased global interactions comes the growth of international trade, ideas, and culture. Globalization is primarily an economic process of interaction and integration that's associated with social and cultural aspects. However, conflicts and diplomacy are also large parts of the history of globalization, and modern globalization.

Economically, globalization involves goods, services, the economic resources of capital, technology, and data. Also, the expansions of global markets liberalize the economic activities of the exchange of goods and funds. Removal of cross-border trade barriers has made formation of global markets more feasible.{[}citation needed{]} The steam locomotive, steamship, jet engine, and container ships are some of the advances in the means of transport while the rise of the telegraph and its modern offspring, the Internet and mobile phones show development in telecommunications infrastructure. All of these improvements have been major factors in globalization and have generated further interdependence of economic and cultural activities around the globe.

Though many scholars place the origins of globalization in modern times, others trace its history long before the European Age of Discovery and voyages to the New World, some even to the third millennium BC. Large-scale globalization began in the 1820s. In the late 19th century and early 20th century, the connectivity of the world's economies and cultures grew very quickly. The term globalization is recent, only establishing its current meaning in the 1970s.

Three major dimensions of globalization can be differentiated. They are:

Economic globalization

It is the increasing economic interdependence of national economies across the world through a rapid increase in cross-border movement of goods, services, technology, and capital. Whereas the globalization of business is centered around the diminution of international trade regulations as well as tariffs, taxes, and other impediments that suppresses global trade, economic globalization is the process of increasing economic integration between countries, leading to the emergence of a global marketplace or a single world market. Depending on the paradigm, economic globalization can be viewed as either a positive or a negative phenomenon. Economic globalization broadly comprises: globalization of production and market but also includes competition, technology, and corporations and industries.

Cultural globalization

Cultural globalization refers to the transmission of ideas, meanings, and values around the world in such a way as to extend and intensify social relations. This process is marked by the common consumption of cultures that have been diffused by the Internet, popular culture media, and international travel. This has added to processes of commodity exchange and colonization which have a longer history of carrying cultural meaning around the globe. The circulation of cultures enables individuals to partake in extended social relations that cross national and regional borders. The creation and expansion of such social relations is not merely observed on a material level. Cultural globalization involves the formation of shared norms and knowledge with which people associate their individual and collective cultural identities. It brings increasing interconnectedness among different populations and cultures. Cross-cultural communication, cultural diffusion and religious thought-schools are the major aspects of cultural globalization. Sports too bear the spirit of globalization; for eg., the FIFA World Cup is the most widely viewed and followed sporting event in the world, exceeding even the Olympic Games; a ninth of the entire population of the planet watched the 2006 FIFA World Cup Final.

Political globalization

Political globalization refers to the growth of the worldwide political system, both in size and complexity. That system includes national governments, their governmental and intergovernmental organizations as well as government-independent elements of global civil society such as international non-governmental organizations and social movement organizations. One of the key aspects of the political globalization is the declining importance of the nation-state and the rise of other actors on the political scene. Adherence to or practice of intergovernmentalism, multi-level governance and multiple citizenship reflect today's globalized politics, although it should also be noted that some countries, like North Korea, have embraced isolationist policies.

Aside from the the above mentioned aspects, globalization touches upon a wide range of issues including population migration, information exchange, environmental protection.

Although globalization, according to many political scientists, is not a new phenomenon and that countries pursued it in various ways since the birth of the nation state, nobody denies the fact that the current phase of globalization is different-- different in the sense that it unleashed three major trends. The first is denationalization of the state involving major changes in the power of the legislature, the executive and the judiciary and shifts in the relative weight of financial, technological, environmental, security and other organs. The second is the de-statization of politics making the state's involvement in society less hierarchical, decentralized, poly-centric and multi-tiered involving both territorial and functional units. And, finally, the internationalization of the policy regime increasing the strategic significance of the international context of domestic state action and blurring the distinction between domestic and foreign policy regimes. The main issue of ``governance in a global economy is the loss of monetary and fiscal options for the nation state and subservience to rules sometimes made by the international trade and financial interests''. Argentina presents an extreme case of the loss of monetary sovereignty. To participate in the global economy, it surrendered its economic policy prerogative to the US. The communications revolution has truly made it possible for different global power centers to execute their agendas almost immediately. If this has opened new unforeseen opportunities for global peace and prosperity, at the same time, risks have also multiplied by as many times.

\hypertarget{national-security}{%
\section{National security}\label{national-security}}

The promulgation of the much-awaited constitution by the Constituent Assembly in September and the subsequent Madhes movement have revealed some complicated and sensitive security threats for Nepal as a nation-state. Ensuring national unity, territorial integrity, people's sovereignty and citizens' security have always been highly sensitive issues for Nepal. But the Madhes movement and subsequent `unofficial' blockade by India have demonstrated some critical internal as well as external security threats for Nepal, undermining its sovereignty, independence and citizens' security. Therefore, taking into consideration the emerging global, regional and domestic security and political contexts, Nepal needs to objectively identify its internal and external threats and should draft a comprehensive national security policy sooner than later, if Nepal is to survive as a nation-state and ensure security of its citizens.

External threats

Following the end of the Cold War and the successive wave of democracy and globalisation, the theoretical concept of security has significantly changed. There has been a redefinition of the traditional notions of security, threat perspectives and role of security agencies. Despite the changing definition of security, the principle external threats remain unchanged. Although Nepal has not fought a war with any country after the Sugauli Treaty excluding a brief war with Tibet in 1856, it is not an external threat-free country. The recent coercive diplomacy and blockade by India have proved it. The global and regional powers' strategic rivalry vis-a-vis Nepal has significant internal and external security implications that have, to some extent, undermined Nepal's sovereignty and independence.

External military intervention is less likely in the 21st century. But according to the realist school of thought, for powerful countries everything is fair to achieve their national interests against weak and developing countries; hence they can go to any extent. Protecting national interests, territorial integrity and people's sovereignty are crucial issues for a buffer state like Nepal sandwiched between two emerging global powers ---India and China. Nepal should take into account its geo-political sensitivity and geo-strategic importance while drafting national security and foreign policy without hurting their fundamental strategic and diplomatic interests. Therefore, Nepal should maintain diplomatic and strategic balance with both its neighbours and with other global powers. Otherwise, it may transform into a battleground for regional and global powerhouses.

Like military intervention and economic blockade, border encroachment, international terrorism, religious fundamentalism, refugee flows, migration, trans-border crimes, arms smuggling and climate change are equally serious external threats to Nepal. Nepal also has border disputes and a trade deficit with India, which are not conducive to cordial Nepal-India relations. Therefore, Nepal needs to develop commercial as well as strategic physical infrastructure with both neighbours in line with its national interests and national security to achieve its strategic objectives.

Internal threats

Nepal is not an external threat-free country, but analysing the emerging threats, internal threats seem more sensitive, complicated and challenging, as more than 17,000 Nepali people lost their lives in a decade-long internal conflict, and the ongoing violent Madhes movement has created some crucial threats. However, the current Madhes movement is a political movement; hence, the government should not suppress it. Instead, the government should resolve the problem through dialogue addressing the movement's legitimate demands. But the movement has exemplified some critical security threats that the government needs to address urgently. If it fails to do so, there is a high possibility of instigating another kind of conflict that may turn into a multidimensional (ideological, religious, communal, ethnic, regional) one sooner or later. The potential multidimensional conflict may weaken the centuries-old social, cultural and religious harmony of Nepal. If the government and political parties cannot deal with these very sensitive issues appropriately, there might be a turn towards religious and ethnic conflicts and secessionist movements.

Violence and crime seem to be a legacy of the decade-long conflict, and they seem to have become an integral part of society, destroying the very Nepali identity of peace and harmonious existence. Analysing the facts and figures, there are some conventional threats like poverty, unemployment, economic disparity, armed-conflict, criminal groups, arms infiltration, impunity and domestic violence. But some threats like environmental degradation, climate change, natural disaster, food insecurity, migration, etc., are unconventional. Likewise, non-state actors have been weakening and paralysing the state agencies and causing threats that may well emerge as the most serious ones in the future.

Redefining security policy

National security and people's security are very serious, challenging and complicated issues for Nepal. But, unfortunately, Nepal does not have a comprehensive written national security policy, and it has been following ad-hoc and outdated approaches even after the historic change of 2006. The successive governments and political parties did not pay heed to national as well as people's security policy; they merely beat the drum of ultranationalism and national independence for public consumption. Following the promulgation of the constitution, it is high time Nepal drafted a comprehensive national security policy. The new policy should be redefined in line with the changed political and security contexts for three reasons.

First, Nepal is in the process of transforming into a federal republic state from a feudal and unitary kingdom, and the security threats for these two types of regimes are fundamentally different. Second, the new constitution has defined human security as a guiding principle of national security, according to which, the role of the state is not only limited to defending its territory, but also to ensure freedom, human rights, peace and security of its citizens. The concept of security, therefore, has shifted focus from state to human-centric approaches. Third, the internal and external threats to Nepal seem to be more complicated and challenging than before, so the new policy should clearly identify both threats and outline strategies to address them.

To conclude, the new government needs to draft and implement a national security policy to safeguard national unity, territorial integrity, people's sovereignty and national interests based on national consensus. Moreover, the policy should embrace the fundamental principles of democracy, rule of law, human rights, and protect the long cherished identity, values and cultures of Nepal. The policy should also recognise the diversity and plurality of Nepali society, but it should not be used as a tool to suppress the people on the pretext of national security.

Wagle is a research fellow of Birmingham University, UK

source: \url{https://kathmandupost.com/opinion/2016/01/08/defending-the-nation}

\hypertarget{order-on-the-border}{%
\section{Order on the border}\label{order-on-the-border}}

Nepal and India have an open border of more than 1,700 km. India borders Nepal on three sides---east, west and north---while China's Tibet Autonomous Region is to the north; thus sandwiching Nepal between the two Asian giants. In territorial, demographic, developmental and strategic terms, there is no comparison between Nepal and its two neighbours. For a variety of reasons, these asymmetries have also provided Nepal with a sense of national security, given that the quirks of today's international politics is impacted as much by conventional strategic competition for great power status and hegemony as by rising non-state actors whose orientation is determined by religious fundamentalism, ethnicity and de-ideologised politics.

Open history

The open border between Nepal and India is neither a new phenomenon nor an imposition by the inequality of powers. It is not a porous border but it can be called an open border for a variety of reasons: unrestricted movement of the peoples of India and Nepal; more-or-less equal treatment meted out

to Nepalis in India and Indians in Nepal; and continuing historical relations in political, social, economic and other fields. The post Sugauli Treaty relationship cemented these political, economic and security relations in the absence of any third power in Nepal's vicinity.

The de facto Chinese presence came much later in 1949, following the emergence of Red China. Yet, such a change did not disturb Nepal-India bilateral relations, although Nepal had to make some adjustments to the emergent geopolitical context. The Nepal-India Treaty of Peace and Friendship of 1950 was only a modified version of the 1923 Nepal-Britain Treaty. And as of today, the 1950 treaty continues to provide a framework for relationships despite being called an `unequal treaty' concluded by the beleaguered Rana rulers.

It is interesting to note that this treaty does not specifically mention `open border', as it simply deals with the special privileges to be granted to the citizens of both countries. Debates relating to the movement of peoples and its implications for changing the demographic balance in Nepal saw their peak in the 1980s and 90s with political party leaders, academics and policymakers trying to link up the open border with the free movement of peoples. Arguing that continuity for such a relationship would pose a serious threat to Nepal's security, they demanded an end to the open border.

Contrary to such views, an American political scientist, Myron Weiner, had already made some observations on the basis of his research and had concluded India to be ``a safety-valve'' for Nepal. His article, `The Political Demography of Nepal', was presented in a seminar in Nepal in 1971 and was published in Asian Survey (Berkeley) in 1973. From time immemorial, the movement of Nepalis and Indians has continued uninterruptedly without losing Nepal's national identity, sovereignty or territorial integrity.

Continuity and change

Thus, taking an interest in the open border regime, the BP Koirala India-Nepal Foundation supported a field-based study of the open border in 2012-13. The Report deals with a variety of aspects, such as benefits and flaws; border management; encroachment controversy; use and misuse of the open border; No Man's Land; human and drug trafficking; security/insecurity; smuggling and the role of state agencies of both countries; water logging due to dams and barrages; and the use of the border for mutual benefits in education, health, livelihood and other socio-cultural relations.

The open border arrangement has been marked by both continuity and change. Its characteristics and compelling reasons give it a permanent feature but emergent trends and imperatives also make it change. Culturally, historically, economically and practically, the open border has been beneficial for both peoples. However, continuity also needs to address the emerging complexity and the problems that arise from time to time. Taking these aspects into account, the two governments have taken steps to streamline the open border. The deployment of the Seema Surakshya Bal (SSB) by India along the entire Indo-Nepal border, the border duties given to the Armed Police Force (APF) by Nepal and the creation of Coordinating Committees that meet when the need arises have considerably improved the regulation problem.

But it has been felt that these coordination committees should meet regularly to both generate trust and cooperation as well as to address the immediate problems that can arise in the field. India's SSB has six principal duties---stop smuggling, the movement of contraband goods and import/export without clearing custom duties; peace keeping and maintenance of internal security; control of criminals using the open border; security of custom and border areas, border pillars; prevent human trafficking; prevent transaction of arms and ammunition and fake currencies. SSB posts are placed along less than 4km intervals in order to enhance their role in regulating the border.

Nepal's APF is less equipped due to a lack of resources, both material and human; hence, skeleton check posts exist only in the Tarai. Our study finds that by and large, there has been a cooperative spirit between the SSB and APF, who meet from time to time to resolve immediate problems. It should be understood that Nepal-India relations are not conducted only through formal mechanisms and practices. Informal cooperation and coordination between security agencies of the two countries has tackled some immediate security and criminal-related problems. Sometimes, such operations go beyond the established norms of cooperation and trigger controversies but efforts should be made to respect each others' sensitivity while conducting such operations.

Delegate authority

Yet, local authorities and security agencies have no mandate to fix new pillars or maintain No Man's Land. The two governments could delegate authority to place new pillars and demarcate the boundary once they decide to allocate such duties to local level authorities.

First, the border should be demarcated on the basis of longitude-latitude so that missing areas can be easily located. When rivers change their course during the monsoon or pillars get lost, such an approach would be useful. The Indo-Nepal border also has a human dimension as children, women and others work as carriers of goods for smugglers. Community-level awareness, the improvement of economic conditions and educational opportunities for children living near the border could help minimise the problem.

Border regulation has become much more streamlined in recent years. But some territories---such as Susta, Pashupatinagar in Ilam and Kalapani---continue to remain disputed. A prompt and cooperative attitude by both sides can resolve these issues. A scientific map prepared by the two sides and a willingness to act on the basis of such a map can settle the vexing problem. A taskforce can be created to finalise the map. Indian Prime Minister Narendra Modi's visit to Nepal could be used as an opportunity to take such a decision and reach an agreement within a timeframe.

We should also realise that an international open border will not be completely free of occasional problems, as some pinpricks may arise from time to time without impairing the basics of the existing relationship. But vigilance on both sides and the delegation of authority to the local level can manage such recurring problems.

Baral was Coordinator of the BPKF Open Border Study Team and is a former Nepali Ambassador to India.

source: \url{https://kathmandupost.com/opinion/2014/06/24/order-on-the-border}

\hypertarget{least-developed-countries}{%
\section{Least developed countries}\label{least-developed-countries}}

The least developed countries (LDCs) is a list of the countries that, according to the United Nations, exhibit the lowest indicators of socioeconomic development, with the lowest Human Development Index ratings of all countries in the world. The concept of LDCs originated in the late 1960s and the first group of LDCs was listed by the UN in its resolution 2768 (XXVI) of 18 November 1971.

The LDCs are defined as low-income countries that are suffering from long-term impediments to growth. They have low levels of human resource development and are vulnerable to both socio-economic, and environmental shocks.

\hypertarget{criteria-for-classification}{%
\subsection{Criteria for classification}\label{criteria-for-classification}}

Countries are categorized as least developed if they meet the following three criteria,

\begin{enumerate}
\def\labelenumi{\arabic{enumi}.}
\tightlist
\item
  Income per capita (poverty indicator): Adjustable criterion based on GNI per capita averaged over three years. As of 2015, a country must have GNI per capita less than US \$1035 to be included on the list, and over \$1242 to graduate from it.
\item
  Human resources weakness (assets; Human Asset Index) based on indicators of nutrition, health, education and adult literacy rate.
\item
  Economic vulnerability based on instability of agricultural production, instability of exports of goods and services, economic importance of non-traditional activities, merchandise export, concentration, handicap of economic smallness, and the percentage of population displaced by natural disasters.
\end{enumerate}

LDC criteria are reviewed every three years by the Committee for Development Policy (CDP) of the UN Economic and Social Council (ECOSOC). Countries may ``graduate'' out of the LDC classification when indicators exceed these criteria if so happens in two consecutive reviews qualify for graduation from the LDC category. They also quality if the GNI per capita of the country is at least twice the graduation threshold (\$2484) in two consecutive reviews.

Since the inception of LDCs categorization following countries have graduated:

\begin{enumerate}
\def\labelenumi{\arabic{enumi}.}
\tightlist
\item
  Botswana (1994)
\item
  Cabo Verde (2007)
\item
  Maldives (2011)
\item
  Samoa (2014)
\item
  Equatorial Guinea (2017)
\item
  Vanuantu and Angola (Scheduled for 2020 and 2021, respectively)
\end{enumerate}

Following is the listing of the third world countries (UNCTAD, Jan 2021):

\begin{table}

\caption{\label{tab:ldcs-unctad}List of LDCs (2020)}
\centering
\begin{tabular}[t]{rlrl}
\toprule
SN & Country & SN & Country\\
\midrule
\rowcolor{gray!6}  1 & Afghanistan & 24 & Madagascar\\
2 & Angola & 25 & Malawi\\
\rowcolor{gray!6}  3 & Bangladesh & 26 & Mali\\
4 & Benin & 27 & Mauritania\\
\rowcolor{gray!6}  5 & Bhutan & 28 & Mozambique\\
\addlinespace
6 & Burkina Faso & 29 & Myanmar\\
\rowcolor{gray!6}  7 & Burundi & 30 & Nepal\\
8 & Cambodia & 31 & Niger\\
\rowcolor{gray!6}  9 & Central African Republic & 32 & Rwanda\\
10 & Chad & 33 & Sao Tome and Principe\\
\addlinespace
\rowcolor{gray!6}  11 & Comoros & 34 & Senegal\\
12 & Democratic Republic of the Congo & 35 & Sierra Leone\\
\rowcolor{gray!6}  13 & Djibouti & 36 & Solomon Islands\\
14 & Eritrea & 37 & Somalia\\
\rowcolor{gray!6}  15 & Ethiopia & 38 & South Sudan\\
\addlinespace
16 & Gambia & 39 & Sudan\\
\rowcolor{gray!6}  17 & Guinea & 40 & Timor-Leste\\
18 & Guinea-Bissau & 41 & Togo\\
\rowcolor{gray!6}  19 & Haiti & 42 & Tuvalu\\
20 & Kiribati & 43 & Uganda\\
\addlinespace
\rowcolor{gray!6}  21 & Laos & 44 & United Republic of Tanzania\\
22 & Lesotho & 45 & Yemen\\
\rowcolor{gray!6}  23 & Liberia & 46 & Zambia\\
\bottomrule
\end{tabular}
\end{table}

There are 46 LDCs, with a population of close to one billion, representing about 12\% of the world's population, but account for less than 2\% of the world GDP and 1\% of global trade in goods.

\hypertarget{problems-characterizing-ldcs}{%
\subsection{Problems characterizing LDCs}\label{problems-characterizing-ldcs}}

\begin{itemize}
\tightlist
\item
  Weak human and institutional capacities
\item
  Low per capita income
\item
  Unequal distribution of income
\item
  Unemployment and systemic poverty
\item
  Poor educational status
\item
  Discrimination, injustice and social inequalities
\item
  Human rights violation
\item
  Low saving and investing incentives
\item
  Poor infrastructure development
\item
  Scarcity of domestic financial resources
\item
  Governance crisis (inefficient, corrupt, opaque)
\item
  Political instability
\item
  Internal and external conflicts
\item
  Agrarian economy with low productivity and low investment
\item
  Low volume of trade
\item
  Demographic explosion
\item
  Poor access to technology and connectivity
\item
  Poor social welfare and brain drain
\item
  Lack of security
\item
  Massive political and or bureaucratic intervention
\item
  Weak internal politics thereby giving way for external interference
\item
  Low industrialization and capital intensive technologies
\item
  Poor resource management and investment capacity
\item
  Weakness in program design and implementation
\item
  High debt burden and dependence on external financing
\end{itemize}

The United Nations Office of the High Representative for Least Developed Countries, Landlocked Developing Countries and Small Island Developing States (UN-OHRLLS) was established by General Assembly Resolution 56/227 as a follow-up mechanism to LDC-III to ensure effective follow-up, implementation, monitoring and review of the implementation of the Brussels Program of Action for the LDCs for the Decade 2001-2010, adopted at the conference.

Despite three successive programs of action and notwithstanding the positive developments recorded by LDCs in the recent past, most of these countries have not met the internationally agreed goals, including the MDGs, and still face massive development challenges.

With a view to exactly tackle that emergency and to reinvigorate the pledge in support of LDCs development and transformation, the international community met in Istanbul, Turkey, for the Fourth United Nations Conference on the LDCs (LDC-IV). The event took place from 9 to 13 May, 2011. It provided a major opportunity to deepen the global partnership in support of LDCs and set the framework for development cooperation for the next decade. The UN-OHRLLS is the coordinator of the LDC-IV process.

The intention behind the creation of the group explicitly identifying the least developed countries is to sensitize the development problems facing these countries among the greater world and thereby urging the international community including the UN to undertake additional, specific and effective measures to support nationally designed development efforts of the LDCs. The UN committee for development policy is responsible for setting and reviewing of the LDCs criteria.

After being recognized as so, these countries may be eligible for a number of special international, regional and bilateral support measures. Basically, the following benefits, opportunities or treatments are available to the LDCs in the area of foreign aid, trade and capacity building.

\begin{itemize}
\tightlist
\item
  Development financing/Foreign aid
\item
  Preferential market (Duty free/Quota free access)
\item
  Special treatment regarding WTO related obligations (Import/export access, technical assistance, trade related facilities)
\item
  Technical cooperation
\end{itemize}

It is not an easy task to resolve all the problems of LDCs at once. However, strong political will is quite necessary to develop a country utilizing and mobilizing available resources (both human and material). Moreover, no single country can remain isolated in a globalization oriented world and connectivity leads to realization and reflection to the prevailing situations of the country striving for progress.

We often hear the analogy that when larger developed nations sneeze, the less developed catch the cold. Building and maintaining the resilience of LDCs to external shocks continues to be a high priority rapidly changing geopolitical, economic and natural environment - with the gamut of challenges from high debt levels, changing migration patterns and projected slowdown in the global economy -- LDCs need support to be able to assess and mitigate the risks and alter their strategies accordingly.

In this day and age of unprecedented global prosperity, rapid advances in science, technoogy and globalization, the citizens of LDCs should not remain entrapped in a vicious cycle of poverty, deprivation and inequity. LDCs represent an enormous human and natural resource potential for the world economic growth, welfare and prosperity. There must be an active effort on part of the developed countries to help identify the strengths, weaknesses, opportunities and threats facing the LDC. Afterall, with the globe as a single community, everyone should play a part in helping their weaker neighbor.

\hypertarget{priorities-for-ldcs-by-the-international-community}{%
\subsection{Priorities for LDCs by the international community}\label{priorities-for-ldcs-by-the-international-community}}

\begin{enumerate}
\def\labelenumi{\arabic{enumi}.}
\tightlist
\item
  A significant reduction in extreme poverty;
\item
  Developing human and institutional resources to support sustained growth and sustainable development;
\item
  Removing the supply side constraints and enhancing productive capacity and promoting the expansion of domestic markets to accelerate growth, income and employment generation;
\item
  Accelerating LDCs' growth with the aim of enhancing their share in world trade and global financial and investment flows;
\item
  Environmental protection, accepting that LDCs and industrialized countries have common but different responsibility;
\item
  Attaining food security and reducing malnutrition
\item
  Specific solutions to overcome problems of LDCs
\end{enumerate}

Graduation from least developed country (LDC) status becomes effective three years after the United Nations General Assembly takes note of the recommendation made by the Committee for Development Policy under the United Nations Economic and Social Council to graduate a country. This means Nepal's graduation will be effective in 2024, if the committee recommends graduation at its next triennial review in 2021.

As of 2020, the gross national per capita income (GNI) of a country needs to be at least \$1,222 for it to be accorded developing country status. At present Nepal's per capita income is \$1090, according to the World Bank.

In the last two triennial reviews conducted in 2015 and 2018, Nepal had met two of the three criteria related to human asset index and economic vulnerability index. It couldn't, however, meet the per capita income criterion.

Nepal must prepare a transition strategy in cooperation with trade and development partners to avoid adverse impacts from the country's graduation from least developed country status as it risks losing preference treatment after graduation, a government report says.

While preparing the transition strategy, Nepal also needs to conduct a fresh review of the scheduled graduation plan considering the impact of the impact of the Covid-19 pandemic (Nepal Human Development Report 2020: Beyond LDC Graduation: Productive Transformation and Prosperity).

There are differences in opinion whether rush towards graduation is a right option currently or not. Graduation makes countries more creditworthy by international credit rating agencies, thus improving access to commercial finance besides emitting a positive message to the global community about Nepal's development prospects. It can help potential branding as competitive destination for foreign direct investment inflows and other private investment. Despite some obvious benefits, there are also costs to Nepal suggesting that a delay in graduation might be more worthwhile in long term,

\begin{enumerate}
\def\labelenumi{\arabic{enumi}.}
\tightlist
\item
  Resilient production sector such as manufacturing and service industry having pharmaceutical, garments and information technology products (similar to that of Bangladesh) are lacking.
\item
  Foreign direct investment and the country's per capita GNI are low.
\item
  International aid agencies, those that provide development support in Nepal, like UNDP and UNICEF have been allocating 60\% of their core resource to the LDCs. Likewise, World Meterological Organization, International Telecommunications Union and Universal Postal Union also conduct programs specific to LDCs in Nepal. After graduation, access to such support programs will be halted.
\item
  Nepal will have to face loss of preferential market access and increased competition in international markets which would disproportionately impact export-oriented small and medium enterprises and employment generated by them.
\item
  The legal obligation of fully implementing the Trade Related Intellectual Property Rights (TRIPs) Agreement may negatively affect access to medicine, and in turn. health care.
\end{enumerate}

\hypertarget{landlocked-developing-countries}{%
\section{Landlocked developing countries}\label{landlocked-developing-countries}}

A land locked country is one that needs to traverse, at least, one another country to grain access to the sea. A land locked country is defined as a state that has no sea-coast. In practical terms, land locked countries may be located in the interior part of the continent far away from seas or ocean. There are 44 land locked countries in the world.

They do not have sea coast at their borders. They need to take the help of at least one another country to reach to the sea ports. All LLDCs share common problem of geographical remoteness and dependence on trade and transport systems in neighboring and coastal countries. The location of LLDCs in the interior part of the continents requires their export and import goods to travel hundreds, if not thousands of kilometers to and from the closest maritime ports.

The landlocked problem affects all aspects of developmetn, but its consequences are most severe in the field of external trade. Most land locked countries are remote from world markets: import and export need to be carried out through another state, in some cases several states. This gives rise to additional costs which reduce the competitiveness of their products on overseas markets and have a negative effect on their balance of payments as a result of higher import bills and currency outflows.

Lack of access to sea also causes legal, administrative and political problems. Land locked countries have to negotiate to meet their transit requirements, and the facilities provided may not in every case be satisfactory because the transit countries have their own development priorities and problems, which are often acute. They may, for example, impose cumbersome documentation procedures and formalities in order to safe guard their interest which could be jeopardized by the diversion of transit cargoes to their domestic markets.

\begin{table}

\caption{\label{tab:lldcs}List of landlocked countries of the world}
\centering
\fontsize{8}{10}\selectfont
\begin{tabular}[t]{>{\raggedleft\arraybackslash}p{3em}>{\raggedleft\arraybackslash}p{6em}>{\raggedleft\arraybackslash}p{3em}>{\raggedleft\arraybackslash}p{6em}>{\raggedleft\arraybackslash}p{3em}>{\raggedleft\arraybackslash}p{6em}>{\raggedleft\arraybackslash}p{3em}>{\raggedleft\arraybackslash}p{6em}}
\toprule
\multicolumn{2}{c}{Africa} & \multicolumn{2}{c}{Asia} & \multicolumn{2}{c}{Europe} & \multicolumn{2}{c}{South America} \\
\cmidrule(l{3pt}r{3pt}){1-2} \cmidrule(l{3pt}r{3pt}){3-4} \cmidrule(l{3pt}r{3pt}){5-6} \cmidrule(l{3pt}r{3pt}){7-8}
SN & Country & SN & Country & SN & Country & SN & Country\\
\midrule
\rowcolor{gray!6}  1 & Botswana & 16 & Afganistan & 26 & Armenia & 30 & Bolivia\\
2 & Burkina faso & 17 & Bhutan & 27 & Azarbaijan & 31 & Paraguay\\
\rowcolor{gray!6}  3 & Burundi & 18 & Kazakhstan & 28 & Republic of Moldova &  & \\
4 & Central african republic & 19 & Kyrgyzstan & 29 & The former Yugoslav Republic of Macedonia &  & \\
\rowcolor{gray!6}  5 & Chad & 20 & Lao peoples' democratic &  &  &  & \\
\addlinespace
6 & Ethiopia & 21 & Mongolia &  &  &  & \\
\rowcolor{gray!6}  7 & Lesotho & 22 & Nepal &  &  &  & \\
8 & Malawi & 23 & Tajikistan &  &  &  & \\
\rowcolor{gray!6}  9 & Mali & 24 & Turkmenistan &  &  &  & \\
10 & Niger & 25 & Uzbekistan &  &  &  & \\
\addlinespace
\rowcolor{gray!6}  11 & Rwanda &  &  &  &  &  & \\
12 & Swaziland &  &  &  &  &  & \\
\rowcolor{gray!6}  13 & Uganda &  &  &  &  &  & \\
14 & Zambia &  &  &  &  &  & \\
\rowcolor{gray!6}  15 & Zimbabwe &  &  &  &  &  & \\
\bottomrule
\end{tabular}
\end{table}

\hypertarget{characteristics-of-the-lldcs}{%
\subsection{Characteristics of the LLDCs}\label{characteristics-of-the-lldcs}}

\begin{enumerate}
\def\labelenumi{\arabic{enumi}.}
\tightlist
\item
  Physical isolation, geographical dispersal, and distance from the main markets, many countries being small island or landlocked developing countries;
\item
  Insignificant participation in the multilateral trading system and a minimal share of total world trade;
\item
  Small, fragmented and highly imperfect markets;
\item
  Very open economy in general;
\item
  Minimal or no export diversification;
\item
  Dependence on very few export markets;
\item
  Inadequate infrastructure;
\item
  High degree of vulnerability;
\item
  Low competitiveness;
\item
  Low levels of productivity and insufficient supply;
\item
  Political instability;
\item
  Weak institutional and productive capacities;
\item
  High transport and transit costs;
\item
  Extreme poverty;
\item
  Weak human development and brain drain;
\item
  Lack of adequate market access opportunities;
\item
  Difficult to attract FDI;
\item
  Political and bureaucratic corruption
\end{enumerate}

Despite significant technological improvements in transport, landlocked developing countries in Africa and Asia continue to face structural challenges to accessing world markets. As a result, landlocked countries often lag behind their maritime neighbors in overall development and external trade. While the poor economic stance of landlocked countries is often solely attributed to geographical distance from the coast, the situation is more arising from the dependence for passage on the host country in various other ways, some of which are:

\begin{enumerate}
\def\labelenumi{\arabic{enumi}.}
\tightlist
\item
  Dependence on neighbors 'infrastructure
\item
  Dependence on sound cross-border political relations
\item
  Dependence on neighbors' peace and stability
\item
  Dependence on neighbors' administrative practices
\end{enumerate}

According to WB, 1999, LLDCs are paying around 50\% more in transport costs than coastal countries, and have up to 60\% lower volumes of trade. Inefficient customs and transit transport procedures are considered to be the main cause of delays and high transport costs and present a huge obstacle to trade for LLDCs.

\hypertarget{solutions-to-mitigating-problems-of-lldcs}{%
\subsection{Solutions to mitigating problems of LLDCs}\label{solutions-to-mitigating-problems-of-lldcs}}

\begin{itemize}
\tightlist
\item
  Enhancement of physical infrastructure and transport sytem
\item
  Maintenance of infrastructure, including improvement of border crossing facilities
\item
  Strengthening of international customs transit systems
\item
  Improvement of transit facilities and support services
\item
  Promotion of collaboration between public and private sector
\item
  Minimize documentation and other formalities
\item
  Control excess slowing down or a prolonged stoppage
\item
  Allow rapid transit without bureaucratic delay
\item
  Eliminate numerous taxes, fees and informal payments
\item
  Human power development and strategies to control brain drain
\item
  Expansion of regional and international trade by increasing export
\item
  Attraction of FDI by creating favorable policy and environment
\item
  International community should provide assistance and special treatment for the development of infrastructure and trade facility and promotion
\item
  Ensure good governance and stop corruption by maintaining political stability
\item
  Full implementation of transit treaties and agreements
\item
  No opening of the cargoes having sealed containers
\item
  Easy and flexible border transit and tariff
\end{itemize}

\hypertarget{nepals-role-in-ldcs-and-lldcs}{%
\subsection{Nepal's role in LDCs and LLDCs}\label{nepals-role-in-ldcs-and-lldcs}}

\begin{itemize}
\tightlist
\item
  Since the beginning, Nepal has been involved in the issues and agendas related to the LDC and LLDC
\item
  During the period 2002-03, Nepal actively took part in activities and events related to LDCs in Brussels in 2001 and the ministerial conference of the LDCs.
\item
  Nepal has been kind and thankful to the developmental partners.
\item
  Nepal has been coordinating and maintaining healthy relationship within the nations of LDCs and LLDCs.
\item
  The eleventh Ministerial Meeting of the LDCs held on Nov 15, 2001 in New york, established a five member coordination Bureau of the group of LDC to strength coordination and promote greater participation and follow up of the Brussels Programme of Action. Nepal was elected as the chairman of the group on 29 September 2009 for the three years taking over from Bangladesh.
\item
  Earlier, Nepal also served as one of the four vice-chairperson in the LDCs coordination Bureau in New york from January 2002 to September 2007.
\item
  Nepal has contributed for facilitating coordination among LDCs by enhancing effective participation of the Group in various conferences and meetings.
\item
  Nepal has coordinated for developing unified approach of the group towards addressing emerging global developmental issues;
\item
  Nepal has consolidated positions and articulated the interests of LDCs in the negotiation of various conventions and resolutions
\item
  Nepal has also contributed providing sustained, sound and visible leadership of the group at the UN and other international forums and meetings.
\item
  Nepal has assumed the chairmanship of the LDCs and LLDCs group at a time when the world is facing multiple challenges of climate change, financial crisis, food crisis, energy shortages and so on.
\item
  Nepal has been advocating for the right of easy and free trade route for the LLDCs for easy movement of goods.
\item
  Nepal has been raising its voice for the LLDCs and its transit routes being the one of the LLDC countries.
\item
  Nepal has been shouting for the flexible transportation to reach up to the sea in the international forum and it has signed several bilateral and multilateral agreements for the right of the LLDCs.
\end{itemize}

\hypertarget{rights-of-land-locked-states}{%
\subsection{Rights of land locked states}\label{rights-of-land-locked-states}}

The UN Convention on the Law of Sea, 1982 provided rights for land locked states on the sea. More importantly, the convention provided them with the right of access to and from seas and freedom of transit. However, the law makes such rights subject to the agreements to be made by land-locked and transit states. This, in turn, depends on the prevailing relations between the concerned states. If they are not in a smooth relation, the transit states may not be willing to negotiate and thereby put impediments on the land-locked states' free transit. The political will and commitment of transit states highly conditioned the rights of land-locked states.

\begin{enumerate}
\def\labelenumi{\arabic{enumi}.}
\tightlist
\item
  The convention defines following terms: Land-locked state, transit state and means of transport (railway rolling stock, sea, lake, river craft and road vehicle; where local conditions so require porters and pack animals).
\item
  Landlocked states and transit states may, by agreement between them, include as means of transport pipelines and gas lines and means of transport other than those included in paragraph (above).
\end{enumerate}

\begin{itemize}
\tightlist
\item
  Article 125 - Right of access to and from the sea and freedom of transit
\item
  Article 126 - Exclusion of application of the most-favored-nation clause
\item
  Article 127 - Customs duties, taxes and other charges
\item
  Article 128 - Free zones and other customs facilities
\item
  Article 129 - Cooperation in the construction and improvement of means of transport
\item
  Article 130 - Measures to avoid or eliminate delays or other difficulties of a technical nature in traffic in transit
\item
  Article 131 - Equal treatment in maritime ports
\item
  Article 132 - Grant of greater transit facilities
\end{itemize}

\hypertarget{news-and-opinion-pieces}{%
\section{News and opinion pieces}\label{news-and-opinion-pieces}}

Nepali state is nurtured by both hard power of institutions and soft power of culture, language and identity. They have stitched Nepalis the world over. National authority, defined by the Westphalian state system, has now been contested by a number of state and non-state actors, institutions, forces, processes and human rights values. Those rooted in other societies often favour international jurisdiction to step in if national politics goes awry. The tax base of Nepal is scant to shore up state-centric order and shared idea of justice. Its knowledge, power, resource and legitimacy rest on the support of global community. The incongruity of Nepali state and its population settlement and surge of many regional and global issues require multi-level cooperation.

Technological innovation and road, rail and air connectivity of both neighbours have reduced the distance, cut the old geopolitical barriers and altered the earlier idea of border. This has exalted Nepal from colonially given buffer status into a more dynamic macro milieu reshaping its leaders' perception about opportunity from the startling outreach of neighbours' power as global game changers. Their neighbourhood priority hooked Nepal from the periphery to geopolitical focus.

The Eminent Persons' Group has, therefore, identified many debatable issues to settle and reset better ties. Apart from civilisational links, Nepal-India ties are shaped by security cooperation, roads, railroads, sea transit, power plants, air fields, economic development and human enterprises. Now, the growing Sino-Indian cooperation on border security, trade, investment and international relations has opened new geopolitical option for Nepal to ease the handicap of landlockedness. India is improving its image in Nepal, executing development promises and cultivating shared bond.

Likewise, Nepal's proximity to Tibet, an underbelly of China, where major powers are converged for its separatism makes Nepal no less strategically vital to its security. Nepal can be a conduit for spurring Sino-Indian collaboration if its leaders can build trust with each than they are with each other and resolve their security dilemma. Fierce contest for security suffocates cooperation. But the perception of power gap between India and China entailed Nepali leaders to gravitate toward the latter's incentives and initiatives on OBOR, Boao Forum for Asia, AIBB, SCO, foreign investment in strategic sectors, Trans-Himalayas Multi-Dimensional Connectivity Networks, adjust policy on regional and international institutions and promote trade and transit diversification. These are blinking in the global eye.

Both neighbours have shared interest to neutralise the soft power of Atlantic nations. They are providing succour to Nepal through their own NGOs without engaging in grassroots activism. China has suggested India to engage in partnership with Nepal avowing the validity of trilateralism. New geopolitics is based on partnership and burden sharing, not hegemonic or dominating. But, Nepali leaders must be aware of the outcome of choice they make in the context of the exhaustion of its institutional strength, fractiousness of leaders and highly remittance, aid, trade and foreign investment-dependent economy.

Diversification of dependence on external world turns Nepal's choice interdependent which means external resources from the global powers and institutions are needed to sustain internal life. Nepal is attractive spot for global community to host useful activities increasing its acceptability. Banding with the LDCs, mountainous, landlocked and small states has amplified its leverage beyond vicinity. But its leaders' view is that the geopolitical weight of neighbours is more important than the shifting global alliances, rise of neo-nationalism, crisis in multilateralism, climate change, etc.

Rising nationalism marked Nepal's shift from buffer to sovereignty. It can free its leaders from the snooze and rethink the utility of employing old concepts - semi-colony, sphere of influence, special relationship, equidistance, balancing or non-alignment. In the context of asymmetry of power between Nepal and its neighbours, self-neutralisation limits its freedom which is vital to apply judgment on international affairs. Nepal can attain middle path between certain shared aims and some differences with the neighbours and global regimes if Nepali leaders do not often invite alien forces for regime change to catapult oneself to power and esteem micro-management of nation's affairs. They need reflection on the nation's history of sacrifices of its heroes and builders for self-rule and gain aplomb, skill, ability and prudence to clip geopolitical issues stirring its vital interests.

Leaders need to rise up from their partisan frame which has converted Nepalis into a polarised mass to be easily maneuverer by external powers. How Nepali leaders view varied geographic spaces, population and their transactions is vital for policy while national angle can set a rational course. It needs to creatively use conference diplomacy, economic statecraft and adaptive foreign policy behaviour in a multi-speed world.

Nepali leaders need both guts and wisdom to absolve this nation from the Hobbesian Trap and punch geopolitical determinism, a concept which erodes the national freedom of manoeuvre and defies national interest-oriented policy. The harmony of whole gamut of foreign policy issues with any power cuts the liberty to defy, differ and create choice of development partners and make it inapt for national self-determination which has been historically prized by Nepalis.

National freedom ensures right course of action in regional and global politics. The ethical freedom rests on freedom to do good for the nation and uphold: responsibility, intrinsic power and political will to settle vital issues - Lipu Lekh, Susta, immigration, flood and cross-border crime control, removal of export barriers, management of trade deficits, proper utilisation of foreign aid on national priority areas, foreign investment in productive sectors, labour management, utilization of hydropower, balancing ties and fostering international cooperation for win-win game. Enlightened self-interest helps reduce many dependencies such as ideological dependence blunts national conscience, aid dependence weakens the integrity of policy, power dependence restricts legitimacy and blanket import of soft power reduces the source of national identity. New geopolitics can ignite a fresh hope for Nepal worthy to realise its power of place and potential.

Source: Dev Raj Dahal, \url{http://therisingnepal.org.np/news/25071}

\hypertarget{security-dilemmas-of-small-states}{%
\subsection{Security Dilemmas of Small States}\label{security-dilemmas-of-small-states}}

The state sovereignty of Nepal and Norway as small states largely rests on prudence, reason and skilful exercise of shifting room of manoeuver and apt realpolitik responses to old and emerging security dilemma created by an underlying, unstable multi-polar world. The gush of geopolitics marks the erosion of liberal international norms and the onset of new oligarchic competition for influence among great powers. It is creating a challenging condition for small states to adjust to a fleeting big picture of a new international system. It has also offered choices for their polities, non-states actors, markets and civil society.

Security dilemma

The ascent of China and global aspiration of India mark a tension with the USA if not its allies, on matters of trade and strategic issues of the Asia-Pacific. China and India are also increasingly present in the strategic Arctic Region. The US seeks to engage Nepal in the Indo-Pacific strategy. They demand to support Tibetan refugees' human rights, support the UN sanctions against North Korea to give up its nuclear ambition, and desist from active engagement in China's Belt and Road Initiative. This puts Nepal into a security dilemma. The aid incentives US is offering are helpful for cash-strapped Nepal's economic development, but it cannot prompt Nepal to abandon One China Policy. Nepali citizens feel at ease with China's connectivity projects and elevated ties with the US. Both the neighbours are calculating its implications to them.

After the Second World War Norway chose to become a member of the NATO Alliance while retaining the pragmatic policy of balanced reassurance and deterrence towards its mighty eastern neighbour (USSR and now Russia). Nepal has since the Rana regime until quite recently, pursued a somewhat circumscribed alliance-free foreign policy, largely skewed toward India and the West, but greatly more balanced ties with India and China. Now, Norway and Nepal as small states face a challenging spectre of old, emerging geopolitical dilemma as the Western liberal order crumbles and their territories attain greater strategic value by competing major powers. They are now challenged to define and refocus on national interests, and less on ideology to escape difficult security dilemmas in an exceedingly complicated world with a retreating hegemonic superpower.

Norway's High-North has emerged as a dynamic global hotspot for fast environmental change, increasing militarisation and presence of both Arctic rim-states and non-rim states, including a number of Asian states. The latter is arguably a result of geopolitical shifts, responding to environmental, technological and oceanic transportation changes of global scale. The Arctic Council has had an increasingly vital role in inter-state collaboration in the post-Cold War period. It addresses issues of environmental change, polar research and trade, but not hard-core security or military issues. Estimation of the new opportunities and risks as a result of trans-Eurasian integration, rapid environmental change and new rivalry between major powers, characterised by trade wars and escalating high-tech militarisation posing risks for interstate wars is an imperative.

Norway's Solberg-led neo-con coalition government is currently in its second term. The government has steadily shifted its foreign and defence policies away from the Cold-War era's balanced reassurance and deterrence policies, in favour of a more one-sided deterrence orientation. Analysts now observe a heavier and more regular military footprint of NATO in the High-North. Norway-US bilateral partnership is another pillar, strikingly under-communicated in national defence plans and by parliamentarians. Both inter-state pillars contribute a remilitarisation of Northern-Norway, notably not with a substantial build of Norway's own national land and naval forces. The High-North's military importance is back at Cold War level, but not with that era's ``simple order'' of only two rivalling powers.

The US military currently pays for upgrading of the sophisticated Globus-2 Radar at Vardø, a Norwegian island situated right on the enormous, increasingly ice-free Barents Sea. In terms of geographical distance from the capital, Vardø is an outpost located strategically close to the Norwegian-Russian border. Vardø's income-strapped local government has struggled to cope with a dramatic drop in the population and a surge in senior citizen. Closures of land-based marine processing industries and pessimistic youth migrating south are main causes.

Neither government funds nor a rising nature tourism industry as a birdwatchers' paradise, create alternative employment of sufficient scale. No wonder Vardø's politicians remain supportive of foreign presence. The glowingly white mega-antenna, perched on the city mountain, can be mistaken to be a temple-like art installation. Officially, it monitors rubble in space, but this is a clumsy distraction. It takes down and transmits onward extremely detailed satellite imagery of much of the Northern hemisphere. Judging experts and open US military sources, the radar is integrated within the US's ballistic missile umbrella. Recently, Russia's fighter jets conducted a full-scale mock attack on the radar, also observed in real time on screens at Norwegian and NATO intelligence quarters.

While the government spokespersons condemned the mock-attack, local reactions were more laidback, revealing Finnmark's special Second World War history. Hitler Germany applied a scorched earth strategy when retreating from East-Finnmark during momentous months in 1944. The Russians came to Finnmark as liberators. Their valiant military actions solidified friendly Norwegian-Russian relations in a border-region with a region with long-distance east-west trade, migration, cultural exchange and reindeer pastoralism. In terms of cross-border integration, the sub-region can be compared with Mustang's riveting geopolitical attention. In order to really grapple with what is what is most consequential in terms of a geopolitically framed border politics, be that in Vardø, Mustang or Rauswa, the influential ``centre versus periphery'' theory of Stein Rokkan, Norway's doyen of political science is inadequate.

Such striking differences between national politicians and border polities in terms of rhetoric and risk perceptions are not uncommon to Nepal either. Nepal's lifted ties with the USA and its strategic allies are considered by its decision makers a strategy to reduce neighbourhood determinism and exercise autonomy in foreign policy. There is, however, perceptual gap on both sides. Nepali leaders do not perceive China as a threat to be feared, in the way the US seeks to convince Nepalis. Instead they deem China's cooperation essential in order to reduce its undue dependence on India, knowing that it is a member of US-led QUAD. Until 1980s, India colluded with the West and Russia against China. Nepal's improved relations with the US are designed to convince the latter to have its own foreign policy lens on Nepal, rather than see it by the Indian lens. Reducing dependence on India, however, does not undermine shared soft power, open border, Gorkha recruitment and complex ties between Nepal and India. Nepal's interest to revise the Treaty of Peace and Friendship, transit diversification and good links to the outside world are based on its efforts to open menu of choice.

The growing Sino-India economic ties on many areas have made Nepal's strategic geography vital for the USA. Nepal may adopt cooperative policy in the Indo-Pacific region without riling the neighbours and accept the USA and China's cooperation in high politics of state security. But it will detest low politics of backing partisan or demographic interests that radicalizes the identity of otherness within the nation and flags the base of national security.

Neither India, nor China, not even Russia has any interest to defend democracy in Nepal, the shared political ideal of many Atlantic nations. For reason of its own national interest, Nepal will not return Tibetan refugees to China violating their human rights, but considers One-China Policy like India. It has committed not to harbour anti-China activities on its soil either through the use of intrinsic or acquired power. Nepalis believe that Sino-US competition in Nepal has constrained India to operate strategically in the sphere of assertive bilateralism and moderated its appetite for regime-changing behaviour. Nepal appreciates the US and China's support to conquer its state's weakness to create national security, stability and authority. The EU, the US and China have shown interest in building the capacity of Nepali Army, supply equipment and expressed readiness to help the proposed Defense University. The US-Nepal cooperation might not be about satellites and high-tech land-based installations as in the High-North. Yet military cooperation is pursued and complemented by civilian cooperation of geopolitical importance.

Despite the background condition of anarchy, international system mirrors global norms, values and laws and overlapping interests among nation-states. Small states, like Nepal and Norway, will not be free from the security dilemma as they find incongruity between domestic order created by the state authority and bigger systemic uncertainty in the world with countless inclinations of various actors entailing them to make security somewhat sacrosanct. Security dilemma imagines that the rise of new powers has generated perceived insecurity to relatively declining USA prompting it to adopt counter measures such as containment, adjustment and new alliance formation.

National interest

Historical memory of national insecurity is the factor behind the formulation of Nepal's new security policy, which aims to resolve security dilemma through rule-based system. But National Security Council needs a team of interdisciplinary experts so that the state-bearing institutions can identify the sources of threats, provide early warning and engage in mitigating them either through justice, law and order or coercion and widen the state's outreach in society. Communication to the public long-term national interests such as national security, stability, wellbeing, national identity and sovereign status is vital. This, like Norway, helps build trust among the state, citizens and experts and bridges the gap between state sovereignty and human security. For Nepal and Norway, the optimum strategy to resolve security dilemma depends on resorting to diversification, international law, organisation, deterrence, self-help or building a modicum of trust among the rival powers so that their peace and security are not endangered.

Source: Dev Raj Dahal, \url{http://therisingnepal.org.np/news/29922}

\hypertarget{foreign-aid}{%
\section{Foreign aid}\label{foreign-aid}}

\hypertarget{politics-of-foreign-aid}{%
\subsection{Politics of Foreign Aid}\label{politics-of-foreign-aid}}

Dev Raj Dahal

Politics of foreign aid is a powerful economic statecraft of global politics. Nepali citizens are socialised to believe that survival imperative and legitimacy of leaders rest on how much foreign aid they bring to the nation. Foreign aid covers various goals and activities and responds to public expectations. But it also reflects political influence in the public life from family planning, education, health, culture to governance. Foreign aid will continue to lead its development discourse if power elites succumb to aid-addictive mentality and behavior. Economic crisis, fall of certain donors, rise of nationalism in aid giving countries and newer alliance patterns have led to the sinking of grant and ascent of loan. Nepal needs to reorder its priority in the use of state funds beyond free market solution of the individual and societal problems.

Foreign aid marks a transition from the patron-client web in the colonial days, economic ``take-off'' in the cold war, transformation of economy, polity and society in the post-cold war years to crisis management and sustainable development now. Donors - bilateral, multilateral and INGOs and their allies NGOs and civil society -- define priorities in Nepal and shift aid conditions, setting its strategic, political and development tone. Those complaints to donors' ideological, strategic, commercial and humanitarian interests collect copious aid under paternal protection. Those in the shadow of great powers collect trickle-down. Ideally, aid emerges from the ethical obligation of the rich countries to help the poor to acquire self-dignity. Strings-free aid wires its sovereignty, not flag by external powers' policy goal. Yet, any talk on aid in Nepal evokes the images of ``weariness'' owing to a constant lack of desired progress, ``pervasiveness'' in all sectors of society and ``permanence'' of aid thus failing to lift domestic capital, technology and skills. It has frequently introduced new paradigm of development.

The cold war ethics of liberal aid sought to foil socialism by exalting state institutions. Socialist nations aimed to free Nepal from reliance on liberal order while others favoured self-help. Secluded long by high mountain chains in the north and dense malarial forests in the south, Nepal was suddenly vaulted into a global strategic game. It drew a stream of aid from irreconcilable worlds, executed class-mediating liberal ideals and adopted economic nationalism, import-substitution, export promotion, neutrality, regionalism and active role in conference diplomacy. The fall of socialism erased the cold war virtues of aiding the state and marked the age of human rights, democracy, market economy, civil society and gender equality. Foreign aid has stabilised the elites' survival as labour classes are stratified into multi-colour stripes. Aid's public awareness strategy, however, questioned the performance of polity and detonated political movements of critical mass aiming to reorder elites. The discontent with official aid due to its bureaucratic-political control over the citizens' lives widened the space for NGOs, civil society and citizens' groups. The liberal donors gave off bulk of their funds through these sub-elites' networks having outreach to the grassroots.

Their rivals fostered state sovereignty though multi-track actions. During the post-cold war phase, market sagacity of official aid did not function as Nepal's economic forces created partisan market of monopoly, cartel and syndicate strutting Nepal into a capricious future. Donors' support to partisan non-state actors also flagged the performance of local, provincial and federal state prompting the global community to review the utility of aid. The alignment of aid to rights-based discourse, national ownership, citizens' participation, transparency, accountability, equity and Nepal Development Forum are the outcome of the realisation of inseparability of human values in aid policy. But it scarcely satisfied the restless aspiration of Nepalis for public goods. The aid flow mustered a general consensus of donors and Nepali polity on social development, ecological protection, human rights, democracy, governance and peace. Proper use of aid can build internal capacity to spur surplus to cover balance of payment deficits and reduce foreign exchange and domestic savings gaps. New geopolitics of aid offers Nepal a choice in a poly-centric global order driven by science, transport and connectivity, not geography or ideology determinism.

The realism of foreign aid lies in strategic, political, economic and commercial interests of donors. Sometimes they are mixed with humanitarian ideals of easing the suffering of citizens and saving the lives of innocent from premature death. It has also contributed to mutually beneficial progress, such as adaptation to climate change, protection of global common, management of existential risks, etc. Nepal's aid practices have experienced many development notions presuming economic growth. Yet, the financial and technical aid which is meant to be its engine suffered from the assault of left and right critics refusing to admit its positive outcome. Left critics view that foreign aid smacks of charity and enhances the power of global capitalism, and its strategic ally comprador class, the enemy of nation's industrialisation. Right critics claim that foreign aid helps to expand bureaucracy, enlarges the state's authority and, therefore, creates obstacles to market economy and individual freedoms.

Bureaucratisation of aid has cut its ability to renew creative social institutions. In Nepal, the ``pervasiveness'' of aid reveals the loss of confidence of leaders and citizens in native knowledge, skill and resource owing to their uncritical habits to take external policy advice, power and legitimacy regardless of their utility. Nepal faces a gap between aid promises of donors and their supply. Similarly, skewed aid does not tally Nepal's aid requirements to resolve state weakness and market failures in meeting basic needs and renewing hope. Aid needs to assess the source of economic paralysis of the nation whether it is rooted in the security and political disharmony, corruption, abuse of resources, mismanagement, or bundle of political, legal, institutional and administrative flaws, contributing to a lack of desired progress. If deficits of political security and civil order are the primary problems, balance of payment support will hardly set Nepal in a sound progress curve. Specially, chronic dependency of Nepal has cut its haggling craft for the mobilisation of aid in the national interests and strap up domestic resources for rebuilding the nation.

If the effectiveness of foreign aid in Nepal is judged by its ability to expand high productive capacity to satisfy basic needs, there is a pause. The Development Cooperation Report of Finance Ministry unveils that donors have spent most of their money in Province 1 and Province 3 where human development index is relatively better. The backward Province 4 got the least. This means aid did not follow the criteria of human development index nor geographic distribution. Aid allocation has become elite-centric while the burden of debt for each citizen is Rs 24,276 which is equally shared. Nepal's foreign debt stands at Rs 436.6 billion while domestic debt is Rs 394.6 billion, unsettled account of government offices is Rs. 120 billion. The flaws of foreign aid can be corrected if elites do not lose integrity of public life, do its need assessment and remain clear about policies, aims and strategies in aid negotiations. It is helpful to know what works and what does not and internalize feedbacks into the aid policy. In the link between knowledge-based advanced society and old Nepali state, foreign aid can serve a mediating platform for mutual learning, not imposing the ideologies, institutions and values of dominant culture.

The extreme reliance of Nepalis on aid working along cross-cutting interests, sometimes centrifugal, makes them its victim beyond the coordinating ability of Finance and Foreign Ministries and Social Welfare Council. If market is regarded as the prime motor of progress and functions of aid underlie filling gaps - saving, investment and foreign exchange- , it is bound to nosh general deficiency, institutional weakness and democratic deficits. Empowering Nepali state and society can spur their equal integration in the global market and reap benefits cutting the density of grievances. Nepal faces huge migration of youth, brain drain and capital flight while real foreign direct investment is stunted. The choice of Nepali policy makers is neither de-linking from the self-regulating international system nor radicalism, but reforms which provide it option for economic diplomacy.

Foreign aid in Nepal's external opening can provide better choice if it does not lack vigour in fulfilling social needs, diversification owing to small size, initial stage of industrialisation and dependent on imports, investment, remittance, grants, loans, export of primary products, tourism, etc. The question is how the priority of Nepal's progress be built on common agenda of aid coordination and controlling duplication and inefficiencies. Aid priorities should focus on domestic resource mobilisation, inclusive economic growth, building knowledge relevant to policy on education and skills, health, institutional framework of regional cooperation and overcoming powerlessness of Nepalis. Beating the flaws of aid is vital to achieve its goals in the mutual interests and validate its ``permanence.'' A search for new ways in which foreign aid can be utilised in a productive manner, set new objectives, methodology and programmes for partners of development is underway.

At a time when foreign aid in Nepal continues to become a tuneful music to elites for the secret of their political success, it is right time to make it citizen-centric, ecological sensitive, non-commercial and non-strategic. To free Nepal from the trap of vicious cycle of poverty, debt and conflict grant components should get precedence. But international redistribution of public goods demands equal justice at home. The genesis of aid emerged as altruistic motive but suffered from general ``weariness'' inflicted by bribery, commission, leakage or capital flight abroad incubating a class of elites which thrives on aid and dies with the drying up of aid oxygen. Appropriate outlay of aid in the nation's priority areas is vital to have symbiotic ties between state and citizens where democracy can mediate the state's need for civic order and delivery of public goods and citizens' conscious impulse for freedom, education and dignity. Tied aid for the one-sided adjustment of Nepal to neoliberal economy appears to be based on military model of homogenisation. It exercises disturbing power over citizens causing political decay. If the reality of conditionalised aid produces democratic recession, Nepal should think about de-conditionalisation. Nepalis cherish the values of freedom which is the linchpin of democracy, human rights, justice and peace. Mutual determination of aid spawns positive outcome for functional necessity of Nepali democracy's chance of survival.

Source: Dev Raj Dahal, \url{http://therisingnepal.org.np/news/24030}

\hypertarget{bs-an-annus-horribilis}{%
\subsection{2072 BS: An Annus Horribilis}\label{bs-an-annus-horribilis}}

The year 2072 BS was an annus horribilis for the Nepalese people. It started with devastation that came in the form of nature- and human-induced disasters. Humans are helpless to tame the fury of nature but can demonstrate a sense of esprit de corps in the time of utter crisis and tragedy. This collective sentiment and cooperation was once again reified in the Nepali society when a 7.8 magnitude earthquake jolted it to the point of apocalypse in the second week of Baishakh (mid-April) last year, killing over 9,000 people and injuring many more. The tremor did not only convulse the Nepali geography but it also exposed political inertia and indecision. The key political actors were shackled to their entrenched positions, and the quake whipped them into effecting the 16-point deal that helped gain spurts in the constitution writing. The elected Constituent Assembly (CA), considered to be the most inclusive and democratic in South Asia, finally delivered the new constitution, defying the diktats and blusters of the southern neighbor. The new national charter turned 2072 into an annus mirabilis for the Nepalese living with the trauma of a high magnitude.

Light in the tunnel

The new statute was a beam of light into the dark tunnel of prolonged transition. The statute promulgation marked the prideful moment and assertion of national sovereignty. To our dismay, the southern neighbour saw red over its contents after the major parties spurned its hegemonic desire to take the ownership of the constitution. Pinched and infuriated, India came with a vengeance - it imposed an inhuman blockade on the Nepalese, who were still struggling to pick up the pieces following the massive destruction of lives, property and infrastructure. Agitating Madhesi parties brazenly collaborated with the Indian regime to intensify their border-centric protest only to rub salt into the wounds of the Nepalese. The illegitimate embargo triggered humanitarian crisis and caused colossal damage to the country's economy. One economist estimated that the blockade caused a loss of Rs 2,300 billion and pushed the country 50 years back.

It was the third Indian blockade against Nepal. India has taken the wicked punitive measure against the Himalayan nation whenever the latter dared to defend its territorial integrity and political sovereignty. However, it has been hoisted by its own petard. The Indian government made an embarrassing retreat in the face of dogged perseverance and surging nationalism of Nepalese, and international isolation. It imposed the blockade unofficially and lifted it without formal announcement. Despite this, Indian PM Narendra Modi has thrown the Nepalese for a loop. When push came to shove, Modi is flexing its diplomatic muscle to desecrate Nepal's constitution in the international forums on the trot. This David-Goliath tussle goes on until the truth prevails over blatant lies, rumours and propaganda.

What saved Nepal from collapsing when it was clobbered hard by the earthquake and blockade? It was perhaps the self-generated power of resilience rooted in the enlightened and altruistic tradition of Nepali civilisation. Resilience is the natural capacity of individual, community, society and the state to pull through from the adverse situation. Political scientist Dev Raj Dahal offers insights into the sources of resilience the Nepalese have showed in extremes. He writes: ``Nepal's earliest history of enlightenment inherited from classical treatises written during Vedic, Janak, and Buddhist eras, syncretic culture, Nepali language serving as lingua franca, heritage of tolerance of diversity, overlapping values, cosmopolitan attitude to asylum-seekers and history of independence infused the virtues of resilience.'' The temblor hit 14 districts hard but it spurred the cohesive elements of society across the country, transcending narrow interest and identity. This is a reason why the micro-minority communities of non-affected areas such as Kamalaris, Shikhs, Telis, Tharus, Muslims and Buddhists commiserated with the quake victims and provided relief to them. This altruistic feeling did not only knock the political parochialism sideways but also bore testimony to the fact that when political leadership wimps out, the community rises to the challenge.

Prime Minister KP Sharma Oli ascended to power hard on the heels of earthquake and blockade. His political odyssey from a Naxalite juvenile to executive head of government sounds a bit apocryphal and is full of exciting adventures. Only six months into office, Oli as a premier has not only added a feather to his cap but he has also set a shining example of patriotism for the nation. He unwaveringly stuck up for independence, dignity and right to self-determination. He stood up to the haughty hegemon and refused to give in. For all his shortcomings - his inability to accelerate the reconstruction campaign, and control the black-marketing and runaway inflation - Oli stands tall in the crowd of political mediocres.

Bizarre posture

His courage and conviction impels us to compare him with another charismatic democratic leader, BP Koirala, who had dealt with the giants of Asia - chairman Mao and Jawaharlal Nehru - with great sangfroid. BP never buckled under when Mao pressed him to accept the division of Sagarmatha (Mt Everest) between the two countries. BP insisted that entire Sagarmatha belongs to Nepal and said there was no question of splitting its sovereignty. Nehru had once asked BP about the agenda of his visit to China when the latter was in India in his capacity of Prime Minister. BP took umbrage at the uncalled for curiosity of Nehru and laid it on the line - `it is our internal affairs. India does not need to poke its nose into this matter.' But, to our chagrin, BP's party - the Nepali Congress - is shying away from calling a spade a spade. It was afraid of using the term `blockade' for the disruption of supplies during the cruel embargo. It ratted on the gentleman agreement with CPN-UML, throwing their robust partnership into disarray. In its latest bizarre posture, the mealy-mouthed NC was reluctant to criticise the EU-India joint statement that denigrated Nepal's democratic constitution. The resentful divorce between the NC and UML remained one of the nastier political episodes of 2072 BS. This has put the task of statute implementation in quandary. Much of the country's political course will largely be set by amity or enmity between these two big parties.

\begin{enumerate}
\def\labelenumi{\arabic{enumi}.}
\setcounter{enumi}{2}
\tightlist
\item
  Diplomacy has been defined as ``the management of relations between independent states by the process of negotiation''. Write an essay citing some major events of modern international diplomacy that have resolved conflict through the process of negotiation and have shaped the world in which we live.
\end{enumerate}

\hypertarget{diplomatic-correspondence}{%
\chapter{Diplomatic Correspondence}\label{diplomatic-correspondence}}

\hypertarget{correspondence}{%
\section{Correspondence}\label{correspondence}}

``Communication is to diplomacy as blood is to the human body. Whenever communication ceases, the body of international politics, the process of diplomacy, is dead, and the result is violent conflict or atrophy''.

-- Tran, 1987:8

The practice of preparing proper forms of diplomatic communications dates back to early periods of history when contacts among nations assumed great importance. Diplomatic correspondence is the art of communicating among states and putting into written form important information, discussions or arguments essential to the conduct of foreign relations.

\hypertarget{note}{%
\section{Note}\label{note}}

A written communication from a minister of foreign affairs to foreign diplomatic envoys or high foreign government officials and vice-versa. It is the most generally used form of correspondence between a sending state, Nepal government and the receiving state, a foreign government. The reply to an incoming not is in the same form as the note it is answering.

This is written either in the first person or third person. When they are written in the first person, they are signed and addressed and are known as \textbf{first-person notes}. They may be formal notes or informal notes.

When the notes are written in the thrid-person, they usually take one of the following forms:

\begin{enumerate}
\def\labelenumi{\arabic{enumi}.}
\tightlist
\item
  Note verbale
\end{enumerate}

A note verbale is a formal form of note and is so named by originally representing a formal record of information delivered orally. It is less formal than a note (also called a letter of protest) but more formal than an aide-memoire. A note verbale can also be referred to as a third person note (TPN). Notes verbale are written in the third person and printed on official letterhead; they are typically sealed with an embosser or, in some cases, a stamp. All notes verbale begin with a formal salutation, typically:

``The {[}name of sending state's{]} Embassy presents its compliments to the Ministry of Foreign Affairs and has the honor to invite their attention to the following matter.''

Notes verbales will also close with a formal valediction, typically:

``The Embassy avails itself of this opportunity of assuring the Ministry of its highest consideration.''

\begin{enumerate}
\def\labelenumi{\arabic{enumi}.}
\setcounter{enumi}{1}
\tightlist
\item
  Aide-memoire
\item
  Memorandum
\item
  Pro-memoria
\item
  Note collective
\end{enumerate}

These third-person notes are neither signed nor addressed.

\begin{itemize}
\tightlist
\item
  Note verbale is marked by the office seal and initialed in lower right-hand corner of the last page, by a duly authorized signing officer.
\end{itemize}

\hypertarget{first-person-note}{%
\subsection{First person note}\label{first-person-note}}

\begin{itemize}
\tightlist
\item
  An important phrase in writing a formal first-person note is to mention: ``I have the honor to'' which is usually placed near the beginning of the note. This may be suitably repeated in the body.
\item
  In some countries the aforementioned phrase is not used when the note is sent from an Ambassador to a Charge d' Affaires. In such cases following expressions are considered more apt:
  - ``I have the pleasure to''
  - ``I have the pleasure of''
  - ``I take the pleasure in''
\item
  ``I have the honor to'' is also not used in informal first-person note.
\item
  Generally, the word ``you'' is considered an appropriate pronoun in the body of an informal note, however for informal notes addressed to a minister of foreign affairs or a foreign ambassador, the phrase ``Your Excellency'' is not infrequent.
\end{itemize}

\textbf{Sample congratulatory message on the assumption of the office by the prime minister of India}

(Logo of Government of Nepal)
Kathmandu Nepal
Prime Minister

(Date to the right side)

His Excellency
Mr.~Narendra Modi
Prime Minister of India
New Delhi

Excellency,

On behalf of the Government and people of Nepal as well as on my own, I extend to your Excellency my heartly congratulations adn best wishes at your assumption of office of the Prime Minister of India.

I also take this opportunity to express my best wishes for the continued progress, prosperity and happiness of the people of India under your able premiership.

I am confident that the friendly relationsh, so happily existing between our two countries, will be further strengthened and expanded for the mutual benefit of our two country and people during the term of your office.

(to the right\ldots)
(Signature)
K.P Sharma (Oli)

\textbf{Sample formal first person note from the Minister of Foreign Affairs to the Minister of Foreign Affairs}

Excellency,

I have the honor to acknowledge the receipt of your letter of the 10th of June in which you extend an invitation for a Special Mission representing Nepal to attend the ceremony of inaguration of His Excellency (name of president), President-Elect of (country name), to be held in (place of program) on 19th August, 1999.

My government is deeply appreciative of Your Excellency's invitation and regret that circumstances do not permit us to avail ourselves of your courtesy by dispatching a Special Mission at the present.

I am to convey to Your Excellency and through you to His Excellency, the President-Elect our good wishes on the auspicious occassion of his inaguration.

Please accept, Excellency, the assurances of my highest consideration.

(to the right\ldots)
Minister of Foreign Affairs
H.E. Mr.~Pradeep Gyawali

\hypertarget{third-person-note}{%
\subsection{Third person note}\label{third-person-note}}

\textbf{Sample note seeking agreement of an Ambassador for the concurrent accreditation}

(Logo of Government of India)
Kathmandu, Nepal
Note number: 000/00

The Embassy of India in Kathmandu presents its compliments to the Protocol Department of the Ministry of Foreign Affairs of Nepal and has the honour to inform the Ministry that the Government of India decided to concurrently accredit, H.E. Mr.~Rakesh Sudh, Ambassador Extrordinary and Plenipotentiary of India to Nepal as non-resident Ambassador.

This embassy would highly appreciate it if the Government of Nepal could kindly grant agreement of the proposed accreditation of the Ambassador at its earliest convenience. A brief curriculum vitae and biography of H.E. Mr.~Anthony is attached herewith.

The embassy of India in Kathmandu avails itself of this opportunity to renew to the Protocol Department of the Ministry of Foreign Affairs of Nepal the assurances of its highest considerations.

(Date to the right side, without mention of the word ``Date'')
The Protocol Department
Ministry of Foreign Affairs
Kathmandu, Nepal

\textbf{Sample message of death of the former head of the government}

(Logo of Government of Nepal)
Embassy of Nepal,
New Delhi, India
Note number: 000/00

The embassy of Nepal in New Delhi presents its compliments to the Ministry of Foreign Affairs of India and to all the Diplomatic Missions and the International Oraganizations in New Delhi and has regretted to inform the sad passing away of H.E. Girija Prasad Koirala, former Prime Minister of Nepal on Feb 5, 2011. The funeral procession of the former Prime Minister was made at Pashupatinath ghat in Kathmandu, the capital city of Nepal, on February 6, 3011.

The embassy of Nepal in New Delhi avails itself of this opportunity to renew to the Ministry of Foreign Affairs of India and to all the Diplomatic Missions and the International Oraganizations in New Delhi the assurances of its highest consideration.

(Date to the right)

Ministry of External Affairs
All the Diplomatic Missions, and
International Organizations
New Delhi

\hypertarget{letter}{%
\section{Letter}\label{letter}}

\hypertarget{letter-of-credence-letter-of-accreditation}{%
\subsection{Letter of credence (Letter of accreditation)}\label{letter-of-credence-letter-of-accreditation}}

\textbf{Sample letter of accreditation of Ambassador of the United States of America to Bulgaria by the President of the USA}

(center\ldots)
Theodore Rosevelt,
President of the United States of America,

To His Royal Highness
Ferdinand,
Prince of Bulgaria.

Great and Good Friend,

Desiring to manifest the cordial friendship which this Government has for that of Your Royal Highness, I have made choice of John B. Jackson, one of our distinguished citiznes, as Diplomatic Agent of the United States of America to Bulgaria, and have charged him to conduct the affairs of his post in a manner to cultivate and maintain harmony and goodwill between the two countries.

Therefore, I request Your Royal Highness to receive him in that capacity and to give full credit to his representations on behalf of this Government, especially to the assurances which he will convey to you of the best wishes of this Government for the prosperity of Bulgaria.

May god have Your Royal Highness in His wise keeping.

\emph{Written at Washington, this 6th day of June, in the year 1903}

Your Good Friend,
(Signature of Rosevelt)

\hypertarget{bout-de-papier-or-speaking-notes}{%
\section{\texorpdfstring{\emph{Bout de papier} or Speaking notes}{Bout de papier or Speaking notes}}\label{bout-de-papier-or-speaking-notes}}

An unofficial and personal adjunct to oral communication frequently used is known as a \emph{bout de papier} or more frequently now as speaking notes. When an ambassador or a member of his staff makes an appointment to discuss some matter with a foreign representative, or in the ministry of foreign affairs, he sometimes has a few lines of written notes to assist his memory. He may decide to leave this piece of paper (which is much more informal than a Note) with the person to whom he has been speaking, in order to ensure that there will be no room for doubt regarding the main points which he has sought to make. The recipient generally finds this helpful and is grateful for it, especially if the conversation has been in a language in which one or other participant is only moderately proficient. The piece of paper is prepared in such a way that it bears no attribution. While it is thus both a personal courtesy and a practical convenience, it cannot be claimed by either side as possessing any official status.

\hypertarget{non-paper}{%
\section{Non-paper}\label{non-paper}}

A non-paper is even less official than a bout de papier. It is defined as an off-the-record or unofficial presentation of (government) policy and is used when a government does not wish formally to avow the position they have adopted or advanced in the paper but wishes to air certain ideas to see how they will be received. It can also be used to good effect in multilateral diplomacy where one government undertakes to draft a non-paper to identify the greatest amount of common ground between the other participants in a negotiation while not accepting to be bound by or taking any responsibility for the positions set out in the non-paper.

Of greater currency in the modern world is the use of letters between heads of state and government sent electronically and delivered by hand of the ambassador or a member of his mission to the private office of the president or prime minister. This may or may not involve a brief covering letter from the ambassador which might be along the following lines:

Your Excellency

I have been asked by my President to convey to you the enclosed message. I am of course available to convey any response or to attempt to clarify any points in the message.

Courtesy ending

\hypertarget{demarche}{%
\section{Demarche}\label{demarche}}

A less formal way of making diplomatic representations or protests is called a démarche. The following from the US Department of State handbook sets out the typical procedures and purposes of a démarche.

A U.S. Government démarche to a foreign government is made on the basis of `front-channel cable' instructions from the Department of State. Although the content of a given démarche may originate in another U.S. Government agency, only the State Department may also instruct a post to deliver the démarche. Unless specifically authorized by the State Department, posts should not act on instructions transmitted directly from another post, or from another agency, whether by cable or other means (e.g., e-mail, FAX, or phone).

Any State Department officer or other official under the authority of the chief of mission can make a démarche. Unless the Department provides specific instructions as to rank (for example: `the Ambassador should call on the Foreign Minister'), the post has discretion to determine who should make the presentation and which official(s) in the host government should receive it.

Preparation of the Démarche

Démarche instruction cables from the Department should include the following elements:

\begin{enumerate}
\def\labelenumi{\arabic{enumi}.}
\item
  Objective: The objective is a clear statement of the purpose of the démarche, and of what the U.S. Government hopes to achieve.
\item
  Arguments: This section outlines how the Department proposes to make an effective case for its views. It should include a rationale for the U.S. Government's position, supporting arguments, likely counter-arguments, and suggested rebuttals.
\item
  Background: The background should spell out pitfalls; particular sensitivities of other bureaus, departments, or agencies; and any other special considerations.
\item
  Suggested taking points: Suggested talking points should be clear, conversational, and logically organized. Unless there are compelling reasons to require verbatim delivery, the démarche instruction cable should make it clear that post may use its discretion and local knowledge to structure and deliver the message in the most effective way. (`Embassy may draw from the following points in making this presentation to appropriate host government officials.')
\item
  Written material: Use this section to provide instructions on any written material to be left with the host government official(s). Such material could take the form of an aide-mémoire, a letter, or a `non-paper' that provides a written version of the verbal presentation (i.e., the talking points as delivered). Unless otherwise instructed, post should normally provide an aide-mémoire or non-paper at the conclusion of a démarche. Classified aide-memoire or non-paper must be appropriately marked and caveated as to the countries authorized for receipt, i.e., Rel. UK (Releasable to UK).
\end{enumerate}

\textbf{Delivery and Follow-up Action}

Upon receipt of démarche instructions from the Department, post should make every effort to deliver the démarche to the appropriate foreign government official(s) as soon as possible.

After delivering the démarche, post should report to the Department via front-channel cable. The reporting cable should include the instruction cable as a reference, but it need not repeat the talking points transmitted in that cable. It should provide the name and title of the person(s) to whom the démarche was made, and record that official's response to the presentation. As appropriate, the reporting cable should also describe any specific follow-up action needed by post, Department, or the foreign government.

\hypertarget{press-release}{%
\section{Press release}\label{press-release}}

\hypertarget{the-right-honourable-president-participates-in-women-in-power}{%
\subsection{The Right Honourable President Participates in `Women in Power'}\label{the-right-honourable-president-participates-in-women-in-power}}

Mardi, 12 Mars 2019
Presenter: Lok B. Chhetri
Lieu: New York

The Right Honourable President Mrs.~Bidya Devi Bhandari participated this morning in the High-Level event on `Women in Power' organized by the President of General Assembly (PGA) during the 63rd session of the Commission of the Status of Women (CSW-63). Speaking in the first among three high-level round tables entitled `How Women Leaders Change the World', the Rt. Hon.~President shared with the General Assembly Nepal's initiatives, achievements and lessons learned in the field of gender equality.

The President highlighted that the constitutional and legal arrangements have increased participation of women in leadership positions in Nepal. She also shared the inspiring experience of her journey from a grass-root woman activist in a remote village to the first President of Nepal after the promulgation of the new Constitution.

In the global context, the President stressed that `discrimination against women is a social construct, not a natural condition'. She underscored the need to guarantee women's rights in constitutions and laws, adding that participation on governance structures is essential to bring impact on socio-economic sectors.

Other speakers sharing their perspectives at the roundtable together with the Rt. Hon.~President were the Presidents of Lithuania and Trinidad and Tobago; Vice-President of Colombia; and High Representative of the Union for Foreign Affairs and Security Policy of the European Union.

Also in the morning, the President held a meeting with H.E. António Guterres, the Secretary-General of the United Nations at his office. During the meeting, the two sides exchanged views on various issues, including multilateralism, Climate Change, SDGs, peacekeeping, and post-earthquake reconstruction. The President appreciated the Secretary-General for his reform agenda across all three pillars of UN in general and for maintaining gender parity in all UN bodies, in particular. The Secretary-General recalled his first visit to Nepal in 1978, and appreciated Nepal for playing an active role at the UN, including through contributions in peacekeeping.

In the afternoon, the President met with H.E. María Fernanda Espinosa Garcés, the PGA, and appreciated her for taking leadership on important issues, including through organizing the event, `Women in Power'. The PGA thanked the President for her presence and for sharing enlightening experience.

Later in the afternoon, the President held separate bilateral meetings with the President of Estonia H.E. Kersti Kaljulaid and the President of Croatia H.E. Kolinda Grabar-Kitarović. During the meeting with the Estonian counterpart, the two sides discussed pressing global issues such as Climate Change, sustainable development, multilateralism, among others. Similarly, during her meeting with the President of Croatia, matters of mutual interests, including women in political leadership, were discussed. In both the meetings, the leaders expressed satisfaction over the warm bilateral relations while exchanging views on ways to enhance them.

The Rt. Hon.~President also met with H.E. Mary Robinson, the president of The Elders and the former President of Ireland. During the meeting, the two leaders shared their views on climate justice, women empowerment and leadership, among others.

The Rt. Hon.~President attended a dinner hosted in her honour by Ambassador Amrit Bahadur Rai, Permanent Representative of Nepal to the United Nations at his official residence. Earlier today, the Rt. Hon.~President attended a high-level luncheon with `Young Women Leaders' jointly hosted by the PGA and the Permanent Mission of Qatar to the United Nations.

\hypertarget{state-visit-of-the-president-of-peoples-republic-of-china-to-nepal}{%
\subsection{State visit of the President of People's Republic of China to Nepal}\label{state-visit-of-the-president-of-peoples-republic-of-china-to-nepal}}

At the friendly invitation of President of Nepal, Right Honourable Mrs.~Bidya Devi Bhandari, the President of the People's Republic of China His Excellency Mr.~Xi Jinping is paying a state visit to Nepal on 12 and 13 October 2019.

During the visit, President H.E. Mr.~Xi Jinping will meet with President Rt.Hon. Mrs.~Bidya Devi Bhandari. The Right Honourable President of Nepal will host a Banquet in honour of President H.E. Xi Jinping and the Chinese delegation.

President H.E. Mr.~Xi Jinping will hold delegation level official talks with Right Honourable Prime Minister Mr.~K P Sharma Oli. Following the talks, both leaders will witness the signing of bilateral agreements and memoranda of understanding.

Senior leaders of Nepal will call on President H.E. Xi Jinping during his sojourn in Kathmandu.

Ministry of Foreign Affairs
Singhadurbar
Kathmandu, Nepal
09 October 2019

\hypertarget{rt.-hon.-presidents-visit-to-japan}{%
\subsection{Rt. Hon.~President's visit to Japan}\label{rt.-hon.-presidents-visit-to-japan}}

The Rt. Hon.~President had a meeting with the President of India H.E. Shri Ram Nath Kovind in a warm and cordial environment in Tokyo this morning. The two Presidents expressed satisfaction over the excellent state of age-old, multifaceted bilateral relations.

The two Presidents appreciated the timely completion of the cross-border petroleum pipeline and expressed hope that other projects would also be implemented in an expeditious manner.

The Presidents underlined the importance of high-level exchanges on a regular basis in order to further advance bilateral relations. In this context, Rt. Hon.~President Mrs.~Bidya Devi Bhandari renewed her invitation to the President of India H.E. Shri Ram Nath Kovind to visit Nepal. Expressing his sincere thanks for the invitation, President H.E. Kovind expressed his willingness to visit Nepal and said that he would pay the visit at a convenient time to both sides.

Minister for Home Affairs Hon.~Mr.~Ram Bahadur Thapa `Badal', Foreign Secretary Mr.~Shanker Das Bairagi, Ambassador of Nepal to Japan H.E. Mrs.~Prativa Rana, Secretary at the Office of the President Dr.~Hari Paudel, Principal Private Secretary to the Rt. Hon.~President Dr.~Bhesh Raj Adhikari and other high-ranking officials of the Government of Nepal participated in the meeting. From Indian side, H.E. Shri Sanjay Kothari, Secretary to the President, H.E. Shri Vijay Gokhale, Foreign Secretary, H.E. Shri Sanjay Kumar Verma, Ambassador of India to Japan and other high ranking officials of the Government of India were present during the meeting.

Embassy of Nepal
Tokyo
22 October 2019

\hypertarget{address-by-rt.-hon.-pm-at-the-sdg-moment}{%
\subsection{Address by Rt. Hon.~PM at the SDG Moment}\label{address-by-rt.-hon.-pm-at-the-sdg-moment}}

Date: (Vendredi) 18 Septembre 2020

Press Release

Prime Minister Rt. Hon.~Mr.~K P Sharma Oli addressed this morning the SDG Moment 2020, a high-level event convened by UN Secretary-General on the margins of the 75th session of the UN General Assembly.

In his address, the Prime Minister expressed concerns that the pandemic threatened the hard-earned development gains particularly in the countries such as LDCs, LLDCs and SIDs. The freezing economy, shrinking revenue, increasing public expenditure, rising poverty and unemployment have severe impact on these countries' efforts to realize the SDGs, he added.

He stated that it was not a moment to shift the goalposts further and that the SDGs must continue to serve as the compass of clarity for global action.

The Prime Minister further said that the sustainable development was at the center of Nepal's development vision and it underpinned the national aspiration of `Prosperous Nepal, Happy Nepali'. He also touched upon Nepal's achievements in key areas including poverty reduction, food security, education, gender equality, basic sanitation, and energy access.

He underlined the need to move beyond rhetoric and accelerate action at all levels in order to recover and rebuild better, and to ensure that no one is left behind.

Nepal was one of the five countries from the Asia Pacific region to be featured in the SDG Moment 2020. There were 22 Heads of State/Government from around the world representing different regional groups, a number of SDG advocates and partners, and senior UN officials, among others.

Permanent Mission of Nepal to the United Nations

New York

18 September 2020

\hypertarget{th-session-of-the-united-nations-general-assembly-unga}{%
\subsection{75th session of the United Nations General Assembly (UNGA)}\label{th-session-of-the-united-nations-general-assembly-unga}}

Prime Minister Rt. Hon.~Mr.~K P Sharma Oli is leading Nepali delegation to the 75th session of the United Nations General Assembly (UNGA) being held in New York.

The Prime Minister is scheduled to address the General Debate of the UNGA virtually on 25 September 2020.

The theme of the general debate of the UNGA this year has been set as ``The Future We Want, the United Nations We Need: Reaffirming Our Collective Commitment to Multilateralism -- Confronting COVID-19 through Effective Multilateral Action''.

High-level week of the UNGA is beginning from 17 September 2020.

The Prime Minister will also address the high-level meeting to commemorate the 75th anniversary of the United Nations on 21 September 2020. He will also address the 2020 SDG Moment being held on 18 September 2020; and the Biodiversity Summit on 30 September 2020.

Similarly, Minister for Foreign Affairs, Hon.~Mr.~Pradeep Kumar Gyawali will address virtually the high-level meeting to commemorate and promote the International Day for the Total Elimination of Nuclear Weapons on 2 October 2020. He will also address the Ministerial Meetings of LDCs and LLDCs, among others.

On 24 September, Nepal will host the virtual meeting of the SAARC Council of Ministers traditionally held coinciding with the UNGA.

Other members of the delegation include Minister for Foreign Affairs Hon.~Pradeep Kumar Gyawali and senior officials of the Government of Nepal from Kathmandu. Ambassador of Nepal to the United States of America H.E. Mr.~Arjun Kumar Karki, Permanent Representative of Nepal to the United Nations in New York H.E. Mr.~Amrit Bahadur Rai and other officials at the Permanent Mission of Nepal will participate at the UNGA virtually/physically as appropriate.

Ministry of Foreign Affairs
Singh Durbar
Kathmandu

17 September 2020

\hypertarget{president-of-the-republic-of-serbia-aleksandar-vuux10diux107-conveyed-today-a-letter-of-congratulations-to-the-president-of-the-united-states-of-america-donald-trump-for-the-independence-day}{%
\subsection{President of the Republic of Serbia Aleksandar Vučić conveyed today a Letter of Congratulations to the President of the United States of America Donald Trump for the Independence Day}\label{president-of-the-republic-of-serbia-aleksandar-vuux10diux107-conveyed-today-a-letter-of-congratulations-to-the-president-of-the-united-states-of-america-donald-trump-for-the-independence-day}}

Your Excellency, Dear President,

Please accept my sincere congratulations for the Independence Day - national holiday of the United States of America, as well as my best wishes for further progress of your country and wellbeing of all its citizens.

Every fourth of July citizens of the USA celebrate universal values, written in the Declaration of Independence, adopted 243 years ago. Those are, like you said, key values that make the American nation and whose strength inspires and gives hope, but they are likewise important for all those striving to liberty, equality and prosperity.

In previous period we celebrated important anniversaries from our common history, remembering bright examples of our alliance in great historical moments. At the end of the World War I, the American President Woodrow Wilson said that ``the principles Serbia had heroically fought for and due to which it had suffered in that war are identical to those the USA advocate.''

I am convinced that now we also have the opportunity to bring back the relations of our countries, through the development of political dialogue and strengthening of economic and every other cooperation, to the level of true partnership, the way they used to be and the way they should be. I hope that a desire for richer and better Serbian-American bonds is mutual and that we will remind ourselves more in the future of what connects us.

I avail myself of this opportunity to thank you for the support that the United States of America provide us on our European path, and particularly for supporting the efforts Serbia has been making in promoting the politics of peace and cooperation in the Western Balkans, as well as in finding compromise solution within the dialogue with Priština.

With a hope that you will visit us soon, please accept Your Excellency my cordial regards and the assurances of my highest consideration", states the congratulations letter of President Vučić.

Belgrade,
4 July 2019

\hypertarget{working-visit-of-the-president-of-benin-his-excellency-dr.-boni-yayi-to-the-internatioal-court-of-justice.}{%
\subsection{Working Visit of the President of Benin, His Excellency Dr.~Boni Yayi to the Internatioal Court of Justice.}\label{working-visit-of-the-president-of-benin-his-excellency-dr.-boni-yayi-to-the-internatioal-court-of-justice.}}

The HAGUE,
21 January, 2009

His Excellency Dr.~Boni Yayi, President of the Republic of Benin, paid a working visit today to the seat of the International Court of Justice, the principal judicial organ of the United Nations, at the Peace Palace in The Hague.

President Boni Yayi was greeted upon his arrival by the President of the Court, Judge Rosalyn Higgins, and by the Registrar, Mr Philippe Couvreur. President Higgins introduced him to Judges Raymond Ranjeva, Ronny Abraham and Mohamed Bennouna, who sat in the Chamber which dealt with the case concerning the Frontier Dispute (Benin/Niger). The Registrar then introduced him to senior officials of the Registry and to two Beninese nationals working for the Court.

His Excellency Dr.~Boni Yayi then participated in a meeting with the Members of the Court present on the activities of the ICJ. At the end of this meeting, the President of Benin signed the visitors' book and exchanged gifts with President Higgins, before finishing his visit in the Great Hall of Justice.

The President of Benin was accompanied by a delegation consisting of several Beninese Government Ministers, a number of counsellors, the Ambassador of Benin to the Netherlands, the Ambassador of the Netherlads to Benin, and other Beninese diplomats.

Information Department

\hypertarget{president-golitsyn-paying-courtesy-call-on-secretary-general-ban-ki-moon}{%
\subsection{President Golitsyn paying courtesy call on secretary-general Ban Ki-Moon}\label{president-golitsyn-paying-courtesy-call-on-secretary-general-ban-ki-moon}}

International Tribunal for the Law of the Sea (ITLOS)
18 December, 2014

Judge Vladimir Golitsyn, President of the Tribunal, paid a courtesy call on the Secretary-General of the United Nations, Mr Ban Ki-Moon, at his office in New York on 17 December, 2014. He was accompanied by the Registrar of the Tribunal, Mr Philippe Gautier.

President Golitsyn briefed the Secretary-General on the Judicial work of the Tribunal and highlighted the special role played by the Tribunal in the dispute settelement system establihsed under the United Nations Convention on the Law of the Sea. He also underlined the importance of providing states with information on the mechanisms for the settlement of disputes under the convention and referred in this respect to the regional workshops organized by the Tribunal.

The President extended an invitation to the Secretary-General to visit the Tribunal at a time convenient for him.

\hypertarget{un-security-council-mission-visit-to-burundi}{%
\subsection{UN Security Council mission visit to Burundi}\label{un-security-council-mission-visit-to-burundi}}

11 March 2015

A United Nations Security Council mission is expected to visit Burundi on 13 March 2015 to hold discussions with the country's Head of State and key stakeholders on peace and stability issues, including the ongoing electoral process. The visit to Burundi is part of a regional trip by the UN Security Council which also includes stops in the Central African Republic (CAR) and in Addis Ababa, Ethiopia, to meet with the African Union.

Council members will meet with Burundian President Pierre Nkurunziza as well as with other members of the Government, officials of the National Electoral commission's (CENI), representatives of political parties and politically affiliated youth groups as well as diplomats, representatives of the UN Electoral Observation Mission in Burundi (MENUB) and the UN country team. The delegation is also expected to speak to the press at the end of its one-day visit.

The visit to Burundi occurs as the country is preparing to hold five elections: legislatives and communal (26 May); presidential (first round on 23 June and possible run-off on 27 July); senatorial (17 July); and local elections (hills and districts councils - 24 August).

Following a request by the Government of Burundi to the Secretary-General, the United Nations established an electoral observation mission in Burundi headed by his Special Envoy to Burundi, M. Cassam Uteem, to follow and report on the electoral process the country. The United Nations Electoral Observation Mission in Burundi (MENUB) officially started its operations in the country on 1 January 2015, as stated in SC resolution 2137 (2014). France - who holds the rotating presidency of the Security Council for the month of March and the United States are co-leading the visit to Burundi.

\hypertarget{credential-presentation}{%
\subsection{Credential Presentation}\label{credential-presentation}}

H.E. Prof Dr.~Chop Lal Bhusal Ambassador of Nepal to the People's Republic of Bangladesh has presented his letter of credence to Honourable Abdul Hamid, President of the People's Republic of Bangladesh at a ceremony organized today at Bangabandhu Bhabhawan on Tuesday, 18 July 2017 today. Mrs.~Puspha Bhusal, spouse of the Ambassador was accompanied during the occassion.

National Anthem of both the nations were played during the occassion. After presentation of the credentials, Ambassador Prof.~Dr Bhusal had a courtesy meeting with Honourable President of Bangladesh at his office. During the meeting, matters related to bilateral relations including connectivity, trade and transit between the two countries were discussed. At the meeting Honourable President expressed his willingness to visit Nepal during his tenure.

Foreign Secretary, Military Secretary to the Preseident, Secretary to the President, Chief of Protocol, high dignitaries and officials from the President's Office and from the Ministry of Foreign Affairs including media persons were present during the occassion. Mr.~Dhan Bahadur Oli, Deputy Chief of Mission and Minister Counsellor and Brig. Gen Sagar K.C., Military Attache of the Embassy were also present on the occasion.

18 July, 2017
Embassy of Nepal, Dhaka

  \bibliography{bibliography/bib.bib}

\end{document}
